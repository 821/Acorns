\chapter{序}

晉人樂曠多奇情,故其言語文章別是一色,世說可覩已。說為晉作。及于漢、魏者,其餘耳。雖典雅不如左氏國語,馳騖不如諸國策,而清微簡遠,居然玄勝。概舉如衛虎渡江,安石教兒,機鋒似沈滑稽,又冷類入人夢思,有味有情,嚥之愈多,嚼之不見。蓋于時諸公剸以一言半句為終身之目,未若後來人士俛焉下筆,始定名價。臨川善述,更自高簡有法。反正之評,戾實之載,豈不或有?亦當頌之,使與諸書並行也。晚後淺俗,柰解人正不可得。嗚呼!人言江左清談遺事,槃槃一老出其游戲餘力,尚足辦此百萬之敵,茲非談之宗歟?抑吾取其文,而非論其人也。丙戌長夏,病思無聊,因手校家本精滅。其長註,間疏其滯義。明年以授梓,迺五月既望梓成。耘廬劉應登自書其端,是為序。

\chapter{德行第一}

陳仲舉言為士則,行為世範,登車攬轡,有澄清天下之志。【《汝南先賢傳》曰:「陳蕃字仲舉,汝南平輿人。有室荒蕪不掃除,曰:『大丈夫當為國家掃天下。』值漢桓之末,閹豎用事,外戚豪橫。及拜太傅,與大將軍竇武謀誅宦官,反為所害。」】為豫章太守,至,【《海內先賢傳》曰:「蕃為尚書,以忠正忤貴戚,不得在臺,遷豫章太守。」】便問徐孺子所在,欲先看之。【謝承《後漢書》曰:「徐穉字孺子,豫章南昌人。清妙高跱,超世絕俗。前後爲諸公所辟,雖不就,及其死,萬里赴弔。常豫炙雞⼀隻,以綿漬酒中,㬥乾以裹雞,徑到所赴冢隧外,以水漬綿,斗米飯,白茅爲藉,以雞置前。酹酒畢,留謁即去,不見喪主。」】主簿白:「群情欲府君先入廨。」陳曰:「武王式商容之閭,席不暇煖;【許叔重曰:「商容,殷之賢人,老子師也。」車上跽曰式。」】吾之禮賢,有何不可!」【袁宏《漢紀》曰:「蕃在豫章,為稚獨設一榻,去則懸之,見禮如此。」】

周子居常云:「吾時月不見黃叔度,則鄙吝之心已復生矣。」【子居別見。《典略》曰:「黃憲字叔度,汝南慎陽人。時論者咸云『顏子復生』。而族出孤鄙,父爲牛醫。潁川荀季和執憲手曰:『足下吾師範也。』後見袁奉高曰:『卿國有顏子,寧知之乎?』奉高曰:『卿見吾叔度邪?』戴良少所服下,見憲則自降簿,悵然若有所失。母問:『汝何不樂乎?復從牛醫兒所來邪?』良曰:『瞻之在前,忽焉在後,所謂良之師也。』」】

郭林宗至汝南造袁奉高,【《續漢書》曰:「郭泰字林宗,太原介休人。泰少孤,年二十,行學至成皋屈伯彥精廬。乏食,衣不蓋形,而處約味道,不改其樂。李元禮⼀見稱之曰:『吾見士多矣,無如林宗者也。』及卒,蔡伯喈爲作碑,曰:『吾爲人作銘,未嘗不有慚容,唯爲郭有道碑頌無愧耳。』初,以有道君子徵。泰曰:『吾觀乾象人事,天之所廢不可支也。』遂辭以疾。」《汝南先賢傳》曰:「袁宏字奉高,慎陽人。友黃叔度於童齒,薦陳仲舉於家巷。辟太尉掾,卒。」】車不停軌,鸞不輟軛。詣黃叔度,乃彌日信宿。人問其故,林宗曰:「叔度汪汪如萬頃之陂。澄之不清,擾之不濁,其器深廣,難測量也。」【《泰別傳》曰:「薛恭祖問之,泰曰:『奉高之器,譬諸汎濫,雖清易挹也。』」】

李元禮風格秀整,高自標持,欲以天下名教是非為己任。【薛瑩《後漢書》曰:「李膺字元禮,潁川襄城人。抗志清妙,有文武雋才。遷司隸校尉,爲黨事自殺。」】後進之士,有升其堂者,皆以為登龍門。【《三秦記》曰:「龍門,⼀名河津,去長安九百里。水懸絕,龜魚之屬莫能上,上則化爲龍矣。」】

李元禮嘗嘆荀淑、鍾皓【《先賢行狀曰》:「荀淑,字季和,潁川潁陰人也。所拔韋褐芻牧之中,執案刀筆之吏,皆爲英彥。舉方正,補朗陵侯相,所在流化。鍾皓字季明,潁川長社人。父、祖至德著名。皓高風承世,除林慮長,不之官。人位不足,天爵有餘。」】曰:「荀君清識難尚,鍾君至德可師。」【《海內先賢傳》曰:「潁川先輩,爲海內所師者:定陵陳穉叔、潁陰荀淑、長社鍾皓。少府李膺宗此三君,常言:『荀君清識難尚,陳鍾至德可師。』」】

陳太丘詣荀朗陵,貧儉無僕役。【陳寔字仲弓,潁川許昌人。爲聞喜令、太丘長,風化宣流。】乃使元方將車,【《先賢行狀》曰:「陳紀字元方,寔長子也。至德絕俗,與寔高名並著,而弟諶又配之。每宰府辟召,羔雁成群,世號『三君』,百城皆圖畫。」】季方持杖後從。長文尚小,載箸車中。既至,荀使叔慈應門,慈明行酒,餘六龍下食。【張璠《漢紀》曰:「淑有八子:儉、鯤、靖、燾、汪、爽、肅、敷。淑居西豪里,縣令苑康曰,『昔高陽氏有才子八人』,遂署其里爲高陽里。時人號曰八龍。」】文若亦小,坐箸厀前。于時太史奏:「真人東行。」【檀道鸞《續晉陽秋》曰:「陳仲弓從諸子姪造荀父子,于時德星聚,太史奏:『五百里賢人聚。』」】

客有問陳季方【《海內先賢傳》曰:「陳諶字季方,寔少子也。才識博達,司空掾公車徵,不就。」】:「足下家君太丘,有何功德,而荷天下重名?」季方曰:「吾家君譬如桂樹生泰山之阿,上有萬仞之高,下有不測之深;上為甘露所霑,下為淵泉所潤。當斯之時,桂樹焉知泰山之高,淵泉之深,不知有功德與無也!」

陳元方子長文有英才,【《魏書》曰:「陳群字長文,祖寔,嘗謂宗人曰:『此兒必興吾宗。』及長,有識度。其所善,皆父黨。」】與季方子孝先,【《陳氏譜》曰:「諶子忠,字孝先。州辟不就。」】各論其父功德,爭之不能決,咨於太丘。太丘曰:「元方難為兄,季方難為弟。」【⼀作「元方難爲弟,季方難爲兄」。】

荀巨伯遠看友人疾,【《荀氏家傳》曰:「巨伯,漢桓帝時人也。亦出潁川,未詳其始末。」】值胡賊攻郡;友人語巨伯曰:「吾今死矣,子可去!」巨伯曰:「遠來相視,子令吾去;敗義以求生,豈荀巨伯所行邪?」賊既至,謂巨伯曰:「大軍至,一郡盡空。汝何男子,而敢獨止?」巨伯曰:「友人有疾,不忍委之,寧以我身代友人命。」賊相謂曰:「我輩無義之人,而入有義之國!」遂班軍而還,一郡並獲全。

華歆遇子弟甚整,雖閒室之內,嚴若朝典。【《魏志》曰:「歆字子魚,平原高唐人。」《魏略》曰:「靈帝時與北海邴原、管寧俱遊學相善,時號三人爲⼀龍。謂歆爲龍頭,原爲龍腹,寧爲龍尾。】陳元方兄弟恣柔愛之道,而二門之裏,兩不失雍熙之軌焉。

管寧、華歆共園中鋤菜,【《傅子》曰:「寧字幼安,北海朱虛人,齊相管仲之後也。」】見地有片金,管揮鋤與瓦石不異,華捉而擲去之。又嘗同席讀書,有乘軒冕過門者,寧讀如故,歆廢書出看。寧割席分坐曰:「子非吾友也。」【《魏略》曰:「寧少恬靜,常笑邴原、華子魚有仕宦意。及歆爲司徒,上書讓寧。寧聞之笑曰:『子魚本欲作老吏,故榮之耳。』」】

王朗每以識度推華歆。【《魏書》曰:「朗字景興,東海郯人,魏司徒。」】歆蜡日,【《禮記》曰:「天子大蜡八,伊耆氏始爲蜡。蜡,索也。歲十二月,合聚萬物而索饗之。」《五經要義》曰:「三代名臘:夏曰嘉平,殷曰清祀,周曰大蜡,總謂之臘。」晉博士張亮議曰:「蜡者,合聚百物索饗之,歲終休老息民也。臘者,祭宗廟五祀。《傳》曰:『臘,接也。』祭則新故交接也。秦、漢以來,臘之明日爲祝歲,古之遺語也。」】嘗集子姪燕飲,王亦學之。有人向張華說此事,張曰:「王之學華,皆是形骸之外,去之所以更遠。」【王隱《晉書》曰:「張華字茂先,范陽人也。累遷司空,而爲趙王倫所害。」】

華歆、王朗俱乘船避難,有一人欲依附,歆輒難之。朗曰:「幸尚寬,何為不可?」後賊追至,王欲舍所攜人。歆曰:「本所以疑,正為此耳。既已納其自託,寧可以急相棄邪?」遂攜拯如初。世以此定華、王之優劣。【《華嶠譜敘》曰:「歆爲下邽令,漢室方亂,乃與同志士鄭太等六七人避世。自武關出,道遇⼀丈夫獨行,願得與俱。皆哀許之。歆獨曰:『不可。今在危險中,禍福患害,義猶⼀也。今無故受之,不知其義,若有進退,可中棄乎?』衆不忍,卒與俱行。此丈夫中道墮井,皆欲棄之。歆乃曰:『已與俱矣,棄之不義。』卒共還,出之而後別。」】

王祥事後母朱夫人甚謹,【《晉諸公贊》曰:「祥字休徵,琅邪臨沂人。」祥世家曰:「祥父融,娶高平薛氏,生祥。繼室以廬江朱氏,生覽。」《晉陽秋》曰:「後母數譖祥,屢以非理使祥,弟覽輒與祥俱。又虐使祥婦,覽妻亦趨而共之。母患,方盛寒冰凍,母欲生魚,祥解衣將剖冰求之,會有處冰小解,魚出。」蕭廣濟《孝子傳》曰:「祥後母忽欲黃雀炙,祥念難卒致。須臾,有數十黃雀飛入其幕。母之所須,必自奔走,無不得焉。其誠至如此。」】家有一李樹,結子殊好,母恆使守之。時風雨忽至,祥抱樹而泣。【】祥嘗在別床眠,母自往闇斫之。值祥私起,空斫得被。既還,知母憾之不已,因跪前請死。母於是感悟,愛之如己子。【虞預《晉書》曰:「祥以後母故,陵遲不仕。年向六十,刺史呂虔檄爲別駕,時人歌之曰:『海沂之康,寔賴王祥;邦國不空,別駕之功!』累遷太保。」】

晉文王稱阮嗣宗至慎,每與之言,言皆玄遠,未嘗臧否人物。【《魏書》曰:「文王諱昭,字子上,宣帝第⼆子也。」《魏氏春秋》曰:「阮籍字嗣宗,陳留尉氏人,阮瑀子也。宏達不羈,不拘禮俗。兗州刺史王昶請與相見,終日不得與言。昶愧嘆之,自以不能測也。口不論事,自然高邁。」李康《家誡》曰:「昔嘗侍坐於先帝,時有三長史俱見,臨辭出,上曰:『爲官長當清、當慎、當勤,修此三者,何患不治乎?』並受詔。上顧謂吾等曰:『必不得已而去,於斯三者何先?』或對曰『清固爲本』。復問吾,吾對曰:『清慎之道,相須而成,必不得已,慎乃爲大。』上曰:『辦言得之矣,可舉近世能慎者誰乎?』吾乃舉故太尉荀景倩、尚書董仲達、僕射王公仲。上曰:『此諸人者,溫恭朝夕,執事有恪,亦各其慎也。然天下之至慎者,其唯阮嗣宗乎!每與之言,言及玄遠,而未嘗評論時事,臧否人物,可謂至慎乎!』」】

王戎云:「與嵇康居二十年,未嘗見其喜慍之色。」【《康集敘》曰:「康字叔夜,譙國銍人。」王隱《晉書》曰:「嵇本姓溪,其先避怨徙上虞,移譙國銍縣。以出自會稽,取國⼀支,音同本奚焉。」虞預《晉書》曰:「銍有嵇山,家於其側,因氏焉。」《康別傳》曰:「康性含垢藏瑕,愛惡不爭於懷,喜怒不寄於顏。所知王濬沖在襄城,面數百,未嘗見其疾聲朱顏。此亦方中之美範,人倫之勝業也。」《文章敘錄》曰:「康以魏長樂亭主婿遷郎中,拜中散大夫。」】

王戎、和嶠同時遭大喪,俱以孝稱。王雞骨支床,和哭泣備禮。【《晉諸公贊》曰:「戎字濬沖,琅邪人,太保祥宗族也。文皇帝輔政,鍾會薦之曰:『裴楷清通,王戎簡要。』即俱辟爲掾。晉踐祚,累遷荊州刺史,以平吳功,封安豐侯。」《晉陽秋》曰:「戎爲豫州刺史,遭母憂,性至孝,不拘禮制,飲酒食肉,或觀棊弈,而容貌毀悴,杖而後起。時汝南和嶠,亦名士也,以禮法自持。處大憂,量米而食,然憔悴哀毀,不逮戎也。」】武帝謂劉仲雄曰:「卿數省王、和不?聞和哀苦過禮,使人憂之。」【王隱《晉書》曰:「劉毅字仲雄,東萊掖人,漢城陽景王後也。亮直清方,見有不善,必評論之。王公大人,望風憚之。僑居陽平,太守杜恕致爲功曹,沙汰郡吏三百餘人。三魏僉曰:『但聞劉功曹,不聞杜府君。』累遷尚書、司隸校尉。」】仲雄曰:「和嶠雖備禮,神氣不損;王戎雖不備禮,而哀毀骨立。臣以和嶠生孝,王戎死孝。陛下不應憂嶠,而應憂戎。」【《晉陽秋》曰:「世祖及時談以此貴戎也。」】

梁王、趙王,【朱鳳《晉書》曰:「宣帝張夫人生梁孝王彤,字子徽,位至太宰。桓夫人生趙王倫,字子彝,位至相國。」】國之近屬,貴重當時。裴令公【《晉諸公贊》曰:「裴楷字叔則,河東聞喜人,司空秀之從弟也。父徽,冀州刺史,有雋識。楷特精易義。累遷河南尹、中書令,卒。」】歲請二國租錢數百萬,以恤中表之貧者。或譏之曰:「何以乞物行惠?」裴曰:「損有餘,補不足,天之道也。」【《名士傳》曰:「楷行己取與,任心而動,毀譽雖至,處之晏然,皆此類。」】

王戎云:「太保居在正始中,不在能言之流。及與之言,理中清遠,將無以德掩其言!」【《晉陽秋》曰:「祥少有美德行。」】

王安豐遭艱,至性過人。裴令往弔之,曰:「若使一慟果能傷人,濬沖必不免滅性之譏。」【《曲禮》曰:「居喪之禮,毀瘠不形,視聽不衰,不勝喪,乃比於不慈不孝。」孝經曰:「毀不滅性,聖人之教也。」】

王戎父渾有令名,官至涼州刺史。【《世語》曰:「渾字長源,有才望。歷尚書、涼州刺史。」】渾薨,所歷九郡義故,懷其德惠,相率致賻數百萬,戎悉不受。【虞預《晉書》曰:「戎由是顯名。」】

劉道真嘗為徒,【《晉百官名》曰:「劉寶字道真,高平人。」徒,罪役作者。】扶風王駿【虞預《晉書》曰:「駿字子臧,宣帝第十七子,好學至孝。」《晉諸公贊》曰:「駿八歲爲散騎常侍,侍魏齊王講。晉受禪,封扶風王,鎮關中,爲政最美。薨,贈武王。西土思之,但見其碑贊者,皆拜之而泣。其遺愛如此。」】以五百疋布贖之,既而用為從事中郎。當時以為美事。

王平子、胡毋彥國諸人,皆以任放為達,或有裸體者。【《晉諸公贊》曰:「王澄,字平子,有達識,荊州刺史。」《永嘉流人名》曰:「胡毋輔之字彥國,泰山奉高人,湘州刺史。」王隱《晉書》曰:「魏末阮籍,嗜酒荒放,露頭散髮,裸袒箕踞。其後貴游子弟阮瞻、王澄、謝鯤、胡毋輔之之徒,皆祖述於籍,謂得大道之本。故去巾幘,脫衣服,露醜惡,同禽獸。甚者名之爲通,次者名之爲達也。」】樂廣笑曰:「名教中自有樂地,何為乃爾也!」

郗公值永嘉喪亂,在鄉里甚窮餒,鄉人以公名德,傳共飴之。公常攜兄子邁及外生周翼二小兒往食。鄉人曰:「各自饑困,以君之賢,欲共濟君耳;恐不能兼有所存。」公於是獨往食,輒含飯著兩頰邊,還吐與二兒。後並得存,同過江。【《郗鑒別傳》曰:「鑒字道徽,高平金鄉人。漢御史大夫郗慮後也。少有體正,耽思經籍,以儒雅著名。永嘉末,天下大亂,饑饉相望,冠帶以下,皆割己之資供鑒。元皇徵爲領軍,遷司空、太尉。」《中興書》曰:「鑒兄子邁,字思遠,有幹世才略。累遷少府、中護軍。」】郗公亡,翼為剡縣解職歸,席苫於公靈床頭,心喪終三年。【《周氏譜》曰:「翼字子卿,陳郡人。祖奕,上谷太守。父優,車騎咨議。歷剡令、青州刺史、少府卿,六十四而卒。」】

顧榮在洛陽,嘗應人請,覺行炙人有欲炙之色,因輟己施焉;同坐嗤之。榮曰:「豈有終日執之,而不知其味者乎?」後遭亂渡江,每經危急,常有一人左右己;問其所以,乃受炙人也。【《文士傳》曰:「榮字彥先,吳郡人。其先越王句踐之支庶,封於顧邑,子孫遂氏焉,世爲吳著姓。大父雍,吳丞相。父穆,宜都太守。榮少朗雋機警,⾵穎標徹,歷廷尉正。曾在省與同僚共飲,見行炙者有異於常僕,乃割炙以噉之。後趙王倫篡位,其子爲中領軍,逼用榮爲長史。及倫誅,榮亦被執。凡受戮等輩十有餘人。或有救榮者,問其故。曰:『某省中受炙臣也。』榮乃悟而嘆曰:『⼀餐之惠,恩今不忘,古人豈虛言哉!』」】

祖光祿少孤貧,性至孝,常自為母炊爨作食。【王隱《晉書》曰:「祖納字士言,范陽遒人,九世孝廉。納諸母三兄,最治行操,能清言,歷太子中庶子,廷尉卿。避地江南,溫嶠薦爲光祿大夫。」】王平北聞其佳名,以兩婢餉之,因取為中郎。【《王乂別傳》曰:「乂字叔元,琅邪臨沂人。時蜀新平,⼆將作亂,文帝西之長安,乃徵爲相國司馬,遷大尚書、出督幽州諸軍事、平北將軍。」】有人戲之者曰:「奴價倍婢。」祖云:「百里奚亦何必輕於五羖之皮邪?」【《楚國先賢傳》曰:「百里奚字凡伯,楚國人。少仕於虞,爲大夫。晉欲假道於虞以伐虢,諫而不聽,奚乃去之。」《說苑》曰:「秦穆公使賈人載鹽於虞,諸賈人買百里奚以五羊皮。穆公觀鹽,怪其牛肥,問其故,對曰:『飲食以時,使之不暴,是以肥也。』公令有司沐浴衣冠之。公孫支讓其卿位,號曰五羖大夫。」】

周鎮罷臨川郡還都,未及上住,泊青溪渚。【《永嘉流人名》曰:「鎮字康時,陳留尉氏人也。祖父和,故安令。父震,司空長史。」《中興書》曰:「鎮清約寡欲,所在有異績。」】王丞相往看之。【《丞相別傳》曰:「王導字茂弘,琅邪人。祖覽,以德行稱。父裁,侍御史。導少知名,家世貧約,恬暢樂道,未嘗以風塵經懷也。」】時夏月,暴雨卒至,舫至狹小,而又大漏,殆無復坐處。王曰:「胡威之清,何以過此!」即啟用為吳興郡。【《晉陽秋》曰:「胡威字伯虎,淮南人。父質以忠清顯。質爲荊州,威自京師往省之。及告歸,質賜威絹⼀匹。威跪曰:『大人清高,於何得此?』質曰:『是吾奉祿之餘,故以爲汝糧耳。』威受而去。每至客舍,自放驢取樵爨炊。食畢,復隨旅進道。質帳下都督陰齎糧要之,因與爲伴。每事相助經營之,又進少飯,威疑之,密誘問之,乃知都督也。謝而遣之。後以白質,質杖都督⼀百,除其吏名。父子清慎如此。及威爲徐州,世祖賜見,與論邊事及平生。帝嘆其父清,因謂威曰:『卿清孰與父?』對曰:『臣清不如也。』帝曰:『何以爲勝汝邪?』對曰:『臣父清畏人知,臣清畏人不知,是以不如遠矣。』」】

鄧攸始避難,於道中棄己子,全弟子。【《晉陽秋》曰:「攸字伯道,平陽襄陵人。七歲喪父母及祖父母,持重九年。性清慎平簡。」鄧粲《晉紀》曰:「永嘉中,攸爲石勒所獲,召見,立幕下與語,說之,坐而飯焉。攸車所止,與胡人鄰轂,胡人失火燒車營,勒吏案問胡,胡誣攸。攸度不可與爭,乃曰:『向爲老姥作粥,失火延逸,罪應萬死。』勒知遣之。所誣胡厚德攸,遺其驢馬,護送令得逸。」王隱《晉書》曰:「攸以路遠,斫壞車,以牛馬負妻子以叛,賊又掠其牛馬。攸語妻曰:『吾弟早亡,唯有遺民。今當步走,儋兩兒盡死,不如棄己兒,抱遺民。吾後猶當有兒。』婦從之。」《中興書》曰:「攸棄兒於草中,兒啼呼追之,至莫復及。攸明日繫兒於樹而去,遂渡江,至尚書左僕射,卒。弟子綏服攸齊衰三年。」】既過江,取一妾,甚寵愛。歷年後訊其所由,妾具說,是北人遭亂,憶父母姓名,乃攸之甥也。攸素有德業,言行無玷,聞之哀恨終身,遂不復畜妾。

王長豫為人謹順,事親盡色養之孝。【《中興書》曰:「王悅字長豫,丞相導長子也。仕至中書侍郎。】丞相見長豫輒喜,見敬豫輒嗔。【《文字志》曰:「王恬字敬豫,導次子也。少卓犖不羈,疾學尚武,不爲導所重。至中軍將軍。多才藝,善隸書,與濟陽江虨以善奕聞。」】長豫與丞相語,恒以慎密為端。丞相還臺,及行,未嘗不送至車後。恒與曹夫人併當箱篋。長豫亡後,丞相還臺,登車後,哭至臺門。曹夫人作簏,封而不忍開。【《王氏譜曰:「導娶彭城曹韶女,名淑。」】

桓常侍聞人道深公者,輒曰:「此公既有宿名,加先達知稱,又與先人至交,不宜說之。」【《桓彝別傳》曰:「彝字茂倫,譙國龍亢人,漢五更桓榮十世孫也。父穎,有高名。彝少孤,識鑒明朗,避亂渡江,累遷散騎常侍。」僧法深,不知其俗姓,蓋衣冠之胤也。道徽高扇,譽播山東,爲中州劉公弟子。值永嘉亂,投跡楊土,居止京邑,內持法綱,外允具瞻,弘道之法師也。以業慈清淨,而不耐風塵,考室剡縣東⼆百里𡵙山中,同遊十餘人,高棲浩然。支道林宗其風範,與高麗道人書,稱其德行。年七十有九,終於山中也。」】

庾公乘馬有的盧,【《晉陽秋》曰:「庾亮字元規,潁川鄢陵人,明穆皇后長兄也。淵雅有德量,時人方之夏侯太初、陳長文之倫。侍從父琛,避地會稽,端拱嶷然,郡人嚴憚之。覲接之者,數人而已。累遷征西大將軍、荊州刺史。」《伯樂相馬經》曰:「馬白頟入口至齒者,名曰榆雁,⼀名的盧。奴乘客死,主乘棄市,凶馬也。」】或語令賣去。【《語林》曰:「殷浩勸公賣馬。」】庾云:「賣之必有買者,即復害其主。寧可不安己而移於他人哉?昔孫叔敖殺兩頭蛇以為後人,古之美談,【賈誼《新書》曰:「孫叔敖爲兒時,出道上,見兩頭蛇,殺而埋之。歸見其母,泣。問其故,對曰:『夫見兩頭蛇者,必死。今出見之,故爾。』母曰:『蛇今安在?』對曰:『恐後人見,殺而埋之矣。』母曰:『夫有陰德,必有陽報,爾無憂也。』後遂興於楚朝。及長,爲楚令尹。」】效之,不亦達乎!」

阮光祿在剡,曾有好車,借者無不皆給。有人葬母,意欲借而不敢言。阮後聞之,嘆曰:「吾有車而使人不敢借,何以車為?」遂焚之。【《阮光祿別傳》曰:「裕字思曠,陳留尉氏人。祖略,齊國內史。父顗,汝南太守。裕淹通有理識,累遷侍中。以疾築室會稽剡山。徵金紫光祿大夫,不就。年六十⼀卒。」】

謝奕作剡令,【《中興書》曰:「謝奕字無奕,陳郡陽夏人。祖衡,太子少傅。父裒,吏部尚書。奕少有器鑒,辟太尉掾、剡令,累遷豫州刺史。」】有一老翁犯法,謝以醇酒罰之,乃至過醉,而猶未已。太傅時年七、八歲,箸青布絝,在兄膝邊坐,諫曰:「阿兄,老翁可念,何可作此?」奕於是改容曰:「阿奴欲放去邪?」遂遣之。

謝太傅絕重褚公,常稱:「褚季野雖不言,而四時之氣亦備。」【《文字志》曰:「謝安字安石,奕弟也。世有學行,安弘粹通遠,溫雅融暢。桓彝見其四歲時,稱之曰:『此兒⾵神秀徹,當繼蹤王東海。』善行書。累遷太保、錄尚書事。贈太傅。」《晉陽秋》曰:「褚裒字季野,河南陽翟人。祖䂮,安東將軍。父治,武昌太守。裒少有簡貴之風,沖默之稱。累遷江、兗⼆州刺史。贈侍中、太傅。」】

劉尹在郡,臨終綿惙,聞閣下祠神鼓舞。正色曰:「莫得淫祀!」【《劉尹別傳》曰:「惔字真⾧,沛國蕭人也。漢氏之後。真⾧有雅裁,雖蓽門陋巷,晏如也。歷司徒左長史、侍中、丹陽尹。爲政務鎮靜信誠,⾵塵不能移也。」】外請殺車中牛祭神。真長答曰:「丘之禱久矣,勿復為煩。」【《包氏論語》曰:「禱,請也。」孔安國曰:「孔子素行合於神明,故曰:『丘之禱久矣。』」】

謝公夫人教兒,問太傅:「那得初不見君教兒?」答曰:「我常自教兒。」【《謝氏譜》曰:「安娶沛國劉耽女。」按:太尉劉子真,清潔有志操,行己以禮。而二子不才,並黷貨致罪。子真坐免官。客曰:「子奚不訓導之?」子真曰:「吾之行事,是其耳目所聞見,而不放效,豈嚴訓所變邪?」安石之旨,同子真之意也。】

晉簡文為撫軍時,【《續晉陽秋》曰:「帝諱昱,字道萬,中宗少子也。仁聞有智度。穆帝幼沖,以撫軍輔政。大司馬桓溫廢海西公而立帝,在位三年而崩。」】所坐床上塵不聽拂,見鼠行跡,視以為佳。有參軍見鼠白日行,以手板批殺之,撫軍意色不悅,門下起彈;教曰:「鼠被害,尚不能忘懷;今復以鼠損人,無乃不可乎?」

范宣年八歲,後園挑菜,誤傷指,大啼。人問:「痛邪?」答曰:「非為痛也;但身體髮膚,不敢毀傷,是以啼耳。」【《宣別傳》曰:「宣字子宣,陳留人,漢萊蕪長范丹後也。年十歲,能誦詩書。兒童時,手傷改容,家人以其年幼,皆異之。徵太學博士、散騎常侍,⼀無所就。年五十四卒。」】宣潔行廉約,韓豫章遺絹百匹,不受;【《中興書》曰:「宣家至貧,罕交人事。豫章太守殷羡見宣茅茨不完,欲爲改室,宣固辭。羡愛之,以宣貧,加年饑疾疫,厚餉給之,宣又不受。」《續晉陽秋》曰:「韓伯字康伯,潁川人。好學,善言理。歷豫章太守、領軍將軍。」】減五十匹,復不受。如是減半,遂至一匹,既終不受。韓後與范同載,就車中裂二丈與范,云:「人寧可使婦無褌邪?」范笑而受之。

王子敬病篤,道家上章應首過,問子敬「由來有何異同得失?」子敬云:「不覺有餘事,唯憶與郗家離婚。」【《王氏譜》曰:「獻之娶高平郗曇女,名道茂,後離婚。」《獻之別傳》曰:「祖父曠,淮南太守。父羲之,右將軍。咸寧中,詔尚餘姚公主,遷中書令,卒。」】

殷仲堪既為荊州,值水儉,食常五盌,外無餘肴。飯粒脫落盤席閒,輒拾以噉之。雖欲率物,亦緣其性真素。每語子弟云:「勿以我受任方州,云我豁平昔時意。今吾處之不易。貧者士之常,焉得登枝而捐其本?爾曹其存之!」【《晉安帝紀》曰:「仲堪,陳郡人,太常融孫也。車騎將軍謝玄請爲長史,孝武說之,俄爲黃門侍郎。自殺袁悅之後,上深爲晏駕後計,故先出王恭爲北蕃。荊州刺史王忱死,乃中詔用仲堪代焉。」】

初桓南郡、楊廣共說殷荊州,宜奪殷覬南蠻以自樹。【《桓玄別傳》曰:「玄字敬道,譙國龍亢人,大司馬溫少子也。幼童中,溫甚愛之。臨終命以爲嗣。年七歲,襲封南郡公,拜太子洗馬、義興太守。不得志,少時去職,歸其國。與荊州刺史殷仲堪素舊,情好甚隆。」周祗《隆安記》曰:「廣字德度,弘農人,楊震後也。」《晉安帝紀》曰:「覬字伯道,陳郡人。由中書郎出爲南蠻校尉。覬亦以率易才悟著稱,與從弟仲堪俱知名。」《中興書》曰:「初,仲堪欲起兵,密邀覬,覬不同。楊廣與弟佺期勸殺覬,仲堪不許。」】覬亦即曉其旨,嘗因行散,率爾去下舍,便不復還。內外無預知者,意色蕭然,遠同鬬生之無慍。時論以此多之。【《春秋傳》曰:「楚令尹子文,鬬氏也。」《論語》曰:「令尹子文,三仕爲令尹,無喜色;三已之,無慍色。」】

王僕射在江州,為殷、桓所逐,奔竄豫章,存亡未測。【徐廣《晉紀》曰:「王愉字茂和,太原晉陽人,安北將軍坦之次子也。以輔國司馬,出爲江州刺史。愉始至鎮,而桓玄、楊佺期舉兵以應王恭,乘流奄至,愉無防,惶遽奔臨川,爲玄所得。玄篡位,遷尚書左僕射。」】王綏在都,既憂戚在貌,居處飲食,每事有降。時人謂為試守孝子。【《中興書》曰:「綏字彥猷,愉子也。少有令譽。自王渾至坦之,六世盛德,綏又知名,于時冠冕,莫與爲比。位至中書令、荊州刺史。桓玄敗後,與父愉謀反,伏誅。」】

桓南郡【玄也。】既破殷荊州,收殷將佐十許人,咨議羅企生亦在焉。【《玄別傳》曰:「玄克荊州,殺殷道護及仲堪參軍羅企生、鮑季禮,皆仲堪所親仗也。」】桓素待企生厚,將有所戮,先遣人語云:「若謝我,當釋罪。」企生荅曰:「為殷荊州吏,今荊州奔亡,存亡未判,我何顏謝桓公?」【《中興書》曰:「企生字宗伯,豫章人。殷仲堪初請爲府功曹,桓玄來攻,轉咨議參軍。仲堪多疑少決,企生深憂之,謂其弟遵生曰:『殷侯仁而無斷,事必無成。成敗天也,吾當死生以之。』及仲堪走,文武並無送者,唯企生從焉。路經家門,遵生紿之曰:『作如此分別,何可不執手?』企生回馬授手,遵生便牽下之,謂曰:『家有老母,將欲何行?』企生揮泣曰:『今日之事,我必死之。汝等奉養,不失子道,⼀門之內,有忠與孝,亦復何恨!』遵生抱之愈急,仲堪於路待之。企生遙呼曰:『今日死生是同,願少見待!』仲堪見其無脫埋,策⾺而去。俄而玄至,人士悉詣玄,企生獨不往而營理仲堪家。或謂曰:『玄性猜急,未能取卿誠節,若遂不詣,禍必至矣!』企生正色曰:『我殷侯吏,見遇以國士,不能共殄醜逆,致此奔敗,何面目就桓求生乎?』玄聞,怒而收之。謂曰:『相遇如此,何以見負?』企生曰:『使君口血未乾,而生此奸計,自傷力劣,不能翦定凶逆,我死恨晚爾!』玄遂斬之。時年三十有七,衆咸悼之。」】既出市,桓又遣人問欲何言?【王隱《晉書》曰:「紹字延祖,譙國銍人。父康有奇才雋辯。紹十歲而孤,事母孝謹,累遷散騎常侍。惠帝敗於蕩陰,百官左右皆奔散,唯紹儼然端冕,以身衞帝。兵交御輦,飛箭雨集,遂以見害也。」】答曰:「昔晉文王殺嵇康,而嵇紹為晉忠臣。從公乞一弟以養老母。」桓亦如言宥之。桓先曾以一羔裘與企生母胡,胡時在豫章,企生問至,即日焚裘。

王恭從會稽還,【周祗《隆安記》曰:「恭字孝伯,太原晉陽人。祖父濛,司徒左長史,風流標望。父蘊,鎮軍將軍,亦得世譽。」《恭別傳》曰:「恭清廉貴峻,志存格正。起家著作郎,歷丹陽尹、中書令。出爲五州都督、前將軍、青兗⼆州刺史。」】王大看之,【王忱,小字佛大。《晉安帝紀》曰:「忱字元達,北平將軍坦之第四子也。甚得名於當世,與族子恭少相善,齊聲見稱。仕至荊州刺史。」】見其坐六尺簟,因語恭:「卿東來,故應有此物,可以一領及我?」恭無言。大去後,即舉所坐者送之。既無餘席,便坐薦上。後大聞之,甚驚,曰:「吾本謂卿多,故求耳。」對曰:「丈人不悉恭;恭作人無長物。」

吳郡陳遺,【未詳。】家至孝,母好食鐺底焦飯。遺作郡主簿,恒裝一囊,每煮食,輒貯錄焦飯,歸以遺母。後值孫恩賊出吳郡,【《晉安帝紀》曰:「孫恩⼀名靈秀,琅邪人。叔父泰,事五斗米道,以謀反誅。恩逸逃於海上,聚衆十萬人,攻沒郡縣。後爲臨海太守辛昺斬首送之。」】袁府君【山松別見。】即日便征,遺已聚斂得數斗焦飯,未展歸家,遂帶以從軍。戰於滬瀆,敗。軍人潰散,逃走山澤,皆多餓死;遺獨以焦飯得活。時人以為純孝之報也。

孔僕射為孝武侍中,豫蒙眷接烈宗山陵。孔時為太常,形素羸瘦,著重服,竟日涕泗流漣,見者以為真孝子。【《續晉陽秋》曰:「孔安國字安國,會稽山陰人,⾞騎愉第六子也。少⽽孤貧,能善樹節,以儒素見稱。歷侍中、太常、尚書,遷左僕射、特進,卒。」】

吳道助、附子兄弟,居在丹陽郡。後遭母童夫人艱,【道助,坦之小字。附子,隱之小字也。《吳氏譜》曰:「坦之字處靖,濮陽人。仕至西中郎將功曹。父堅,取東苑童儈女,名秦姬。」】朝夕哭臨。及思至,賓客弔省,號踊哀絕,路人為之落淚。韓康伯時為丹陽尹,母殷在郡,每聞二吳之哭,輒為悽惻。語康伯曰:「汝若為選官,當好料理此人。」康伯亦甚相知。韓後果為吏部尚書。大吳不免哀制,小吳遂大貴達。【鄭緝《孝子傳》曰:「隱之字處默,少有孝行,遭母喪,哀毀過禮。時與太常韓康伯鄰居,康伯母揚州刺史殷浩之妹,聰明婦人也。隱之每哭,康伯母輒輟事流涕,悲不自勝,終其喪如此。謂康伯曰:『汝後若居銓衡,當用此輩人。』後康伯爲吏部尚書,乃進用之。」《晉安帝紀》曰:「隱之既有至性,加以廉潔,奉祿頒九族,冬月無被。桓玄欲革嶺南之弊,以爲廣州刺史。去州二十里有貪泉,世傳飲之者其心無厭。隱之乃至水上,酌而飲之,因賦詩曰:『石門有貪泉,⼀歃重千金。試使夷、齊飲,終當不易⼼。』爲盧循所攻,還京師。歷尚書、領軍將軍。」《晉中興書》曰:「舊云:往廣州,飲貪泉,失廉潔之性。吳隱之爲刺史,自酌貪泉飲之,題石門爲詩云云。」】



\chapter{言語第二}

邊文禮見袁奉高,失次序。奉高曰:「昔堯聘許由,面無怍色;先生何為『顛倒衣裳』?」文禮答曰:「明府初臨,堯德未彰,是以賤民顛倒衣裳耳。」

徐孺子年九歲,嘗月下戲,人語之曰:「若令月中無物,當極明邪!」徐曰:「不然,譬如人眼中有瞳子,無此必不明。」

孔文舉年十歲,隨父到洛。時李元禮有盛名,為司隸校尉,詣門者皆雋才清稱,及中表親戚乃通。文舉至門,謂吏曰:「我是李府君親。」既通,前坐。元禮問曰:「君與僕有何親?」對曰:「昔先君仲尼,與君先人伯陽,有師資之尊;是僕與君奕世為通好也。」元禮及賓客莫不奇之。太中大夫陳韙後至,人以其語語之。韙曰:「小時了了,大未必佳!」文舉曰:「想君小時,必當了了!」韙大踧踖。

孔文舉有二子,大者六歲,小者五歲。晝日父眠,小者床頭盜酒飲之。大兒謂曰:「何以不拜?」荅曰:「偷,那得行禮!」

孔融被收,中外惶怖。時融兒大者九歲,小者八歲。二兒故琢釘戲,了無遽容。融謂使者曰:「冀罪止於身,二兒可得全不?」兒徐進曰:「大人豈見覆巢之下,復有完卵乎?」尋亦收至。

潁川太守髠陳仲弓。客有問元方:「府君何如?」元方曰:「高明之君也。」「足下家君何如?」曰:「忠臣孝子也。」客曰:「易稱『二人同心,其利斷金;同心之言,其臭如蘭。』何有高明之君而刑忠臣孝子者乎?」元方曰:「足下言何其謬也!故不相荅。」客曰:「足下但因傴為恭不能答。」元方曰:「昔高宗放孝子孝己,尹吉甫放孝子伯奇,董仲舒放孝子符起。唯此三君,高明之君;唯此三子,忠臣孝子。」客慚而退。

荀慈明與汝南袁閬相見,問潁川人士,慈明先及諸兄。閬笑曰:「士但可因親舊而已乎?」慈明曰:「足下相難,依據者何經?」閬曰:「方問國士而及諸兄,是以尤之耳。」慈明曰:「昔者祁奚內舉不失其子,外舉不失其讎,以為至公。公旦文王之詩,不論堯舜之德,而頌文武者,親親之義也。春秋之義,內其國而外諸夏。且不愛其親而愛他人者,不為悖德乎?」

禰衡被魏武謫為鼓吏,正月半試鼓。衡揚枹為漁陽摻撾,淵淵有金石聲,四坐為之改容。孔融曰:「禰衡罪同胥靡,不能發明王之夢。」魏武慚而赦之。

南郡龐士元聞司馬德操在潁川,故二千里候之。至,遇德操採桑,士元從車中謂曰:「吾聞丈夫處世,當帶金佩紫,焉有屈洪流之量,而執絲婦之事。」德操曰:「子且下車,子適知邪徑之速,不慮失道之迷。昔伯成耦耕,不慕諸侯之榮;原憲桑樞,不易有官之宅。何有坐則華屋,行則肥馬,侍女數十,然後為奇。此乃許、父所以忼慨,夷、齊所以長嘆。雖有竊秦之爵,千駟之富,不足貴也!」士元曰:「僕生出邊垂,寡見大義。若不一叩洪鍾,伐雷鼓,則不識其音響也。」

劉公幹以失敬罹罪,文帝問曰:「卿何以不謹於文憲?」楨荅曰:「臣誠庸短,亦由陛下綱目不疏。」

鍾毓、鍾會少有令譽。年十三,魏文帝聞之,語其父鍾繇曰:「可令二子來。」於是敕見。毓面有汗,帝曰:「卿面何以汗?」毓對曰:「戰戰惶惶,汗出如漿。」復問會:「卿何以不汗?」對曰:「戰戰慄慄,汗不敢出。」

鍾毓兄弟小時,值父晝寢,因共偷服藥酒。其父時覺,且託寐以觀之。毓拜而後飲,會飲而不拜。既而問毓何以拜?毓曰:「『酒以成禮』,不敢不拜。」又問會何以不拜?會曰:「偷本非禮,所以不拜。」

魏明帝為外祖母築館於甄氏。既成,自行視,謂左右曰:「館當以何為名?」侍中繆襲曰:「陛下聖思齊於哲王;罔極過於曾、閔。此館之興,情鍾舅氏,宜以『渭陽』為名。」

何平叔云:「服五石散,非唯治病,亦覺神明開朗。」

嵇中散語趙景真:「卿瞳子白黑分明,有白起之風;恨量小狹。」趙云:「尺表能審璣衡之度,寸管能測往復之氣;何必在大?但問識如何耳!」

司馬景王東征,取上黨李喜,以為從事中郎。因問喜曰:「昔先公辟君不就,今孤召君,何以來?」喜對曰:「先公以禮見待,故得以禮進退;明公以法見繩,喜畏法而至耳!」

鄧艾口吃,語稱「艾艾……」。晉文王戲之曰:「卿云『艾艾……』,為是幾艾?」對曰:「『鳳兮,鳳兮』,故是一『鳳』。」

嵇中散既被誅,向子期舉郡計入洛,文王引進,問曰:「聞君有箕山之志,何以在此?」對曰:「巢、許狷介之士,不足多慕。」王大咨嗟。

晉武帝始登阼,探策得「一」。王者世數,繫此多少。帝既不說,群臣失色,莫能有言者。侍中裴楷進曰:「臣聞:『天得一以清,地得一以寧,侯王得一以為天下貞。』」帝說。群臣嘆服。

滿奮畏風,在晉武帝坐;北窗作琉璃屏,實密似疏,奮有難色。帝笑之。奮荅曰:「臣猶吳牛,見月而喘。」

諸葛靚在吳,於朝堂大會。孫皓問:「卿字仲思,為何所思?」對曰:「在家思孝,事君思忠,朋友思信,如斯而已。」

蔡洪赴洛,洛中人問曰:「幕府初開,群公辟命,求英奇於仄陋,採賢雋於巖穴。君吳、楚之士,亡國之餘,有何異才而應斯舉?」蔡荅曰:「夜光之珠,不必出於孟津之河;盈握之璧,不必采於崑崙之山。大禹生於東夷,文王生於西羌,聖賢所出,何必常處?昔武王伐紂,遷頑民於洛邑,得無諸君是其苗裔乎?」

諸名士共至洛水戲。還,樂令問王夷甫曰:「今日戲樂乎?」王曰:「裴僕射善談名理,混混有雅致;張茂先論史漢,靡靡可聽;我與王安豐說延陵、子房,亦超超玄箸。」

王武子、孫子荊、各言其土地人物之美。王云:「其地坦而平,其水淡而清,其人廉且貞。」孫云:「其山嶵巍以嵯峨,其水渫而揚波,其人磊呵而英多。」

樂令女適大將軍成都王穎。王兄長沙王執權於洛,遂構兵相圖。長沙王親近小人,遠外君子,凡在朝者,人懷危懼。樂令既允朝望,加有婚親,群小讒於長沙。長沙嘗問樂令,樂令神色自若,徐荅曰:「豈以五男易一女?」由是釋然,無復疑慮。

陸機詣王武子,武子前置數斛羊酪,指以示陸曰:「卿江東何以敵此?」陸云:「有千里蓴羹,但未下鹽豉耳!」

中朝有小兒,父病,行乞藥。主人問病,曰:「患瘧也。」主人曰:「尊侯明德君子,何以病瘧?」荅曰:「來病君子,所以為瘧耳。」

崔正熊詣都郡。都郡將姓陳,問正熊:「君去崔杼幾世?」荅曰:「民去崔杼,如明府之去陳恆。」

元帝始過江,謂顧驃騎曰:「寄人國土,心常懷慚。」榮跪對曰:「臣聞王者以天下為家,是以耿、亳無定處,九鼎遷洛邑。願陛下勿以遷都為念。」

庾公造周伯仁。伯仁曰:「君何所欣說而忽肥?」庾曰:「君復何所憂慘而忽瘦?」伯仁曰:「吾無所憂,直是清虛日來,滓穢日去耳。」

過江諸人,每至美日,輒相邀新亭,藉卉飲宴。周侯中坐而嘆曰:「風景不殊,正自有山河之異。」皆相視流涕,唯王丞相愀然變色曰:「當共戮力王室,克復神州,何至作楚囚相對邪?」

衛洗馬初欲渡江,形神慘顇,語左右云:「見此芒芒,不覺百端交集。苟未免有情,亦復誰能遣此!」

顧司空未知名,詣王丞相。丞相小極,對之疲睡。顧思所以叩會之,因謂同坐曰:「昔每聞元公道公協贊中宗,保全江表;體小不安,令人喘息。」丞相因覺,謂顧曰:「此子珪璋特達,機警有鋒。」

會稽賀生,體識清遠,言行以禮。不徒東南之美,實為海內之秀。

劉琨雖隔閡寇戎,志存本朝,謂溫嶠曰:「班彪識劉氏之復興,馬援知漢光之可輔。今晉阼雖衰,天命未改。吾欲立功於河北,使卿延譽於江南。子其行乎?」溫曰:「嶠雖不敏,才非昔人,明公以桓、文之姿,建匡立之功,豈敢辭命!」

溫嶠初為劉琨使來過江。于時江左營建始爾,綱紀未舉。溫新至,深有諸慮。既詣王丞相,陳主上幽越,社稷焚滅,山陵夷毀之酷,有黍離之痛。溫忠慨深烈,言與泗俱,丞相亦與之對泣。敘情既畢,便深自陳結,丞相亦厚相酬納。既出,懽然言曰:「江左自有管夷吾,此復何憂?」

王敦兄含為光祿勳。敦既逆謀,屯據南州,含委職奔姑孰。王丞相詣闕謝。司徒、丞相、揚州官僚問訊,倉卒不知何辭。顧司空時為揚州別駕,援翰曰:「王光祿遠避流言,明公蒙塵路次,群下不寧,不審尊體起居何如?」

郗太尉拜司空,語同坐曰:「平生意不在多,值世故紛紜,遂至臺鼎。朱博翰音,實愧於懷。」

高坐道人不作漢語,或問此意,簡文曰:「以簡應對之煩。」

周僕射雍容好儀形,詣王公,初下車,隱數人,王公含笑看之。既坐,傲然嘯詠。王公曰:「卿欲希嵇、阮邪?」荅曰:「何敢近舍明公,遠希嵇、阮!」

庾公嘗入佛圖,見臥佛,曰:「此子疲於津梁。」于時以為名言。

摯瞻曾作四郡太守、大將軍戶曹參軍,復出作內史,年始二十九。嘗別王敦,敦謂瞻曰:「卿年未三十,已為萬石,亦太蚤!」瞻曰:「方於將軍,少為太蚤;比之甘羅,已為太老。」

梁國楊氏子,九歲,甚聰惠。孔君平詣其父,父不在,乃呼兒出,為設果。果有楊梅,孔指以示兒曰:「此是君家果。」兒應聲荅曰:「未聞孔雀是夫子家禽。」

孔廷尉以裘與從弟沈,沈辭不受。廷尉曰:「晏平仲之儉,祠其先人,豚肩不掩豆,猶狐裘數十年,卿復何辭此?」於是受而服之。

佛圖澄與諸石遊,林公曰:「澄以石虎為海鷗鳥。」

謝仁祖年八歲,謝豫章將送客,爾時語已神悟,自參上流。諸人咸共嘆之曰:「年少一坐之顏回。」仁祖曰:「坐無尼父,焉別顏回?」

陶公疾篤,都無獻替之言,朝士以為恨。仁祖聞之曰:「時無豎刁,故不貽陶公話言。」時賢以為德音。

竺法深在簡文坐,劉尹問:「道人何以游朱門?」荅曰:「君自見其朱門,貧道如游蓬戶。」或云卞令。

孫盛為庾公記室參軍,從獵,將其二兒俱行。庾公不知,忽於獵場見齊莊,時年七八歲。庾謂曰:「君亦復來邪?」應聲荅曰:「所謂『無小無大,從公于邁』。」

孫齊由、齊莊二人,少時詣庾公,公問:「齊由何字?」荅曰:「字齊由。」公曰:「欲何齊邪?」曰:「齊許由。」「齊莊何字?」荅曰:「字齊莊。」公曰:「欲何齊?」曰:「齊莊周。」公曰:「何不慕仲尼而慕莊周?」對曰:「聖人生知,故難企慕。」庾公大喜小兒對。

張玄之、顧敷,是顧和中外孫,皆少而聰惠。和並知之,而常謂顧勝,親重偏至,張頗不懕。于時張年九歲,顧年七歲,和與俱至寺中。見佛般泥洹像,弟子有泣者,有不泣者,和以問二孫。玄謂「被親故泣,不被親故不泣」。敷曰:「不然,當由忘情故不泣,不能忘情故泣。」

庾法畼造庾太尉,握麈尾至佳,公曰:「此至佳,那得在?」法畼曰:「廉者不求,貪者不與,故得在耳。」

庾穉恭為荊州,以毛扇上武帝。武帝疑是故物。侍中劉劭曰:「柏梁雲構,工匠先居其下;管弦繁奏,鍾夔先聽其音。穉恭上扇,以好不以新。」庾後聞之曰:「此人宜在帝左右。」

何驃騎亡後,徵褚公入。既至石頭,王長史、劉尹同詣褚。褚曰:「真長何以處我?」真長顧王曰:「此子能言。」褚因視王,王曰:「國自有周公。」

桓公北征經金城,見前為琅邪時種柳,皆已十圍,慨然曰:「木猶如此,人何以堪!」攀枝執條,泫然流淚。

簡文作撫軍時,嘗與桓宣武俱入朝,更相讓在前。宣武不得已而先之,因曰:「伯也執殳,為王前驅。」簡文曰:「所謂『無小無大,從公于邁』。」

顧悅與簡文同年,而髮蚤白。簡文曰:「卿何以先白?」對曰:「蒲柳之姿,望秋而落;松柏之質,凌霜猶茂。」

桓公入峽,絕壁天懸,騰波迅急。迺嘆曰:「既為忠臣,不得為孝子,如何?」

初,熒惑入太微,尋廢海西。簡文登阼,復入太微,帝惡之。時郗超為中書在直。引超入曰:「天命脩短,故非所計,政當無復近日事不?」超曰:「大司馬方將外固封疆,內鎮社稷,必無若此之慮。臣為陛下以百口保之。」帝因誦庾仲初詩曰:「志士痛朝危,忠臣哀主辱。」聲甚悽厲。郗受假還東,帝曰:「致意尊公,家國之事,遂至於此!由是身不能以道匡衛,思患預防,愧嘆之深,言何能喻?」因泣下流襟。

簡文在暗室中坐,召宣武。宣武至,問上何在?簡文曰:「某在斯。」時人以為能。

簡文入華林園,顧謂左右曰:「會心處,不必在遠。翳然林水,便自有濠、濮閒想也。覺鳥獸禽魚,自來親人。」

謝太傅語王右軍曰:「中年傷於哀樂,與親友別,輒作數日惡。」王曰:「年在桑榆,自然至此,正賴絲竹陶寫。恆恐兒輩覺,損欣樂之趣。」

支道林常養數匹馬。或言,道人畜馬不韻,支曰:「貧道重其神駿。」

劉尹與桓宣武共聽講禮記。桓云:「時有入心處,便覺咫尺玄門。」劉曰:「此未關至極,自是金華殿之語。」

羊秉為撫軍參軍,少亡,有令譽。夏侯孝若為之敘,極相讚悼。羊權為黃門侍郎,侍簡文坐。帝問曰:「夏侯湛作羊秉敘絕可想。是卿何物?有後不?」權潸然對曰:「亡伯令問夙彰,而無有繼嗣。雖名播天聽,然胤絕聖世。」帝嗟慨久之。

王長史與劉真長別後相見,王謂劉曰:「卿更長進。」荅曰:「此若天之自高耳。」

劉尹云:「人想王荊產佳,此想長松下當有清風耳。」

王仲祖聞蠻語不解,茫然曰:「若使介葛盧來朝,故當不昧此語。」

劉真長為丹陽尹,許玄度出都就劉宿;床帷新麗,飲食豐甘。許曰:「若保全此處,殊勝東山。」劉曰:「卿若知吉凶由人,吾安得不保此!」王逸少在坐曰:「令巢、許遇稷、契,當無此言。」二人並有愧色。

王右軍與謝太傅共登冶城。謝悠然遠想,有高世之志。王謂謝曰:「夏禹勤王,手足胼胝;文王旰食,日不暇給。今四郊多壘,宜人人自效。而虛談廢務,浮文妨要,恐非當今所宜。」謝荅曰:「秦任商鞅,二世而亡,豈清言致患邪?」

謝太傅寒雪日內集,與兒女講論文義。俄而雪驟,公欣然曰:「白雪紛紛何所似?」兄子胡兒曰:「撒鹽空中差可擬。」兄女曰:「未若柳絮因風起。」公大笑樂。即公大兄無奕女,左將軍王凝之妻也。

王中郎令伏玄度、習鑿齒論青、楚人物。臨成,以示韓康伯。康伯都無言,王曰:「何故不言?」韓曰:「無可無不可。」

劉尹云:「清風朗月,輒思玄度。」

荀中郎在京口,登北固望海云:「雖未覩三山,便自使人有凌雲意。若秦、漢之君,必當褰裳濡足。」

謝公云:「賢聖去人,其間亦邇。」子姪未之許。公嘆曰:「若郗超聞此語,必不至河漢。」

支公好鶴,住剡東\ext{𡵙}山,有人遺其雙鶴;少時,翅長欲飛。支意惜之,乃鎩其翮。鶴軒翥不能復起,乃舒翼反頭視之,如有懊喪意。林曰:「既有凌霄之姿,何肯為人作耳目近玩?」養令翮成,置使飛去。

謝中郎經曲阿後湖,問左右:「此是何水?」荅曰:「曲阿湖。」謝曰:「故當淵注渟著,納而不流。」

晉武帝每餉山濤恒少。謝太傅以問子弟,車騎荅曰:「當由欲者不多,而使與者忘少。」

謝胡兒語庾道季:「諸人莫當就卿談,可堅城壘。」庾曰:「若文度來,我以偏師待之;康伯來,濟河焚舟。」

李弘度常嘆不被遇。殷揚州知其家貧,問:「君能屈志百里不?」李荅曰:「北門之嘆,久已上聞;窮猿奔林,豈暇擇木?」遂授剡縣。

王司州至吳興印渚中看。嘆曰:「非唯使人情開滌,亦覺日月清朗。」

謝萬作豫州都督,新拜,當西之都邑,相送累日,謝疲頓。於是高侍中往,徑就謝坐,因問:「卿今仗節方州,當疆理西蕃,何以為政?」謝粗道其意。高便為謝道形勢,作數百語。謝遂起坐。高去後,謝追曰:「阿酃故麤有才具。」謝因此得終坐。

袁彥伯為謝安南司馬,都下諸人送至瀨鄉。將別,既自悽惘,嘆曰:「江山遼落,居然有萬里之勢。」

孫綽賦遂初,築室畎川,自言見止足之分。齋前種一株松,恒自手壅治之。高世遠時亦鄰居,語孫曰:「松樹子非不楚楚可憐,但永無棟梁用耳!」孫曰:「楓柳雖合抱,亦何所施?」

桓征西治江陵城甚麗,會賓僚出江津望之,云:「若能目此城者有賞。」顧長康時為客,在坐,目曰:「遙望層城,丹樓如霞。」桓即賞以二婢。

王子敬語王孝伯曰:「羊叔子自復佳耳,然亦何與人事?」故不如銅雀臺上妓。」

林公見東陽長山曰:「何其坦迤!」

顧長康從會稽還,人問山川之美。顧云:「千巖競秀,萬壑爭流,草木蒙籠其上,若雲興霞蔚。」

簡文崩,孝武年十餘歲立,至暝不臨。左右啟「依常應臨」。帝曰:「哀至則哭,何常之有!」

孝武將講孝經,謝公兄弟與諸人私庭講習,車武子難苦問謝,謂袁羊曰:「不問則德音有遺;多問則重勞二謝。」袁曰:「必無此嫌。」車曰:「何以知爾?」袁曰:「何嘗見明鏡疲於屢照,清流憚於惠風?」

王子敬云:「從山陰道上行,山川自相映發,使人應接不暇;若秋冬之際,尤難為壞。」

謝太傅問諸子姪:「子弟亦何預人事,而正欲使其佳?」諸人莫有言者,車騎荅曰:「譬如芝蘭玉樹,欲使其生於階庭耳。」

道壹道人好整飾音辭,從都下還東山,經吳中。已而會雪下,未甚寒。諸道人問在道所經。壹公曰:「風霜固所不論,乃先集其慘澹。郊邑正自飄瞥,林岫便已皓然。」

張天錫為涼州刺史,稱制西隅。既為苻堅所禽,用為侍中。後於壽陽俱敗,至都,為孝武所器;每入,言論無不竟日。頗有嫉之者,於坐問張:「北方何物可貴?」張曰:「桑椹甘香,鴟鴞革響;淳酪養性,人無嫉心。」

顧長康拜桓宣武墓,作詩云:「山崩溟海竭,魚鳥將何依。」人問之曰:「卿憑重桓乃爾,哭之狀其可見乎?」顧曰:「鼻如廣莫長風,眼如懸河決溜。」或曰:「聲如震雷破山,淚如傾河注海。」

毛伯成既負其才氣,常稱:「寧為蘭摧玉折,不作蕭敷艾榮。」

范甯作豫章,八日請佛有板。眾僧疑,或欲作荅。有小沙彌在坐末曰:「世尊默然,則為許可。眾從其義。

司馬太傅齋中夜坐,于時天月明淨,都無纖翳;太傅嘆以為佳。謝景重在坐,荅曰:「意謂乃不如微雲點綴。」太傅因戲謝曰:「卿居心不淨,乃復強欲滓穢太清邪?」

王中郎甚愛張天錫,問之曰:「卿觀過江諸人經緯,江左軌轍,有何偉異?後來之彥,復何如中原?」張曰:「研求幽邃,自王何以還;因時脩制,荀樂之風。」王曰:「卿知見有餘,何故為符堅所制?」荅曰:「陽消陰息,故天步屯蹇;否剝成象,豈足多譏?」

謝景重女適王孝伯兒,二門公甚相愛美。謝為太傅長史,被彈;王即取作長史,帶晉陵郡。太傅已構嫌孝伯,不欲使其得謝,還取作咨議。外示縶維,而實以乖閒之。及孝伯敗後,太傅繞東府城行散,僚屬悉在南門要望候拜,時謂謝曰:「王甯異謀,云是卿為其計。」謝曾無懼色,斂笏對曰:「樂彥輔有言:『豈以五男易一女?』」太傅善其對,因舉酒勸之曰:「故自佳!故自佳!」

桓玄義興還後,見司馬太傅,太傅已醉,坐上多客,問人云:「桓溫來欲作賊,如何?」桓玄伏不得起。謝景重時為長史,舉板荅曰:「故宣武公黜昏暗,登聖明,功超伊、霍。紛紜之議,裁之聖鑒。」太傅曰:「我知!我知!」即舉酒云:「桓義興,勸卿酒。」桓出謝過。

宣武移鎮南州,制街衢平直。人謂王東亭曰:「丞相初營建康,無所因承,而制置紆曲,方此為劣。」東亭曰:「此丞相乃所以為巧。江左地促,不如中國;若使阡陌條暢,則一覽而盡;故紆餘委曲,若不可測。」

桓玄詣殷荊州,殷在妾房晝眠,左右辭不之通。桓後言及此事,殷云:「初不眠,縱有此,豈不以『賢賢易色』也。」

桓玄問羊孚:「何以共重吳聲?」羊曰:「當以其妖而浮。」

謝混問羊孚:「何以器舉瑚璉?」羊曰:「故當以為接神之器。」

桓玄既篡位,後御床微陷,群臣失色。侍中殷仲文進曰:「當由聖德淵重,厚地所以不能載。」時人善之。

桓玄既篡位,將改置直館,問左右:「虎賁中郎省,應在何處?」有人荅曰:「無省。」當時殊忤旨。問:「何以知無?」荅曰:「潘岳秋興賦敘曰:『余兼虎賁中郎將,寓直散騎之省。』玄咨嗟稱善。

謝靈運好戴曲柄笠,孔隱士謂曰:「卿欲希心高遠,何不能遺曲蓋之貌?」謝荅曰:「將不畏影者,未能忘懷。」



\chapter{政事第三}

陳仲弓為太丘長,時吏有詐稱母病求假,事覺收之,令吏殺焉。主簿請付獄,考眾姦。仲弓曰:「欺君不忠,病母不孝;不忠不孝,其罪莫大。考求眾姦,豈復過此?」

陳仲弓為太丘長,有劫賊殺財主主者,捕之。未至發所,道聞民有在草不起子者,回車往治之。主簿曰:「賊大,宜先按討。」仲弓曰:「盜殺財主,何如骨肉相殘?」

陳元方年十一時,候袁公。袁公問曰:「賢家君在太丘,遠近稱之,何所履行?」元方曰:「老父在太丘,彊者綏之以德,弱者撫之以仁,恣其所安,久而益敬。」袁公曰:「孤往者嘗為鄴令,正行此事。不知卿家君法孤?孤法卿父?」元方曰:「周公、孔子,異世而出,周旋動靜,萬里如一;周公不師孔子,孔子亦不師周公。」

賀太傅作吳郡,初不出門。吳中諸強族輕之,乃題府門云:「會稽雞,不能啼。」賀聞故出行,至門反顧,索筆足之曰:「不可啼,殺吳兒!」於是至諸屯邸,檢校諸顧、陸役使官兵及藏逋亡,悉以事言上,罪者甚眾。陸抗時為江陵都督,故下請孫皓,然後得釋。

山公以器重朝望,年踰七十,猶知管時任。貴勝年少,若和、裴、王之徒,並共言詠。有署閣柱曰:「閣東,有大牛,和嶠鞅,裴楷鞦,王濟剔嬲不得休。」或云:潘尼作之。

賈充初定律令,與羊祜共咨太傅鄭沖。沖曰:「皋陶嚴明之旨,非僕闇懦所探。」羊曰:「上意欲令小加弘潤。」沖乃粗下意。

山司徒前後選,殆周遍百官,舉無失才。凡所題目,皆如其言。唯用陸亮,是詔所用,與公意異,爭之不從。亮亦尋為賄敗。

嵇康被誅後,山公舉康子紹為秘書丞。紹咨公出處,公曰:「為君思之久矣!天地四時,猶有消息,而況人乎?」

王安期為東海郡,小吏盜池中魚,綱紀推之。王曰:「文王之囿,與眾共之。池魚復何足惜!」

王安期作東海郡,吏錄一犯夜人來。王問:「何處來?」云:「從師家受書還,不覺日晚。」王曰:「鞭撻甯越以立威名,恐非致理之本。」使吏送令歸家。

成帝在石頭,任讓在帝前戮侍中鍾雅,右衛將軍劉超。帝泣曰:「還我侍中!」讓不奉詔,遂斬超、雅。事平之後,陶公與讓有舊,欲宥之。許柳兒思妣者至佳,諸公欲全之。若全思妣,則不得不為陶全讓,於是欲并宥之。事奏,帝曰:「讓是殺我侍中者,不可宥!」諸公以少主不可違,并斬二人。

王丞相拜揚州,賓客數百人並加霑接,人人有說色。唯有臨海一客姓任及數胡人為未洽,公因便還到過任邊云:「君出,臨海便無復人。」任大喜說。因過胡人前彈指云:「蘭闍,蘭闍。」群胡同笑,四坐並懽。

陸太尉詣王丞相咨事,過後輒翻異。王公怪其如此,後以問陸。陸曰:「公長民短,臨時不知所言,既後覺其不可耳。」

丞相嘗夏月至石頭看庾公。庾公正料事,丞相云:「暑可小簡之。」庾公曰:「公之遺事,天下亦未以為允。」

丞相末年,略不復省事,正封籙諾之。自嘆曰:「人言我憒憒,後人當思此憒憒。」

陶公性檢厲,勤於事。作荊州時,敕船官悉錄鋸木屑,不限多少,咸不解此意。後正會,值積雪始晴,聽事前除雪後猶濕,於是悉用木屑覆之,都無所妨。官用竹,皆令錄厚頭,積之如山。後桓宣武伐蜀,裝船,悉以作釘。又云:嘗發所在竹篙,有一官長連根取之,仍當足,乃超兩階用之。

何驃騎作會稽,虞存弟謇作郡主簿,以何見客勞損,欲白斷常客,使家人節量,擇可通者作白事成,以見存。存時為何上佐,正與謇共食,語云:「白事甚好,待我食畢作教。」食竟,取筆題白事後云:「若得門庭長如郭林宗者,當如所白。汝何處得此人?」謇於是止。

王、劉與深公共看何驃騎,驃騎看文書不顧之。王謂何曰:「我今故與深公來相看,望卿擺撥常務,應對共言,哪得方低頭看此邪?」何曰:「我不看此,卿等何以得存?」諸人以為佳。

桓公在荊州,全欲以德被江、漢,恥以威刑肅物。令史受杖,正從朱衣上過。桓式年少,從外來,云:「向從閣下過,見令史受杖,上捎雲根,下拂地足。」意譏不著。桓公云:「我猶患其重。」

簡文為相,事動經年,然後得過。桓公甚患其遲,常加勸免。太宗曰:「一日萬機,那得速!」

山遐去東陽,王長史就簡文索東陽云:「承藉猛政,故可以和靜致治。」

殷浩始作揚州,劉尹行,日小欲晚,便使左右取襆,人問其故?荅曰:「刺史嚴,不敢夜行。」

謝公時,兵厮逋亡,多近竄南塘,下諸舫中。或欲求一時搜索,謝公不許,云:「若不容置此輩,何以為京都?」

王大為吏部郎,嘗作選草,臨當奏,王僧彌來,聊出示之。僧彌得便以己意改易所選者近半,王大甚以為佳,更寫即奏。

王東亭與張冠軍善。王既作吳郡,人問小令曰:「東亭作郡,風政何似?」答曰:「不知治化何如,唯與張祖希情好日隆耳。」

殷仲堪當之荊州,王東亭問曰:「德以居全為稱,仁以不害物為名。方今宰牧華夏,處殺戮之職,與本操將不乖乎?」殷答曰:「皋陶造刑辟之制,不為不賢;孔丘居司寇之任,未為不仁。」



\chapter{文學第四}

鄭玄在馬融門下,三年不得相見,高足弟子傳授而已。嘗算渾天不合,諸弟子莫能解。或言玄能者,融召令算,一轉便決,眾咸駭服。及玄業成,辭歸,既而融有「禮樂皆東」之嘆。恐玄擅名而心忌焉。玄亦疑有追,乃坐橋下,在水上據屐。融果轉式逐之,告左右曰:「玄在土下水上而據木,此必死矣。」遂罷追,玄竟以得免。

鄭玄欲注春秋傳,尚未成時,行與服子慎遇宿客舍,先未相識,服在外車上與人說己注傳意。玄聽之良久,多與己同。玄就車與語曰:「吾久欲注,尚未了。聽君向言,多與吾同。今當盡以所注與君。」遂為服氏注。

鄭玄家奴婢皆讀書。嘗使一婢,不稱旨,將撻之。方自陳說,玄怒,使人曳箸泥中。須臾,復有一婢來,問曰:「胡為乎泥中?」荅曰:「薄言往愬,逢彼之怒。」

服虔既善春秋,將為注,欲參考同異,聞崔烈集門生講傳,遂匿姓名,為烈門人賃作食。每當至講時,輒竊聽戶壁間。既知不能踰己,稍共諸生敘其短長。烈聞,不測何人,然素聞虔名,意疑之。明蚤往,及未寤,便呼:「子慎!子慎!」虔不覺驚應,遂相與友善。

鍾會撰四本論,始畢,甚欲使嵇公一見。置懷中,既定,畏其難,懷不敢出,於戶外遙擲,便回急走。

何晏為吏部尚書,有位望,時談客盈坐,王弼未弱冠,往見之。晏聞弼名,因條向者勝理語弼曰:「此理僕以為極,可得復難不?」弼便作難,一坐人便以為屈,於是弼自為客主數番,皆一坐所不及。

何平叔注老子,始成,詣王輔嗣。見王注精奇,迺神伏曰:「若斯人,可與論天人之際矣!」因以所注為道德二論。

王輔嗣弱冠詣裴徽,徽問曰:「夫無者,誠萬物之所資,聖人莫肯致言,而老子申之無已,何邪?」弼曰:「聖人體無,無又不可以訓,故言必及有;老、莊未免於有,恆訓,其所不足。」

傅嘏善言虛勝,荀粲談尚玄遠。每至共語,有爭而不相喻。裴冀州釋二家之義,通彼我之懷,常使兩情皆得,彼此俱暢。

何晏注老子未畢,見王弼自說注老子旨。何意多所短,不復得作聲,但應諾諾。遂不復注,因作道德論。

中朝時,有懷道之流,有詣王夷甫咨疑者。值王昨已語多,小極,不復相酬荅,乃謂客曰:「身今少惡,裴逸民亦近在此,君可往問。」

裴成公作崇有論,時人攻難之,莫能折。唯王夷甫來,如小屈。時人即以王理難裴,理還復申。

諸葛厷年少不肯學問。始與王夷甫談,便已超詣。王嘆曰:「卿天才卓出,若復小加研尋,一無所愧。」厷後看莊、老,更與王語,便足相抗衡。

衛玠總角時問樂令「夢」,樂云:「是想。」衛曰:「形神所不接,而夢豈是想邪?」樂云:「因也。未嘗夢乘車入鼠穴,擣噉鐵杵,皆無想無因故也。」衛思因,經日不得,遂成病。樂聞,故命駕為剖析之。衛即小差。樂嘆曰:「此兒胸中當必無膏肓之疾!」

庾子嵩讀莊子,開卷一尺許便放去,曰:「了不異人意。」

客問樂令「旨不至」者,樂亦不復剖析文句,直以麈尾柄确几曰:「至不?」客曰:「至!」樂因又舉麈尾曰:「若至者,那得去?」於是客乃悟服。樂辭約而旨達,皆此類。

初,注莊子者數十家,莫能究其旨要。向秀於舊注外為解義,妙析奇致,大暢玄風。唯秋水、至樂二篇未竟而秀卒。秀子幼,義遂零落,然猶有別本。郭象者,為人薄行,有雋才。見秀義不傳於世,遂竊以為己注。乃自注秋水、至樂二篇,又易馬蹄一篇,其餘眾篇,或定點文句而已。後秀義別本出,故今有向、郭二莊,其義一也。

阮宣子有令聞,太尉王夷甫見而問曰:「老莊與聖教同異?」對曰:「將無同!」太尉善其言,辟之為掾。世謂「三語掾」。衛玠嘲之曰:「一言可辟,何假於三?」宣子曰:「苟是天下人望,亦可無言而辟,復何假一?」遂相與為友。

裴散騎娶王太尉女。婚後三日,諸壻大會,當時名士,王裴子弟悉集。郭子玄在坐,挑與裴談。子玄才甚豐瞻,始數交未快。郭陳張甚盛,裴徐理前語,理致甚微,四坐咨嗟稱快。王亦以為奇,謂諸人曰:「君輩勿為爾,將受困寡人女壻!」

衛玠始度江,見王大將軍。因夜坐,大將軍命謝幼輿。玠見謝,甚說之,都不復顧王,遂達旦微言。王永夕不得豫。玠體素羸,恆為母所禁。爾夕忽極,於此病篤,遂不起。

舊云:王丞相過江左,止道聲無哀樂、養生、言盡意,三理而已。然宛轉關生,無所不入。

殷中軍為庾公長史,下都,王丞相為之集,桓公、王長史、王藍田、謝鎮西並在。丞相自起解帳帶麈尾,語殷曰:「身今日當與君共談析理。」既共清言,遂達三更。丞相與殷共相往反,其餘諸賢,略無所關。既彼我相盡,丞相乃嘆曰:「向來語,乃竟未知理源所歸,至於辭喻不相負。正始之音,正當爾耳!」明旦,桓宣武語人曰:「昨夜聽殷、王清言甚佳,仁祖亦不寂寞,我亦時復造心,顧看兩王掾,輒翣如生母狗馨。」

殷中軍見佛經云:「理亦應阿堵上。」

謝安年少時,請阮光祿道白馬論。為論以示謝,于時謝不即解阮語,重相咨盡。阮乃嘆曰:「非但能言人不可得,正索解人亦不可得!」

褚季野語孫安國云:「北人學問,淵綜廣博。」孫答曰:「南人學問,清通簡要。」支道林聞之曰:「聖賢固所忘言。自中人以還,北人看書,如顯處視月;南人學問,如牖中窺日。」

劉真長與殷淵源談,劉理如小屈,殷曰:「惡,卿不欲作將善雲梯仰攻。」

殷中軍云:「康伯未得我牙後慧。」

謝鎮西少時,聞殷浩能清言,故往造之。殷未過有所通,為謝標榜諸義,作數百語。既有佳致,兼辭條豐蔚,甚足以動心駭聽。謝注神傾意,不覺流汗交面。殷徐語左右:「取手巾與謝郎拭面。」

宣武集諸名勝講易,日說一卦。簡文欲聽,聞此便還。曰:「義自當有難易,其以一卦為限邪?」

有北來道人好才理,與林公相遇於瓦官寺,講小品。于時竺法深、孫興公悉共聽。此道人語,屢設疑難,林公辯答清析,辭氣俱爽。此道人每輒摧屈。孫問深公:「上人當是逆風家,向來何以都不言?」深公笑而不答。林公曰:「白旃檀非不馥,焉能逆風?」深公得此義,夷然不屑。

孫安國往殷中軍許共論,往反精苦,客主無間。左右進食,冷而復煗者數四。彼我奮擲麈尾,悉脫落滿餐飯中。賓主遂至莫忘食。殷乃語孫曰:「卿莫作強口馬,我當穿卿鼻。」孫曰:「卿不見決鼻牛,人當穿卿頰。」

莊子逍遙篇,舊是難處,諸名賢所可鑽味而不能拔理於郭向之外。支道林在白馬寺中,將馮太常共語,因及逍遙。支卓然標新理於二家之表,立異義於眾賢之外,皆是諸名賢尋味之所不得。後遂用支理。

殷中軍嘗至劉尹所清言。良久,殷理小屈,遊辭不已,劉亦不復答。殷去後,乃云:「田舍兒,強學人作爾馨語。」

殷中軍雖思慮通長,然於才性偏精。忽言及四本,便苦湯池鐵城,無可攻之勢。

支道林造即色論,論成,示王中郎。中郎都無言。支曰:「默而識之乎?」王曰:「既無文殊,誰能見賞?」

王逸少作會稽,初至,支道林在焉。孫興公謂王曰:「支道林拔新領異,胸懷所及,乃自佳,卿欲見不?」王本自有一往雋氣,殊自輕之。後孫與支共載往王許,王都領域,不與交言。須臾,支退,後正值王當行,車已在門。支語王曰:「君未可去,貧道與君小語。」因論莊子逍遙遊。支作數千言,才藻新奇,花爛映發。王遂披襟解帶,留連不能已。

三乘佛家滯義,支道林分判,使三乘炳然。諸人在下坐聽,皆云可通。支下坐,自共說,正當得兩,入三便亂。今義弟子雖傳,猶不盡得。

許掾年少時,人以比王苟子,許大不平。時諸人士及於法師並在會稽西寺講,王亦在焉。許意甚忿,便往西寺與王論理,共決優劣。苦相折挫,王遂大屈。許復執王理,王執許理,更相覆疏;王復屈。許謂支法師曰:「弟子向語何似?」支從容曰:「君語佳則佳矣,何至相苦邪?豈是求理中之談哉!」

林道人詣謝公,東陽時始總角,新病起,體未堪勞。與林公講論,遂至相苦。母王夫人在壁後聽之,再遣信令還,而太傅留之。王夫人因自出云:「新婦少遭家難,一生所寄,唯在此兒。」因流涕抱兒以歸。謝公語同坐曰:「家嫂辭情忼慨,致可傳述,恨不使朝士見。」

支道林、許掾諸人共在會稽王齋頭。支為法師,許為都講。支通一義,四坐莫不厭心。許送一難,眾人莫不抃舞。但共嗟詠二家之美,不辯其理之所在。

謝車騎在安西艱中,林道人往就語,將夕乃退。有人道上見者問云:「公何處來?」答云:「今日與謝孝劇談一出來。」

支道林初從東出,住東安寺中。王長史宿構精理,并撰其才藻,往與支語,不大當對。王敘致作數百語,自謂是名理奇藻。支徐徐謂曰:「身與君別多年,君義言了不長進。」王大慚而退。

殷中軍讀小品,下二百籤,皆是精微,世之幽滯。嘗欲與支道林辯之,竟不得。今小品猶存。

佛經以為袪練神明,則聖人可致。簡文云:「不知便可登峰造極不?然陶練之功,尚不可誣。」

于法開始與支公爭名,後情漸歸支,意甚不分,遂遁跡剡下。遣弟子出都,語使過會稽。于時支公正講小品。開戒弟子:「道林講,比汝至,當在某品中。」因示語攻難數十番,云:「舊此中不可復通。」弟子如言詣支公。正值講,因謹述開意。往反多時,林公遂屈。厲聲曰:「君何足復受人寄載來!」

殷中軍問:「自然無心於稟受,何以正善人少,惡人多?」諸人莫有言者。劉尹答曰:「譬如寫水著地,正自縱橫流漫,略無正方圓者。」一時絕嘆,以為名通。

康僧淵初過江,未有知者,恆周旋市肆,乞索以自營。忽往殷淵源許,值盛有賓客,殷使坐,麤與寒溫,遂及義理。語言辭旨,曾無愧色。領略麤舉,一往參詣。由是知之。

殷謝諸人共集。謝因問殷:「眼往屬萬形,萬形來入眼不?」

人有問殷中軍:「何以將得位而夢棺器,將得財而夢矢穢?」殷曰:「官本是臭腐,所以將得而夢棺屍;財本是糞土,所以將得而夢穢汙。」時人以為名通。

殷中軍被廢東陽,始看佛經。初視維摩詰,疑般若波羅密太多,後見小品,恨此語少。

支道林、殷淵源俱在相王許。相王謂二人:「可試一交言。而才性殆是淵源崤函之固,君其慎焉!」支初作,改轍遠之,數四交,不覺入其玄中。相王撫肩笑曰:「此自是其勝場,安可爭鋒!」

謝公因子弟集聚,問毛詩何句最佳?遏稱曰:「昔我往矣,楊柳依依;今我來思,雨雪霏霏。」公曰:「訏謨定命,遠猷辰告。」謂此句偏有雅人深致。

張憑舉孝廉出都,負其才氣,謂必參時彥。欲詣劉尹,鄉里及同舉者共笑之。張遂詣劉。劉洗濯料事,處之下坐,唯通寒暑,神意不接。張欲自發無端。頃之,長史諸賢來清言。客主有不通處,張乃遙於末坐判之,言約旨遠,足暢彼我之懷,一坐皆驚。真長延之上坐,清言彌日,因留宿至曉。張退,劉曰:「卿且去,正當取卿共詣撫軍。」張還船,同侶問何處宿?張笑而不答。須臾,真長遣傳教覓張孝廉船,同侶惋愕。即同載詣撫軍。至門,劉前進謂撫軍曰:「下官今日為公得一太常博士妙選!」既前,撫軍與之話言,咨嗟稱善曰:「張憑勃窣為理窟。」即用為太常博士。

汰法師云:「『六通』、『三明』同歸,正異名耳。」

支道林、許、謝盛德,共集王家。謝顧謂諸人:「今日可謂彥會,時既不可留,此集固亦難常。當共言詠,以寫其懷。」許便問主人,有莊子不?正得漁父一篇。謝看題,便各使四坐通。支道林先通,作七百許語,敘致精麗,才藻奇拔,眾咸稱善。於是四坐各言懷畢。謝問曰:「卿等盡不?」皆曰:「今日之言,少不自竭。」謝後麤難,因自敘其意,作萬餘語,才峰秀逸。既自難干,加意氣擬託,蕭然自得,四坐莫不厭心。支謂謝曰:「君一往奔詣,故復自佳耳。」

殷中軍、孫安國、王、謝能言諸賢,悉在會稽王許。殷與孫共論易象妙於見形。孫語道合,意氣干雲。一坐咸不安孫理,而辭不能屈。會稽王慨然嘆曰:「使真長來,故應有以制彼。」既迎真長,孫意己不如。真長既至,先令孫自敘本理。孫麤說己語,亦覺殊不及向。劉便作二百許語,辭難簡切,孫理遂屈。一坐同時拊掌而笑,稱美良久。

僧意在瓦官寺中,王苟子來,與共語,便使其唱理。意謂王曰:「聖人有情不?」王曰:「無。」重問曰:「聖人如柱邪?」王曰:「如籌算,雖無情,運之者有情。」僧意云:「誰運聖人邪?」苟子不得答而去。

司馬太傅問謝車騎:「惠子其書五車,何以無一言入玄?」謝曰:「故當是其妙處不傳。」

殷中軍被廢,徙東陽,大讀佛經,皆精解。唯至「事數」處不解。遇見一道人,問所籤,便釋然。

殷仲堪精覈玄論,人謂莫不研究。殷乃嘆曰:「使我解四本,談不翅爾。」

殷荊州曾問遠公:「易以何為體?」答曰:「易以感為體。」殷曰:「銅山西崩,靈鍾東應,便是易耶?」遠公笑而不答。

羊孚弟娶王永言女。及王家見壻,孚送弟俱往。時永言父東陽尚在,殷仲堪是東陽女壻,亦在坐。孚雅善理義,乃與仲堪道齊物。殷難之,羊云:「君四番後,當得見同。」殷笑曰:「乃可得盡,何必相同?」乃至四番後一通。殷咨嗟曰:「僕便無以相異。」嘆為新拔者久之。

殷仲堪云:「三日不讀道德經,便覺舌本間強。」

提婆初至,為東亭第講阿毗曇。始發講,坐裁半,僧彌便云:「都已曉。」即於坐分數四有意道人更就餘屋自講。提婆講竟,東亭問法岡道人曰:「弟子都未解,阿彌那得已解?所得云何?」曰:「大略全是,故當小未精覈耳。」

桓南郡與殷荊州共談,每相攻難。年餘後,但一兩番。桓自嘆才思轉退。殷云:「此乃是君轉解。」

文帝嘗令東阿王七步中作詩,不成者行大法。應聲便為詩曰:「煮豆持作羹,漉菽以為汁。萁在釜下然,豆在釜中泣。本自同根生,相煎何太急?」帝深有慚色。

魏朝封晉文王為公,備禮九錫,文王固讓不受。公卿將校當詣府敦喻。司空鄭沖馳遣信就阮籍求文。籍時在袁孝尼家,宿醉扶起,書札為之,無所點定,乃寫付使。時人以為神筆。

左太沖作三都賦初成,時人互有譏訾,思意不愜。後示張公。張曰:「此二京可三,然君文未重於世,宜以經高名之士。」思乃詢求於皇甫謐。謐見之嗟嘆,遂為作敘。於是先相非貳者,莫不斂袵讚述焉。

劉伶著酒德頌,意氣所寄。

樂令善於清言,而不長於手筆。將讓河南尹,請潘岳為表。潘云:「可作耳。要當得君意。」樂為述己所以為讓,標位二百許語。潘直取錯綜,便成名筆。時人咸云:「若樂不假潘之文,潘不取樂之旨,則無以成斯矣。」

夏侯湛作周詩成,示潘安仁。安仁曰:「此非徒溫雅,乃別見孝悌之性。」潘因此遂作家風詩。

孫子荊除婦服,作詩以示王武子。王曰:「未知文生於情,情生於文。覽之悽然,增伉儷之重。」

太叔廣甚辯給,而摯仲治長於翰墨,俱為列卿。每至公坐,廣談,仲治不能對。退著筆難廣,廣又不能答。

江左殷太常父子,並能言理,亦有辯訥之異。揚州口談至劇,太常輒云:「汝更思吾論。」

庾子嵩作意賦成,從子文康見,問曰:「若有意邪?非賦之所盡;若無意邪?復何所賦?」答曰:「正在有意無意之間。」

郭景純詩云:「林無靜樹,川無停流。」阮孚云:「泓崢蕭瑟,實不可言。每讀此文,輒覺神超形越。」

庾闡始作揚都賦,道溫庾云:「溫挺義之標,庾作民之望。方響則金聲,比德則玉亮。」庾公聞賦成,求看,兼贈貺之。闡更改「望」為「雋」,以「亮」為「潤」云。

孫興公作庾公誄。袁羊曰:「見此張緩。」于時以為名賞。

庾仲初作揚都賦成,以呈庾亮;亮以親族之懷,大為其名價云:「可三二京,四三都。」於此人人競寫,都下紙為之貴。謝太傅云:「不得爾。此是屋下架屋耳!事事擬學,而不免儉狹。」

習鑿齒史才不常,宣武甚器之,未三十,便用為荊州治中。鑿齒謝牋亦云:「不遇明公,荊州老從事耳!」後至都見簡文,返命,宣武問「見相王何如?」答云:「一生不曾見此人!」從此忤旨,出為衡陽郡,性理遂錯。於病中猶作漢晉春秋,品評卓逸。

孫興公云:「三都、二京,五經鼓吹。」

謝太傅問主簿陸退「張憑何以作母誄而不作父誄?」退答曰:「故當是丈夫之德,表於事行;婦人之美,非誄不顯。」

王敬仁年十三,作賢人論。長史送示真長,真長答云:「見敬仁所作論,便足參微言。」

孫興公云:「潘文爛若披錦,無處不善;陸文若排沙簡金,往往見寶。」

簡文稱許掾云:「玄度五言詩,可謂妙絕時人。」

孫興公作天臺賦成,以示范榮期,云:「卿試擲地,要作金石聲。」范曰:「恐子之金石,非宮商中聲!」然每至佳句,輒云:「應是我輩語。」

桓公見謝安石作簡文謚議,看竟,擲與坐上諸客曰:「此是安石碎金。」

袁虎少貧,嘗為人傭載運租。謝鎮西經船行,其夜清風朗月,聞江渚間估客船上有詠詩聲,甚有情致。所誦五言,又其所未嘗聞,嘆美不能已。即遣委曲訊問,乃是袁自詠其所作詠史詩。因此相要,大相賞得。

孫興公云:「潘文淺而淨,陸文深而蕪。」

裴郎作語林,始出,大為遠近所傳。時流年少,無不傳寫,各有一通。載王東亭作經王公酒壚下賦,甚有才情。

謝萬作八賢論,與孫興公往反,小有利鈍。謝後出以示顧君齊,顧曰:「我亦作,知卿當無所名。」

桓宣武命袁彥伯作北征賦,既成,公與時賢共看,咸嗟嘆之。時王珣在坐云:「恨少一句,得寫字足韻,當佳。」袁即於坐攬筆益云:「感不絕於余心,泝流風而獨寫。」公謂王曰:「當今不得不以此事推袁。」

孫興公道:「曹輔佐才如白地明光錦,裁為負版絝,非無文采,酷無裁製。」

袁伯彥作名士傳成,見謝公。公笑曰:「我嘗與諸人道江北事,特作狡獪耳!」彥伯遂以箸書。

王東亭到桓公吏,既伏閣下;桓公令人竊取其白事。東亭即於閣下更作,無復向一字。

桓宣武北征,袁虎時從,被責免官。會須露布文,喚袁倚馬前會作;手不輟筆,俄得七紙,絕可觀。東亭在側,極嘆其才。袁虎云:「當令齒舌間得利。」

袁宏始作東征賦,都不道陶公。胡奴誘之狹室中,臨以白刃,曰:「先公勳業如是!君作東征賦,云何相忽略?」宏窘蹙無計,便答:「我大道公,何以云無?」因誦曰:「精金百鍊,在割能斷。功則治人,職思靖亂。長沙之勳,為史所讚。」

或問顧長康:「君箏賦何如嵇康琴賦?」顧曰:「不賞者作後出相遺,深識者亦以高奇見貴。」

殷仲文天才宏贍,而讀書不甚廣,博亮嘆曰:「若使殷仲文讀書半袁豹,才不減班固。」

羊孚作雪贊云:「資清以化,乘氣以霏。遇象能鮮,即潔成輝。」桓胤遂以書扇。

王孝伯在京行散,至其弟王睹戶前,問:「古詩中何句為最?」睹思未答。孝伯詠「所遇無故物,焉得不速老」,此句為佳。

桓玄嘗登江陵城南樓云:「我今欲為王孝伯作誄。」因吟嘯良久,隨而下筆。一坐之間,誄以之成。

桓玄初并西夏,領荊江二州,二府一國。于時始雪,五處俱賀,五版並入。玄在聽事上,版至即答。版後皆粲然成章,不相揉雜。

桓玄下都,羊孚時為兗州別駕,從京來詣門,牋云:「自頃世故睽離,心事淪薀。明公啟晨光於積晦,澄百流以一源。」桓見牋,馳喚前,云:「子道,子道,來何遲?」即用為記室參軍。孟昶為劉牢之主簿,詣門謝,見云:「羊侯,羊侯,百口賴卿!」



\chapter{方正第五}

陳太丘與友期行,期日中,過中不至,太丘舍去,去後乃至。元方時年七歲,門外戲。客問元方:「尊君在不?」答曰:「待君久不至,已去。」友人便怒曰:「非人哉!與人期行,相委而去。」元方曰:「君與家君期日中,日中不至,則是無信;對子罵父,則是無禮。」友人慚,下車引之。元方入門不顧。

南陽宗世林,魏武同時,而甚薄其為人,不與之交。及魏武作司空,總朝政,從容問宗曰:「可以交未?」答曰:「松柏之志猶存。」世林既以忤旨見疏,位不配德。文帝兄弟每造其門,皆獨拜床下,其見禮如此。

魏文帝受禪,陳群有慼容。帝問曰:「朕應天受命,卿何以不樂?」群曰:「臣與華歆,服膺先朝,今雖欣聖化,猶義形於色。」

郭淮作關中都督,甚得民情,亦屢有戰庸。淮妻,太尉王凌之妹,坐凌事當并誅。使者徵攝甚急,淮使戒裝,克日當發。州府文武及百姓勸淮舉兵,淮不許。至期,遣妻,百姓號泣追呼者數萬人。行數十里,淮乃命左右追夫人還,於是文武奔馳,如徇身首之急。既至,淮與宣帝書曰:「五子哀戀,思念其母,其母既亡,則無五子。五子若殞,亦復無淮。」宣帝乃表,特原淮妻。

諸葛亮之次渭濱,關中震動。魏明帝深懼晉宣王戰,乃遣辛毗為軍司馬。宣王既與亮對渭而陳,亮設誘譎萬方。宣王果大忿,將欲應之以重兵。亮遣間諜覘之;還曰:「有一老夫,毅然仗黃鉞,當軍門立,軍不得出。」亮曰:「此必辛佐治也。」

夏侯玄既被桎梏,時鍾毓為廷尉,鍾會先不與玄相知,因便狎之。玄曰:「雖復刑餘之人,未敢聞命!」考掠初無一言,臨刑東市,顏色不異。

夏侯泰初與廣陵陳本善。本與玄在本母前宴飲,本弟騫行還,徑入,至堂戶。泰初因起曰:「可得同,不可得而雜。」

高貴鄉公薨,內外諠譁。司馬文王問侍中陳泰曰:「何以靜之?」泰云:「唯殺賈充,以謝天下。」文王曰:「可復下此不?」對曰:「但見其上,未見其下。」

和嶠為武帝所親重,語嶠曰:「東宮頃似更成進,卿試往看。」還問「何如?」答云:「皇太子聖質如初。」

諸葛靚後入晉,除大司馬,召不起。以與晉室有讎,常背洛水而坐。與武帝有舊,帝欲見之而無由,乃請諸葛妃呼靚。既來,帝就太妃間相見。禮畢,酒酣,帝曰:「卿故復憶竹馬之好不?」靚曰:「臣不能吞炭漆身,今日復覩聖顏。」因涕泗百行。帝於是慚悔而出。

武帝語和嶠曰:「我欲先痛罵王武子,然後爵之。」嶠曰:「武子雋爽,恐不可屈。」帝遂召武子,苦責之,因曰:「知愧不?」武子曰:「『尺布斗粟』之謠,常為陛下恥之!它人能令疎親,臣不能使親疎,以此愧陛下。」

杜預之荊州,頓七里橋,朝士悉祖。預少賤,好豪俠,不為物所許。楊濟既名氏,雄俊不堪,不坐而去。須臾,和長輿來,問:「楊右衛何在?」客曰:「向來,不坐而去。」長輿曰:「必大夏門下盤馬。」往大夏門,果大閱騎。長輿抱內車,共載歸,坐如初。

杜預拜鎮南將軍,朝士悉至,皆在連榻坐。時亦有裴叔則。羊穉舒後至,曰:「杜元凱乃復連榻坐客!」不坐便去。杜請裴追之,羊去數里住馬,既而俱還杜許。

晉武帝時,荀勖為中書監,和嶠為令。故事,監、令由來共車。嶠性雅正,常疾勖諂諛。後公車來,嶠便登,正向前坐,不復容勖。勖方更覓車,然後得去。監、令各給車自此始。

山公大兒著短帢,車中倚。武帝欲見之,山公不敢辭,問兒,兒不肯行。時論乃云勝山公。

向雄為河內主簿,有公事不及雄,而太守劉淮橫怒,遂與杖遣之。雄後為黃門郎,劉為侍中,初不交言。武帝聞之,敕雄復君臣之好,雄不得已,詣劉,再拜曰:「向受詔而來,而君臣之義絕,何如?」於是即去。武帝聞尚不和,乃怒問雄曰:「我令卿復君臣之好,何以猶絕?」雄曰:「古之君子,進人以禮,退人以禮;今之君子,進人若將加諸厀,退人若將墜諸淵。臣於劉河內,不為戎首,亦已幸甚,安復為君臣之好?」武帝從之。

齊王冏為大司馬輔政,嵇紹為侍中,詣冏咨事。冏設宰會,召葛旟董艾等共論時宜。旟等白冏:「嵇侍中善於絲竹,公可令操之。」遂送樂器。紹推卻不受。冏曰:「今日共為歡,卿何卻邪?」紹曰:「公協輔皇室,令作事可法。紹雖官卑,職備常伯。操絲比竹,蓋樂官之事,不可以先王法服,為伶人之業。今逼高命,不敢苟辭,當釋冠冕,襲私服,此紹之心也。」旟等不自得而退。

盧志於眾坐,問陸士衡:「陸遜、陸抗,是君何物?」答曰:「如卿於盧毓、盧珽。」士龍失色。既出戶,謂兄曰:「何至如此,彼容不相知也?」士衡正色曰:「我父祖名播海內,寧有不知?鬼子敢爾!」議者疑二陸優劣,謝公以此定之。

羊忱性甚貞烈。趙王倫為相國,忱為太傅長史,乃版以參相國軍事。使者卒至,忱深懼豫禍,不暇被馬,於是帖騎而避。使者追之,忱善射,矢左右發,使者不敢進,遂得免。

王太尉不與庾子嵩交,庾卿之不置。王曰:「君不得為爾。」庾曰:「卿自君我,我自卿卿。我自用我法,卿自用卿法。」

阮宣子伐社樹,有人止之。宣子曰:「社而為樹,伐樹則社亡;樹而為社,伐樹則社移矣。」

阮宣子論鬼神有無者,或以人死有鬼。宣子獨以為無,曰:「今見鬼者云,箸生時衣服;若人死有鬼,衣服復有鬼邪?」

元皇帝既登阼,以鄭后之寵,欲舍明帝而立簡文。時議者咸謂:「舍長立少,既於理非倫,且明帝以聰亮英斷,益宜為儲副。」周、王諸公,並苦爭懇切。唯刁玄亮獨欲奉少主,以阿帝旨。元帝便欲施行,慮諸公不奉詔。於是先喚周侯、丞相入,然後欲出詔付刁。周、王既入,始至階頭,帝逆遣傳詔,遏使就東廂。周侯未悟,即卻略下階。丞相披撥傳詔,徑至御床前曰:「不審陛下何以見臣。」帝默然無言,乃探懷中黃紙詔裂擲之。由此皇儲始定。周侯方慨然愧嘆曰:「我常自言勝茂弘,今始知不如也!」

王丞相初在江左,欲結援吳人,請婚陸太尉。對曰:「培塿無松柏,薰蕕不同器。玩雖不才,義不為亂倫之始。」

諸葛恢大女適太尉庾亮兒,次女適徐州刺史羊忱兒。亮子被蘇峻害,改適江虨。恢兒娶鄧攸女。于時謝尚書求其小女婚。恢乃云:「羊、鄧是世婚,江家我顧伊,庾家伊顧我,不能復與謝裒兒婚。」及恢亡,遂婚。於是王右軍往謝家看新婦,猶有恢之遺法,威儀端詳,容服光整。王嘆曰:「我在遣女裁得爾耳!」

周叔治作晉陵太守,周侯、仲智往別。叔治以將別,涕泗不止。仲智恚之曰:「斯人乃婦女,與人別唯啼泣!」便舍去。周侯獨留,與飲酒言話,臨別流涕,撫其背曰:「奴好自愛。」

周伯仁為吏部尚書,在省內夜疾危急。時刁玄亮為尚書令,營救備親好之至。良久小損。明旦,報仲智,仲智狼狽來。始入戶,刁下床對之大泣,說伯仁昨危急之狀。仲智手批之,刁為辟易於戶側。既前,都不問病,直云:「君在中朝,與和長輿齊名,那與佞人刁協有情?」逕便出。

王含作廬江郡,貪濁狼籍。王敦護其兄,故於眾坐稱:「家兄在郡定佳,廬江人士咸稱之!」時何充為敦主簿,在坐,正色曰:「充即廬江人,所聞異於此!」敦默然。旁人為之反側,充晏然,神意自若。

顧孟著嘗以酒勸周伯仁,伯仁不受。顧因移勸柱,而語柱曰:「詎可便作棟梁自遇。」周得之欣然,遂為衿契。

明帝在西堂,會諸公飲酒,未大醉,帝問:「今名臣共集,何如堯、舜?」時周伯仁為僕射,因厲聲曰:「今雖同人主,復那得等於聖治!」帝大怒,還內,作手詔滿一黃紙,遂付廷尉令收,因欲殺之。後數日,詔出周,群臣往省之。周曰:「近知當不死,罪不足至此。」

王大將軍當下,時咸謂無緣爾。伯仁曰:「今主非堯、舜,何能無過?且人臣安得稱兵以向朝廷?處仲狼抗剛愎,王平子何在?」

王敦既下,住船石頭,欲有廢明帝意。賓客盈坐,敦知帝聰明,欲以不孝廢之。每言帝不孝之狀,而皆云溫太真所說。溫嘗為東宮率,後為吾司馬,甚悉之。須臾,溫來,敦便奮其威容,問溫曰:「皇太子作人何似?」溫曰:「小人無以測君子。」敦聲色並厲,欲以威力使從己,乃重問溫:「太子何以稱佳?」溫曰:「鉤深致遠,蓋非淺識所測。然以禮侍親,可稱為孝。」

王大將軍既反,至石頭,周伯仁往見之。謂周曰:「卿何以相負?」對曰:「公戎車犯正,下官忝率六軍,而王師不振,以此負公。」

蘇峻既至石頭,百僚奔散,唯侍中鍾雅獨在帝側。或謂鍾曰:「見可而進,知難而退,古之道也。君性亮直,必不容於寇讎;何不用隨時之宜,而坐待其弊邪?」鍾曰:「國亂不能匡,君危不能濟,而各遜遁以求免,吾懼董狐將執簡而進矣!」

庾公臨去,顧語鍾後事,深以相委。鍾曰:「棟折榱崩,誰之責邪?」庾曰:「今日之事,不容復言,卿當期克復之效耳!」鍾曰:「想足下不愧荀林父耳。」

蘇峻時,孔群在橫塘為匡術所逼。王丞相保存術,因眾坐戲語,令術勸酒,以釋橫塘之憾。群答曰:「德非孔子,厄同匡人。雖陽和布氣,鷹化為鳩,至於識者,猶憎其眼。」

蘇子高事平,王、庾諸公欲用孔廷尉為丹陽。亂離之後,百姓彫弊,孔慨然曰:「昔肅祖臨崩,諸君親升御床,並蒙眷識,共奉遺詔。孔坦疎賤,不在顧命之列。既有艱難,則以微臣為先,今猶俎上腐肉,任人膾截耳!」於是拂衣而去,諸公亦止。

孔車騎與中丞共行,在御道逢匡術,賓從甚盛,因往與車騎共語。中丞初不視,直云:「鷹化為鳩,眾鳥猶惡其眼。」術大怒,便欲刃之。車騎下車,抱術曰:「族弟發狂,卿為我宥之!」始得全首領。

梅頤嘗有惠於陶公。後為豫章太守,有事,王丞相遣收之。侃曰:「天子富於春秋,萬機自諸侯出,王公既得錄,陶公何為不可放?」乃遣人於江口奪之。頤見陶公,拜,陶公止之。頤曰:「梅仲真厀,明日豈可復屈邪?」

王丞相作女伎,施設床席。蔡公先在坐,不說而去,王亦不留。

何次道、庾季堅二人並為元輔。成帝初崩,于時嗣君未定,何欲立嗣子,庾及朝議以外寇方強,嗣子沖幼,乃立康帝。康帝登阼,會群臣,謂何曰:「朕今所以承大業,為誰之議?」何答曰:「陛下龍飛,此是庾冰之功,非臣之力。于時用微臣之議,今不覩盛明之世。」帝有慚色。

江僕射年少,王丞相呼與共棊。王手嘗不如兩道許,而欲敵道戲,試以觀之。江不即下。王曰:「君何以不行?」江曰:「恐不得爾。」傍有客曰:「此年少戲迺不惡。」王徐舉首曰:「此年少非唯圍棊見勝。」

孔君平疾篤,庾司空為會稽,省之,相問訊甚至,為之流涕。庾既下床,孔慨然曰:「大丈夫將終,不問安國寧家之術,迺作兒女子相問!」庾聞,回謝之,請其話言。

桓大司馬詣劉尹,臥不起。桓彎彈彈劉枕,丸迸碎床褥間。劉作色而起曰:「使君如馨地,寧可鬬戰求勝?」桓甚有恨容。

後來年少,多有道深公者。深公謂曰:「黃吻年少,勿為評論宿士。昔嘗與元明二帝、王庾二公周旋。」

王中郎年少時,江虨為僕射領選,欲擬之為尚書郎。有語王者。王曰:「自過江來,尚書郎正用第二人,何得擬我?」江聞而止。

王述轉尚書令,事行便拜。文度曰:「故應讓杜許。」藍田云:「汝謂我堪此不?」文度曰:「何為不堪!但克讓自是美事,恐不可闕。」藍田慨然曰:「既云堪,何為復讓?人言汝勝我,定不如我。」

孫興公作庾公誄,文多託寄之辭。既成,示庾道恩。庾見,慨然送還之,曰:「先君與君,自不至於此。」

王長史求東陽,撫軍不用。後疾篤,臨終,撫軍哀嘆曰:「吾將負仲祖於此,命用之。」長史曰:「人言會稽王痴,真痴。」

劉簡作桓宣武別駕,後為東曹參軍,頗以剛直見疏。嘗聽記,簡都無言。宣武問:「劉東曹何以不下意?」答曰:「會不能用。」宣武亦無怪色。

劉真長、王仲祖共行,日旰未食。有相識小人貽其餐,肴案甚盛,真長辭焉。仲祖曰:「聊以充虛,何苦辭?」真長曰:「小人都不可與作緣。」

王脩齡嘗在東山甚貧乏。陶胡奴為烏程令,送一船米遺之,卻不肯取。直答語:「王脩齡若飢,自當就謝仁祖索食,不須陶胡奴米。」

阮光祿赴山陵,至都,不往殷、劉許,過事便還。諸人相與追之,阮亦知時流必當逐己,乃遄疾而去,至方山不相及。劉尹時為會稽,乃嘆曰:「我入當泊安石渚下耳。不敢復近思曠傍,伊便能捉杖打人,不易。」

王、劉與桓公共至覆舟山看。酒酣後,劉牽腳加桓公頸。桓公甚不堪,舉手撥去。既還,王長史語劉曰:「伊詎可以形色加人不?」

桓公問桓子野:「謝安石料萬石必敗,何以不諫?」子野答曰:「故當出於難犯耳!」桓作色曰:「萬石撓弱凡才,有何嚴顏難犯?」

羅君章曾在人家,主人令與坐上客共語。答曰:「相識已多,不煩復爾。」

韓康伯病,拄杖前庭消搖。見諸謝皆富貴,轟隱交路,嘆曰:「此復何異王莽時?」

王文度為桓公長史時,桓為兒求王女,王許咨藍田。既還,藍田愛念文度,雖長大猶抱著厀上。文度因言桓求己女婚。藍田大怒,排文度下厀曰:「惡見,文度已復痴,畏桓溫面?兵,那可嫁女與之!」文度還報云:「下官家中先得婚處。」桓公曰:「吾知矣,此尊府君不肯耳。」後桓女遂嫁文度兒。

王子敬數歲時,嘗看諸門生樗蒲。見有勝負,因曰:「南風不競。」門生輩輕其小兒,迺曰:「此郎亦管中窺豹,時見一斑。」子敬瞋目曰:「遠慚荀奉倩,近愧劉真長!」遂拂衣而去。

謝公聞羊綏佳,致意令來,終不肯詣。後綏為太學博士,因事見謝公,公即取以為主簿。

王右軍與謝公詣阮公,至門語謝:「故當共推主人。」謝曰:「推人正自難。」

太極殿始成,王子敬時為謝公長史,謝送版,使王題之。王有不平色,語信云:「可擲箸門外。」謝後見王曰:「題之上殿何若?昔魏朝韋誕諸人,亦自為也。」王曰:「魏阼所以不長。」謝以為名言。

王恭欲請江盧奴為長史,晨往詣江,江猶在帳中。王坐,不敢即言。良久乃得及,江不應。直喚人取酒,自飲一盌,又不與王。王且笑且言:「那得獨飲?」江云:「卿亦復須邪?」更使酌與王,王飲酒畢,因得自解去。未出戶,江嘆曰:「人自量,固為難。」

孝武問王爽:「卿何如卿兄。」王答曰:「風流秀出,臣不如恭,忠孝亦何可以假人!」

王爽與司馬太傅飲酒。太傅醉,呼王為「小子。」王曰:「亡祖長史,與簡文皇帝為布衣之交。亡姑、亡姊,伉儷二宮。何小子之有?」

張玄與王建武先不相識,後遇於范豫章許,范令二人共語。張因正坐斂衽,王孰視良久,不對。張大失望,便去。范苦譬留之,遂不肯住。范是王之舅,乃讓王曰:「張玄,吳士之秀,亦見遇於時,而使至於此,深不可解。」王笑曰:「張祖希若欲相識,自應見詣。」范馳報張,張便束帶造之。遂舉觴對語,賓主無愧色。



\chapter{雅量第六}

豫章太守顧邵,是雍之子。邵在郡卒,雍盛集僚屬,自圍棊。外啟信至,而無兒書,雖神氣不變,而心了其故。以爪掐掌,血流沾褥。賓客既散,方嘆曰:「已無延陵之高,豈可有喪明之責?」於是豁情散哀,顏色自若。

嵇中散臨刑東市,神氣不變。索琴彈之,奏廣陵散。曲終,曰:「袁孝尼嘗請學此散,吾靳,固不與,廣陵散於今絕矣!」太學生三千人上書,請以為師,不許。文王亦尋悔焉。

夏侯太初嘗倚柱作書。時大雨,霹靂破所倚柱,衣服焦然,神色無變,書亦如故。賓客左右,皆跌蕩不得住。

王戎七歲,嘗與諸小兒遊,看道邊李樹多子折枝,諸兒競走取之,唯戎不動。人問之,答曰:「樹在道邊而多子,此必苦李。」取之信然。

魏明帝於宣武場上斷虎爪牙,縱百姓觀之。王戎七歲,亦往看。虎承閒攀欄而吼,其聲震地,觀者無不辟易顛仆;戎湛然不動,了無恐色。

王戎為侍中,南郡太守劉肇遺筒中箋布五端,戎雖不受,厚報其書。

裴叔則被收,神氣無變,舉止自若。求紙筆作書。書成,救者多,乃得免。後位儀同三司。

王夷甫嘗屬族人事,經時未行,遇於一處飲燕,因語之曰:「近屬尊事,那得不行?」族人大怒,便舉樏擲其面。夷甫都無言,盥洗畢,牽王丞相臂,與共載去。在車中照鏡語丞相曰:「汝看我眼光,迺出牛背上。」

裴遐在周馥所,馥設主人。遐與人圍棊,馥司馬行酒。遐正戲,不時為飲。司馬恚,因曳遐墜地。遐還坐,舉止如常,顏色不變,復戲如故。王夷甫問遐:「當時何得顏色不異?」答曰:「直是闇當故耳。」

劉慶孫在太傅府,于時人士,多為所構。唯庾子嵩縱心事外,無迹可閒。後以其性儉家富,說太傅令換千萬,冀其有吝,於此可乘。太傅於眾坐中問庾,庾時頹然已醉,幘墜几上,以頭就穿取,徐答云:「下官家故可有兩娑千萬,隨公所取。」於是乃服。後有人向庾道此,庾曰:「可謂以小人之慮,度君子之心。」

王夷甫與裴景聲志好不同。景聲惡欲取之,卒不能回。乃故詣王,肆言極罵,要王答己,欲以分謗。王不為動色,徐曰:「白眼兒遂作。」

王夷甫長裴成公四歲,不與相知。時共集一處,皆當時名士,謂王曰:「裴令令望何足計!」王便卿裴。裴曰:「自可全君雅志。」

有往來者云庾公有東下意,或謂王公可潛稍嚴,以備不虞。王公曰:「我與元規雖俱王臣,本懷布衣之好,若其欲來,吾角巾徑還烏衣。」

王丞相主簿欲檢校帳下,公語主簿,欲與主簿周旋,無為知人几案間事。

祖士少好財,阮遙集好屐並恒自經營。同是一累而未判其得失。人有詣祖,見料視財物。客至,屏當未盡,餘兩小簏箸背後,傾身障之,意未能平。或有詣阮,見自吹火蠟屐,因嘆曰:「未知一生當箸幾量屐?」神色閑暢。於是勝負始分。

許侍中、顧司空俱作丞相從事,爾時已被遇,遊宴集聚,略無不同。嘗夜至丞相許戲,二人歡極,丞相便命使入己帳眠。顧至曉回轉,不得快孰。許上床便咍臺大鼾。丞相顧諸客曰:「此中亦難得眠處。」

庾太尉風儀偉長,不輕舉止,時人皆以為假。亮有大兒數歲,雅重之質,便自如此,人知是天性。溫太真嘗隱幔怛之,此兒神色恬然,乃徐跪曰:「君侯何以為此?」論者謂不減亮。蘇峻時遇害。或云:「見阿恭,知元規非假。」

褚公於章安令遷太尉記室參軍,名字已顯而位微,人未多識。公東出,乘估客船,送故吏數人投錢唐亭住。爾時吳興沈充為縣令,當送客過浙江,客出,亭吏驅公移牛屋下。潮水至,沈令起彷徨,問:「牛屋下是何物人?」吏云:「昨有一傖父來寄亭中,有尊貴客,權移之。」令有酒色,因遙問「傖父欲食\ext{䴵}不?姓何等?可共語。」褚因舉手答曰:「河南褚季野。」遠近久承公名,令於是大遽,不敢移公,便於牛屋下脩刺詣公。更宰殺為饌,具於公前,鞭撻亭吏,欲以謝慚。公與之酌宴,言色無異,狀如不覺。令送公至界。

郗太傅在京口,遣門生與王丞相書,求女壻。丞相語郗信:「君往東廂,任意選之。」門生歸,白郗曰:「王家諸郎,亦皆可嘉,聞來覓婿,咸自矜持;唯有一郎,在東床上坦腹臥,如不聞。」郗公云:「正此好!」訪之,乃是逸少,因嫁女與焉。

過江初,拜官,輿飾供饌。羊曼拜丹陽尹,客來蚤者,並得佳設。日晏漸罄,不復及精,隨客早晚,不問貴賤。羊固拜臨海,竟日皆美供。雖晚至,亦獲盛饌。時論以固之豐華,不如曼之真率。

周仲智飲酒醉,瞋目還面謂伯仁曰:「君才不如弟,而橫得重名!」須臾,舉蠟燭火擲伯仁。伯仁笑曰:「阿奴火攻,固出下策耳!」

顧和始為楊州從事。月旦當朝,未入頃,停車州門外。周侯詣丞相,歷和車邊。和覓蝨,夷然不動。周既過,反還,指顧心曰:「此中何所有?」顧搏蝨如故,徐應曰:「此中最是難測地。」周侯既入,語丞相曰:「卿州吏中有一令僕才。」

庾太尉與蘇峻戰,敗,率左右十餘人,乘小船西奔。亂兵相剝掠,射誤中柂工,應弦而倒。舉船上咸失色分散,亮不動容,徐曰:「此手那可使箸賊!」眾迺安。

庾小征西嘗出未還。婦母阮是劉萬安妻,與女上安陵城樓上。俄頃翼歸,策良馬,盛輿衛。阮語女:「聞庾郎能騎,我何由得見?」婦告翼,翼便為於道開鹵簿盤馬,始兩轉,墜馬墮地,意色自若。

宣武與簡文、太宰共載,密令人在輿前後鳴鼓大叫。鹵簿中驚擾,太宰惶怖求下輿。顧看簡文,穆然清恬。宣武語人曰:「朝廷間故復有此賢。」

王劭、王薈共詣宣武,正值收庾希家。薈不自安,逡巡欲去;劭堅坐不動,待收信還,得不定迺出。論者以劭為優。

桓宣武與郗超議芟夷朝臣,條牒既定,其夜同宿。明晨起,呼謝安、王坦之入,擲疏示之。郗猶在帳內,謝都無言,王直擲還,云:多!宣武取筆欲除,郗不覺竊從帳中與宣武言。謝含笑曰:「郗生可謂入幕賓也。」

謝太傅盤桓東山時,與孫興公諸人汎海戲。風起浪涌,孫、王諸人色並遽,便唱使還。太傅神情方王,吟嘯不言。舟人以公貌閑意說,猶去不止。既風轉急,浪猛,諸人皆諠動不坐。公徐云:「如此,將無歸!」眾人即承響而回。於是審其量,足以鎮安朝野。

桓公伏甲設饌,廣延朝士,因此欲誅謝安、王坦之。王甚遽,問謝曰:「當作何計?」謝神意不變,謂文度曰:「晉阼存亡,在此一行。」相與俱前。王之恐狀,轉見於色。謝之寬容,愈表於貌。望階趨席,方作洛生詠,諷「浩浩洪流」。桓憚其曠遠,乃趣解兵。王、謝舊齊名,於此始判優劣。

謝太傅與王文度共詣郗超,日旰未得前,王便欲去。謝曰:「不能為性命忍俄頃?」

支道林還東,時賢並送於征虜亭。蔡子叔前至,坐近林公。謝萬石後來,坐小遠。蔡暫起,謝移就其處。蔡還,見謝在焉,因合褥舉謝擲地,自復坐。謝冠幘傾脫,乃徐起振衣就席,神意甚平,不覺瞋沮。坐定,謂蔡曰:「卿奇人,殆壞我面。」蔡答曰:「我本不為卿面作計。」其後,二人俱不介意。

郗嘉賓欽崇釋道安德問,餉米千斛,修書累紙,意寄殷勤。道安答直云:「損米。」愈覺有待之為煩。

謝安南免吏部尚書還東,謝太傅赴桓公司馬出西,相遇破岡。既當遠別,遂停三日共語。太傅欲慰其失官,安南輒引以它端。雖信宿中塗,竟不言及此事。太傅深恨在心未盡,謂同舟曰:「謝奉故是奇士。」

戴公從東出,謝太傅往看之。謝本輕戴,見但與論琴書。戴既無吝色,而談琴書愈妙。謝悠然知其量。

謝公與人圍棊,俄而謝玄淮上信至。看書竟,默然無言,徐向局。客問淮上利害?答曰:「小兒輩大破賊。」意色舉止,不異於常。

王子猷、子敬曾俱坐一室,上忽發火,子猷遽走避,不惶取屐;子敬神色恬然,徐喚左右,扶憑而出,不異平常。世以此定二王神宇。

符堅遊魂近境,謝太傅謂子敬曰:「可將當軸,了其此處。」

王僧彌、謝車騎共王小奴許集。僧彌舉酒勸謝云:「奉使君一觴。」謝曰:「可爾。」僧彌勃然起,作色曰:「汝故是吳興溪中釣碣耳!何敢譸張!」謝徐撫掌而笑曰:「衛軍,僧彌殊不肅省,乃侵陵上國也。」

王東亭為桓宣武主簿,既承藉,有美譽,公甚欲其人地為一府之望。初,見謝失儀,而神色自若。坐上賓客即相貶笑。公曰:「不然,觀其情貌,必自不凡。吾當試之。」後因月朝閣下伏,公於內走馬直出突之,左右皆宕仆,而王不動。名價於是大重,咸云「是公輔器也」。

太元末,長星見,孝武心甚惡之。夜,華林園中飲酒,舉桮屬星云:「長星!勸爾一桮酒。自古何時有萬歲天子?」

殷荊州有所識,作賦,是束皙慢戲之流。殷甚以為有才,語王恭:「適見新文,甚可觀。」便於手巾函中出之。王讀,殷笑之不自勝。王看竟,既不笑,亦不言好惡,但以如意帖之而已。殷悵然自失。

羊綏第二子孚,少有雋才,與謝益壽相好,嘗蚤往謝許,未食。俄而王齊、王暏來。既先不相識,王向席有不說色,欲使羊去。羊了不眄,唯腳委几上,詠矚自若。謝與王敘寒溫數語畢,還與羊談賞,王方悟其奇,乃合共語。須臾食下,二王都不得餐,唯屬羊不暇。羊不大應對之,而盛進食,食畢便退。遂苦相留,羊義不住,直云:「向者不得從命,中國尚虛。」二王是孝伯兩弟。



\chapter{識鑒第七}

\href{https://zh.wikipedia.org/wiki/zh:\%E6\%9B\%B9\%E6\%93\%8D}\ext{曹公}少時見喬玄,玄謂曰:「天下方亂,群雄虎爭,撥而理之,非君乎?然君實亂世之英雄,治世之姦賊。恨吾老矣,不見君富貴,當以子孫相累。」

曹公問裴潛曰:「卿昔與\href{https://zh.wikipedia.org/wiki/zh:\%E5\%8A\%89\%E5\%82\%99}\ext{劉備}共在荊州,卿以備才如何?」潛曰:「使居中國,能亂人,不能為治。若乘邊守險,足為一方之主。」

何晏、鄧颺、夏侯玄並求傅嘏交,而嘏終不許。諸人乃因荀粲說合之,謂嘏曰:「夏侯太初一時之傑士,虛心於子,而卿意懷不可,交合則好成,不合則致隟。二賢若穆,則國之休,此藺相如所以下廉頗也。」傅曰:「夏侯太初,志大心勞,能合虛譽,誠所謂利口覆國之人。何晏、鄧颺有為而躁,博而寡要,外好利而內無關籥,貴同惡異,多言而妬前。多言多釁,妬前無親。以吾觀之:此三賢者,皆敗德之人爾!遠之猶恐罹禍,況可親之邪?」後皆如其言。

\href{https://zh.wikipedia.org/wiki/zh:\%E6\%99\%89\%E6\%AD\%A6\%E5\%B8\%9D}{晉武帝}講武於宣武場,帝欲偃武修文,親自臨幸,悉召群臣。山公謂不宜爾,因與諸尚書言孫、吳用兵本意。遂究論,舉坐無不咨嗟。皆曰:「山少傅乃天下名言。」後諸王驕汰,輕遘禍難,於是寇盜處處蟻合,郡國多以無備,不能制服,遂漸熾盛,皆如公言。時人以謂山濤不學孫、吳,而闇與之理會。王夷甫亦嘆云:「公闇與道合。」

王夷甫父乂為平北將軍,有公事,使行人論不得。時夷甫在京師,命駕見僕射羊祜、尚書山濤。夷甫時總角,姿才秀異,敘致既快,事加有理,濤甚奇之。既退,看之不輟,乃嘆曰:「生兒不當如王夷甫邪?」羊祜曰:「亂天下者,必此子也!」

潘陽仲見王敦少時,謂曰:「君『蜂目』已露,但『豺聲』未振耳。必能食人,亦當為人所食。」

\href{https://zh.wikipedia.org/wiki/zh:\%E7\%9F\%B3\%E5\%8B\%92}\ext{石勒}不知書,使人讀漢書。聞酈食其勸立六國後,刻印將授之,大驚曰:「此法當失,云何得遂有天下?」至\href{https://zh.wikipedia.org/wiki/zh:\%E5\%BC\%B5\%E8\%89\%AF}\ext{留侯}諫,迺曰:「賴有此耳!」

衛玠年五歲,神衿可愛。祖太保曰:「此兒有異,顧吾老,不見其大耳!」

劉越石云:「華彥夏識能不足,彊果有餘。」

張季鷹辟\href{https://zh.wikipedia.org/wiki/zh:\%E5\%8F\%B8\%E9\%A6\%AC\%E5\%86\%8F}\ext{齊王}東曹掾,在洛見秋風起,因思吳中菰菜羹、鱸魚膾,曰:「人生貴得適意爾,何能羈宦數千里以要名爵!」遂命駕便歸。俄而齊王敗,時人皆謂為見機。

諸葛道明初過江左,自名道明,名亞王、庾之下。先為臨沂令,丞相謂曰:「明府當為黑頭公。」

王平子素不知眉子,曰:「志大其量,終當死塢壁間。」

王大將軍始下,楊朗苦諫不從,遂為王致力,乘「中鳴雲露車」逕前曰:「聽下官鼓音,一進而捷。」王先把其手曰:「事克,當相用為荊州。」既而忘之,以為南郡。王敗後,明帝收朗,欲殺之。帝尋崩,得免。後兼三公,署數十人為官屬。此諸人當時並無名,後皆被知遇,于時稱其知人。

周伯仁母冬至舉酒賜三子曰:「吾本謂度江託足無所。爾家有相,爾等並羅列吾前,復何憂?」周嵩起,長跪而泣曰:「不如阿母言。伯仁為人志大而才短,名重而識闇,好乘人之弊,此非自全之道。嵩性狼抗,亦不容於世。唯阿奴碌碌,當在阿母目下耳!」

王大將軍既亡,王應欲投世儒,世儒為江州。王含欲投王舒,舒為荊州。含語應曰:「大將軍平素與江州云何?而汝欲歸之。」應曰:「此迺所以宜往也。江州當人彊盛時,能抗同異,此非常人所行。及覩衰厄,必興愍惻。荊州守文,豈能作意表行事?」含不從,遂共投舒。舒果沈含父子于江。彬聞應當來,密具船以待之,竟不得來,深以為恨。

武昌孟嘉作庾太尉州從事,已知名。褚太傅有知人鑒,罷豫章還,過武昌,問庾曰:「聞孟從事佳,今在此不?」庾云:「卿自求之。」褚眄睞良久,指嘉曰:「此君小異,得無是乎?」庾大笑曰:「然!」于時既嘆褚之默識,又欣嘉之見賞。

戴安道年十餘歲,在瓦官寺畫。王長史見之曰:「此童非徒能畫,亦終當致名。恨吾老,不見其盛時耳!」

王仲祖、謝仁祖、劉真長俱至丹陽墓所省殷揚州,殊有確然之志。既反,王、謝相謂曰:「淵源不起,當如蒼生何?」深為憂嘆。劉曰:「卿諸人真憂淵源不起邪?」

小庾臨終,自表以子園客為代。朝廷慮其不從命,未知所遣,乃共議用桓溫。劉尹曰:「使伊去,必能克定西楚,然恐不可復制。」

桓公將伐蜀,在事諸賢咸以李勢在蜀既久,承藉累葉,且形據上流,三峽未易可克。唯劉尹云:「伊必能克蜀。觀其蒲博,不必得,則不為。」

謝公在東山畜妓,簡文曰:「安石必出。既與人同樂,亦不得不與人同憂。」

郗超與謝玄不善。符堅將問晉鼎,既已狼噬梁、岐,又虎視淮陰矣。于時朝議遣玄北討,人間頗有異同之論。唯超曰:「是必濟事。吾昔嘗與共在桓宣武府,見使才皆盡,雖履屐之間,亦得其任。以此推之,容必能立勳。」\href{https://zh.wikipedia.org/wiki/zh:\%E6\%B7\%9D\%E6\%B0\%B4\%E4\%B9\%8B\%E6\%88\%B0}{元功既舉},時人咸嘆超之先覺,又重其不以愛憎匿善。

韓康伯與謝玄亦無深好。玄北征後,巷議疑其不振。康伯曰:「此人好名,必能戰。」玄聞之甚忿,常於眾中厲色曰:「丈夫提千兵,入死地,以事君親,故發,不得復云為名!」

褚期生少時,謝公甚知之,恆云:「褚期生若不佳者,僕不復相士。」

郗超與傅瑗周旋,瑗見其二子並總髮。超觀之良久,謂瑗曰:「小者才名皆勝,然保卿家,終當在兄。」即傅亮兄弟也。

王恭隨父在會稽,王大自都來拜墓。恭暫往墓下看之,二人素善,遂十餘日方還。父問恭:「何故多日?」對曰:「與阿大語,蟬連不得歸。」因語之曰:「恐阿大非爾之友。」終乖愛好,果如其言。

車胤父作南平郡功曹,太守王胡之避司馬無忌之難,置郡于酆陰。是時胤十餘歲,胡之每出,嘗於籬中見而異焉。謂胤父曰:「此兒當致高名。」後遊集,恆命之。胤長,又為桓宣武所知。清通於多士之世,官至選曹尚書。

王忱死,西鎮未定,朝貴人人有望。時殷仲堪在門下,雖居機要,資名輕小,人情未以方嶽相許。晉孝武欲拔親近腹心,遂以殷為荊州。事定,詔未出。王珣問殷曰:「陝西何故未有處分?」殷曰:「已有人。」王歷問公卿,咸云「非」。王自計才地必應在己,復問:「非我邪?」殷曰:「亦似非。」其夜詔出用殷。王語所親曰:「豈有黃門郎而受如此任?仲堪此舉迺是國之亡徵。」



\chapter{賞譽第八}

陳仲舉嘗嘆曰:「若周子居者,真治國之器!譬諸寶劍,則世之干將。」

世目李元禮:「謖謖如勁松下風。」

謝子微見許子將兄弟曰:「平輿之淵,有二龍焉。」見許子政弱冠之時,嘆曰:「若許子政者,有榦國之器。正色忠謇,則陳仲舉之匹;伐惡退不肖,范孟博之風。」

公孫度目邴原:所謂雲中白鶴,非燕雀之網所能羅也。

鍾士季目王安豐:阿戎了了解人意。謂裴公之談,經日不竭。吏部郎闕,文帝問其人於鍾會。會曰:「裴楷清通,王戎簡要,皆其選也。」於是用裴。

王濬沖、裴叔則二人,總角詣鍾士季。須臾去後,客問鍾曰:「向二童何如?」鍾曰:「裴楷清通,王戎簡要。後二十年,此二賢當為吏部尚書,冀爾時天下無滯才。」

諺曰:「後來領袖有裴秀。」

裴令公目夏侯太初:「肅肅如入廊廟中,不脩敬而人自敬。」一曰:「如入宗廟,琅琅但見禮樂器。見鍾士季,如觀武庫,但覩矛戟。見傅蘭碩,江廧靡所不有。見山巨源,如登山臨下,幽然深遠。」

羊公還洛,郭奕為野王令。羊至界,遣人要之。郭便自往。既見,嘆曰:「羊叔子何必減郭太業!」復往羊許,小悉還,又嘆曰:「羊叔子去人遠矣!」羊既去,郭送之彌日,一舉數百里,遂以出境免官。復嘆曰:「羊叔子何必減顏子!」

王戎目山巨源:「如璞玉渾金,人皆欽其寶,莫知名其器。」

羊長和父繇,與太傅祜同堂相善,仕至車騎掾。蚤卒。長和兄弟五人,幼孤。祜來哭,見長和哀容舉止,宛若成人,迺嘆曰:「從兄不亡矣!」

山公舉阮咸為吏部郎,目曰:「清真寡欲,萬物不能移也。」

王戎目阮文業:「清倫有鑒識,漢元以來,未有此人。」

武元夏目裴、王曰:「戎尚約,楷清通。」

庾子嵩目和嶠:「森森如千丈松,雖磊砢有節目,施之大廈,有棟梁之用。」

王戎云:「太尉神姿高徹,如瑤林瓊樹,自然是風塵外物。」

王汝南既除所生服,遂停墓所。兄子濟每來拜墓,略不過叔,叔亦不候。濟脫時過,止寒溫而已。後聊試問近事,答對甚有音辭,出濟意外,濟極惋愕。仍與語,轉造精微。濟先略無子姪之敬,既聞其言,不覺懍然,心形俱肅。遂留共語,彌日累夜。濟雖雋爽,自視缺然,乃喟然嘆曰:「家有名士,三十年而不知!」濟去,叔送至門。濟從騎有一馬,絕難乘,少能騎者。濟聊問叔:「好騎乘不?」曰:「亦好爾。」濟又使騎難乘馬,叔姿形既妙,回策如縈,名騎無以過之。濟益嘆其難測,非復一事。既還,渾問濟:「何以暫行累日?」濟曰:「始得一叔。」渾問其故?濟具嘆述如此。渾曰:「何如我?」濟曰:「濟以上人。」武帝每見濟,輒以湛調之曰:「卿家痴叔死未?」濟常無以答。既而得叔,後武帝又問如前,濟曰:「臣叔不痴。」稱其實美。帝曰:「誰比?」濟曰:「山濤以下,魏舒以上。」於是顯名。年二十八,始宦。

裴僕射時人謂為言談之林藪。

張華見褚陶,語陸平原曰:「君兄弟龍躍雲津,顧彥先鳳鳴朝陽。謂東南之寶已盡,不意復見褚生。」陸曰:「公未覩不鳴不躍者耳!」

有問秀才:「吳舊姓何如?」答曰:「吳府君聖王之老成,明時之雋乂。朱永長理物之至德,清選之高望。嚴仲弼九皋之鳴鶴,空谷之白駒。顧彥先八音之琴瑟,五色之龍章。張威伯歲寒之茂松,幽夜之逸光。陸士衡、士龍鴻鵠之裴回,懸鼓之待槌。凡此諸君:以洪筆為鉏耒,以紙札為良田。以玄默為稼穡,以義理為豐年。以談論為英華,以忠恕為珍寶。著文章為錦繡,蘊五經為繒帛。坐謙虛為席薦,張義讓為帷幙。行仁義為室宇,修道德為廣宅。」

人問王夷甫:「山巨源義理何如?是誰輩?」王曰:「此人初不肯以談自居,然不讀老、莊,時聞其詠,往往與其旨合。」

洛中雅雅有三嘏:劉粹字純嘏,宏字終嘏,漠字沖嘏,是親兄弟。王安豐甥,並是王安豐女壻。宏,真長祖也。洛中錚錚馮惠卿,名蓀,是播子。蓀與邢喬俱司徒李胤外孫,及胤子順並知名。時稱:「馮才清,李才明,純粹邢。」

衛伯玉為尚書令,見樂廣與中朝名士談議,奇之曰:「自昔諸人沒已來,常恐微言將絕。今乃復聞斯言於君矣!」命子弟造之曰:「此人,人之水鏡也,見之若披雲霧覩青天。」

王太尉曰:「見裴令公精明朗然,籠蓋人上,非凡識也。若死而可作,當與之同歸。」或云王戎語。

王夷甫自嘆:「我與樂令談,未嘗不覺我言為煩。」

郭子玄有雋才,能言老、莊。庾敳嘗稱之,每曰:「郭子玄何必減庾子嵩!」

王平子目太尉:「阿兄形似道,而神鋒太雋。」太尉答曰:「誠不如卿落落穆穆。」

太傅府有三才:劉慶孫長才,潘陽仲大才,裴景聲清才。

林下諸賢,各有雋才子。籍子渾,器量弘曠。康子紹,清遠雅正。濤子簡,疏通高素。咸子瞻,虛夷有遠志。瞻弟孚,爽朗多所遺。秀子純、悌,並令淑有清流。戎子萬子,有大成之風,苗而不秀。唯伶子無聞。凡此諸子,唯瞻為冠,紹、簡亦見重當世。

庾子躬有廢疾,甚知名。家在城西,號曰城西公府。

王夷甫語樂令:「名士無多人,故當容平子知。」

王太尉云:「郭子玄語議如懸河寫水,注而不竭。」

司馬太傅府多名士,一時雋異。庾文康云:「見子嵩在其中,常自神王。」

太傅東海王鎮許昌,以王安期為記室參軍,雅相知重。敕世子毗曰:「夫學之所益者淺,體之所安者深。閑習禮度,不如式瞻儀形。諷味遺言,不如親承音旨。王參軍人倫之表,汝其師之!」或曰:「王、趙、鄧三參軍,人倫之表,汝其師之!」謂安期、鄧伯道、趙穆也。袁宏作名士傳直云王參軍。或云,趙家先猶有此本。

庾太尉少為王眉子所知。庾過江,嘆王曰:「庇其宇下,使人忘寒暑。」

謝幼輿曰:「友人王眉子清通簡暢,嵇延祖弘雅劭長,董仲道卓犖有致度。」

王公目太尉:「巖巖清峙,壁立千仞。」

庾太尉在洛下,問訊中郎。中郎留之云:「諸人當來。」尋溫元甫、劉王喬、裴叔則俱至,酬酢終日。庾公猶憶劉、裴之才雋,元甫之清中。

蔡司徒在洛,見陸機兄弟住參佐廨中,三間瓦屋,士龍住東頭,士衡住西頭。士龍為人,文弱可愛。士衡長七尺餘,聲作鍾聲,言多忼慨。

王長史是庾子躬外孫,丞相目子躬云:「入理泓然,我已上人。」

庾太尉目庾中郎:家從談談之許。

庾公目中郎:「神氣融散,差如得上。」

劉琨稱祖車騎為朗詣,曰:「少為王敦所嘆。」

時人目庾中郎:「善於託大,長於自藏。」

王平子邁世有雋才,少所推服。每聞衛玠言,輒嘆息絕倒。

王大將軍與元皇表云:「舒風概簡正,允作雅人,自多於邃。最是臣少所知拔。中間夷甫、澄見語:『卿知處明、茂弘。茂弘已有令名,真副卿清論;處明親疎無知之者,吾常以卿言為意,殊未有得,恐已悔之?』臣慨然曰:『君以此試,頃來始乃有稱之者。』言常人正自患知之使過,不知使負實。」

周侯於荊州敗績,還,未得用。王丞相與人書曰:「雅流弘器,何可得遺?」

時人欲題目高坐而未能。桓廷尉以問周侯,周侯曰:「可謂卓朗。」桓公曰:「精神淵箸。」

王大將軍稱其兒云:「其神候似欲可。」

卞令目叔向:「朗朗如百間屋。」

王敦為大將軍,鎮豫章。衛玠避亂,從洛投敦,相見欣然,談話彌日。于時謝鯤為長史,敦謂鯤曰:「不意永嘉之中,復聞正始之音。阿平若在,當復絕倒。」

王平子與人書,稱其兒:「風氣日上,足散人懷。」

胡毋彥國吐佳言如屑,後進領袖。

王丞相云:「刁玄亮之察察,戴若思之巖巖,卞望之之峯距。」

大將軍語右軍:「汝是我佳子弟,當不減阮主簿。」

世目周侯:嶷如斷山。

王丞相招祖約夜語,至曉不眠。明旦有客,公頭鬢未理,亦小倦。客曰:「公昨如是,似失眠。」公曰:「昨與士少語,遂使人忘疲。」

王大將軍與丞相書,稱楊朗曰:「世彥識器理致,才隱明斷,既為國器,且是楊侯淮之子。位望殊為陵遲,卿亦足與之處。」

何次道往丞相許,丞相以麈尾指坐呼何共坐曰:「來!來!此是君坐。」

丞相治楊州廨舍,按行而言曰:「我正為次道治此爾!」何少為王公所重,故屢發此嘆。

王丞相拜司徒而嘆曰:「劉王喬若過江,我不獨拜公。」

王藍田為人晚成,時人乃謂之痴;王丞相以其東海子,辟為掾。常集聚,王公每發言,衆人競讚之。述於末坐曰:「主非堯、舜,何得事事皆是?」丞相甚相歎賞。

世目楊朗:「沈審經斷。」蔡司徒云:「若使中朝不亂,楊氏作公方未已。」謝公云:「朗是大才。」

劉萬安即道真從子。庾公所謂「灼然玉舉」。又云:「千人亦見,百人亦見。」

庾公為護軍,屬桓廷尉覓一佳吏,乃經年。桓後遇見徐寧而知之,遂致於庾公曰:「人所應有,其不必有;人所應無,己不必無。真海岱清士。」

桓茂倫云:「褚季野皮裡陽秋。」謂其裁中也。

何次道嘗送東人,瞻望見賈寧在後輪中,曰:「此人不死,終為諸侯上客。」

杜弘治墓崩,哀容不稱。庾公顧謂諸客曰:「弘治至羸,不可以致哀。」又曰:「弘治哭不可哀。」

世稱「庾文康為豐年玉,稺恭為荒年穀」。庾家論云是文康稱「恭為荒年穀,庾長仁為豐年玉。」

世目「杜弘治標鮮,季野穆少」。

有人目杜弘治:「標鮮清令,盛德之風,可樂詠也。」

庾公云:「逸少國舉。」故庾倪為碑文云:「拔萃國舉。」

庾稺恭與桓溫書,稱「劉道生日夕在事,大小殊快。義懷通樂,既佳,且足作友,正實良器,推此與君,同濟艱不者也。」

王藍田拜揚州,主簿請諱,教云:「亡祖先君,名播海內,遠近所知。內諱不出於外,餘無所諱。」

蕭中郎,孫丞公婦父。劉尹在撫軍坐,時擬為太常,劉尹云:「蕭祖周不知便可作三公不?自此以還,無所不堪。」

謝太傅未冠,始出西,詣王長史,清言良久。去後,苟子問曰:「向客何如尊?」長史曰:「向客亹亹,為來逼人。」

王右軍語劉尹:「故當共推安石。」劉尹曰:「若安石東山志立,當與天下共推之。」

謝公稱藍田:「掇皮皆真。」

桓溫行經王敦墓邊過,望之云:「可兒!可兒!」

殷中軍道王右軍云:「逸少清貴人。吾於之甚至,一時無所後。」

王仲祖稱殷淵源:「非以長勝人,處長亦勝人。」

王司州與殷中軍語,嘆云:「己之府奧,蚤已傾寫而見,殷陳勢浩汗,眾源未可得測。」

王長史謂林公:「真長可謂金玉滿堂。」林公曰:「金玉滿堂,復何為簡選?」王曰:「非為簡選,直致言處自寡耳。」

王長史道江道羣:「人可應有,乃不必有;人可應無,己必無。」

會稽孔沈、魏顗、虞球、虞存、謝奉,並是四族之雋,于時之傑。孫興公目之曰:「沈為孔家金,顗為魏家玉,虞為長、琳宗,謝為弘道伏。」

王仲祖、劉真長造殷中軍談,談竟,俱載去。劉謂王曰:「淵源真可。」王曰:「卿故墮其雲霧中。」

劉尹每稱王長史云:「性至通,而自然有節。」

王右軍道謝萬石「在林澤中,為自遒上」。嘆林公「器朗神雋」。道祖士少「風領毛骨,恐沒世不復見如此人」。道劉真長「標雲柯而不扶疎」。

簡文目庾赤玉:「省率治除。」謝仁祖云:「庾赤玉胷中無宿物。」

殷中軍道韓太常曰:「康伯少自標置,居然是出羣器。及其發言遣辭,往往有情致。」

簡文道王懷祖:「才既不長,於榮利又不淡;直以真率少許,便足對人多多許。」

林公謂王右軍云:「長史作數百語,無非德音,如恨不苦。」王曰:「長史自不欲苦物。」

殷中軍與人書,道謝萬「文理轉遒,成殊不易」。

王長史云:「江思悛思懷所通,不翅儒域。」

許玄度送母,始出都,人問劉尹:「玄度定稱所聞不?」劉曰:「才情過於所聞。」

阮光祿云:「王家有三年少:右軍、安期、長豫。」

謝公道豫章:「若遇七賢,必自把臂入林。」

王長史嘆林公:「尋微之功,不減輔嗣。」

殷淵源在墓所幾十年。于時朝野以擬管、葛,起不起,以卜江左興亡。

殷中軍道右軍:「清鑒貴要。」

謝太傅為桓公司馬,桓詣謝,值謝梳頭,遽取衣幘,桓公云:「何煩此。」因下共語至暝。既去,謂左右曰:「頗曾見如此人不?」

謝公作宣武司馬,屬門生數十人於田曹中郎趙悅子。悅子以告宣武,宣武云:「且為用半。」趙俄而悉用之,曰:「昔安石在東山,縉紳敦逼,恐不豫人事;況今自鄉選,反違之邪?」

桓宣武表云:「謝尚神懷挺率,少致民譽。」

世目謝尚為令達,阮遙集云:「清暢似達。」或云:「尚自然令上。」

桓大司馬病。謝公往省病,從東門入。桓公遙望,嘆曰:「吾門中久不見如此人!」

簡文目敬豫為「朗豫」。

孫興公為庾公參軍,共遊白石山。衛君長在坐,孫曰:「此子神情都不關山水,而能作文。」庾公曰:「衛風韻雖不及卿諸人,傾倒處亦不近。」孫遂沐浴此言。

王右軍目陳玄伯:「壘塊有正骨。」

王長史云:「劉尹知我,勝我自知。」

王、劉聽林公講,王語劉曰:「向高坐者,故是凶物。」復東聽,王又曰:「自是鉢釪後王,何人也。」

許玄度言:「琴賦所謂『非至精者,不能與之析理』,劉尹其人;『非淵靜者,不能與之閑止』,簡文其人。」

魏隱兄弟,少有學義,總角詣謝奉。奉與語,大說之,曰:「大宗雖衰,魏氏已復有人。」

簡文云:「淵源語不超詣簡至;然經綸思尋處,故有局陳。」

初,法汰北來未知名,王領軍供養之。每與周旋,行來往名勝許,輒與俱。不得汰,便停車不行。因此名遂重。

王長史與大司馬書,道淵源「識致安處,足副時談。」

謝公云:「劉尹語審細。」

桓公語嘉賓:「阿源有德有言,向使作令僕,足以儀刑百揆。朝廷用違其才耳。」

簡文語嘉賓:「劉尹語末後亦小異,回復其言,亦乃無過。」

孫興公、許玄度共在白樓亭,共商略先往名達。林公既非所關,聽訖云:「二賢故自有才情。」

王右軍道東陽「我家阿林,章清太出」。

王長史與劉尹書,道淵源「觸事長易」。

謝中郎云:「王脩載樂託之性,出自門風。」

林公云:「王敬仁是超悟人。」

劉尹先推謝鎮西,謝後雅重劉曰:「昔嘗北面。」

謝太傅稱王脩齡曰:「司州可與林澤遊。」

諺曰:「楊州獨步王文度,後來出人郗嘉賓。」

人問王長史江虨兄弟羣從,王答曰:「諸江皆復足自生活。」

謝太傅道安北:「見之乃不使人厭,然出戶去,不復使人思。」

謝公云:「司州造勝遍決。」

劉尹云:「見何次道飲酒,使人欲傾家釀。」

謝太傅語真長:「阿齡於此事,故欲太厲。」劉曰:「亦名士之高操者。」

王子猷說:「世目士少為朗,我家亦以為徹朗。」

謝公云:「長史語甚不多,可謂有令音。」

謝鎮西道敬仁「文學鏃鏃,無能不新」。

劉尹道江道羣「不能言而能不言」。

林公云:「見司州警悟交至,使人不得住,亦終日忘疲。」

世稱:「苟子秀出,阿興清和。」

簡文云:「劉尹茗柯有實理。」

謝胡兒作著作郎,嘗作王堪傳。不諳堪是何似人,咨謝公。謝公答曰:「世冑亦被遇。堪,烈之子,阮千里姨兄弟,潘安仁中外。安仁詩所謂『子親伊姑,我父唯舅』。是許允壻。」

謝太傅重鄧僕射,常言「天地無知,使伯道無兒」。

謝公與王右軍書曰:「敬和棲託好佳。」

吳四姓舊目云:「張文、朱武、陸忠、顧厚。」

謝公語王孝伯:「君家藍田,舉體無常人事。」

許掾嘗詣簡文,爾夜風恬月朗,乃共作曲室中語。襟情之詠,偏是許之所長。辭寄清婉,有逾平日。簡文雖契素,此遇尤相咨嗟。不覺造厀,共叉手語,達于將旦。既而曰:「玄度才情,故未易多有許。」

殷允出西,郗超與袁虎書云:「子思求良朋,託好足下,勿以開美求之。」世目袁為「開美」,故子敬詩曰:「袁生開美度。」

謝車騎問謝公:「真長性至峭,何足乃重?」答曰:「是不見耳!阿見子敬,尚使人不能已。」

謝公領中書監,王東亭有事應同上省,王後至,坐促,王、謝雖不通,太傅猶斂厀容之。王神意閑暢,謝公傾目。還謂劉夫人曰:「向見阿瓜,故自未易有。雖不相關,正是使人不能已已。」

王子敬語謝公:「公故蕭灑。」謝曰:「身不蕭灑。君道身最得,身正自調暢。」

謝車騎初見王文度曰:「見文度雖蕭灑相遇,其復愔愔竟夕。」

范豫章謂王荊州:「卿風流雋望,真後來之秀。」王曰:「不有此舅,焉有此甥?」

子敬與子猷書,道「兄伯蕭索寡會,遇酒則酣暢忘反,乃自可矜」。

張天錫世雄涼州,以力弱詣京師,雖遠方殊類,亦邊人之桀也。聞皇京多才,欽羡彌至。猶在渚住,司馬著作往詣之。言容鄙陋,無可觀聽。天錫心甚悔來,以遐外可以自固。王彌有雋才美譽,當時聞而造焉。既至,天錫見其風神清令,言話如流,陳說古今,無不貫悉。又諳人物氏族,中來皆有證據。天錫訝服。

王恭始與王建武甚有情,後遇袁悅之間,遂致疑隟。然每至興會,故有相思。時恭嘗行散至京口射堂,于時清露晨流,新桐初引,恭目之曰:「王大故自濯濯。」

司馬太傅為二王目曰:「孝伯亭亭直上,阿大羅羅清疎。」

王恭有清辭簡旨,能敘說,而讀書少,頗有重出。有人道孝伯常有新意,不覺為煩。

殷仲堪喪後,桓玄問仲文:「卿家仲堪,定是何似人?」仲文曰:「雖不能休明一世,足以映徹九泉。」



\chapter{品藻第九}

汝南陳仲舉,潁川李元禮二人,共論其功德,不能定先後。蔡伯喈評之曰:「陳仲舉彊於犯上,李元禮嚴於攝下。犯上難,攝下易。」仲舉遂在三君之下,元禮居八俊之上。

龐士元至吳,吳人並友之。見陸績、顧劭、全琮而為之目曰:「陸子所謂駑馬有逸足之用,顧子所謂駑牛可以負重致遠。」或問:「如所目,陸為勝邪?」曰:「駑馬雖精速,能致一人耳。駑牛一日行百里,所致豈一人哉?」吳人無以難。「全子好聲名,似汝南樊子昭。」

顧劭嘗與龐士元宿語,問曰:「聞子名知人,吾與足下孰愈?」曰:「陶冶世俗,與時浮沈,吾不如子;論王霸之餘策,覽倚仗之要害,吾似有一日之長。」劭亦安其言。

諸葛瑾弟亮及從弟誕,並有盛名,各在一國。于時以為「蜀得其龍,吳得其虎,魏得其狗」。誕在魏與夏侯玄齊名;瑾在吳,吳朝服其弘量。

司馬文王問武陔:「陳玄伯何如其父司空?」陔曰:「通雅博暢,能以天下聲教為己任者,不如也。明練簡至,立功立事,過之。」

正始中,人士比論,以五荀方五陳:荀淑方陳寔,荀靖方陳諶,荀爽方陳紀,荀彧方陳群,荀顗方陳泰。又以八裴方八王:裴徽方王祥,裴楷方王夷甫,裴康方王綏,裴綽方王澄,裴瓚方王敦,裴遐方王導,裴頠方王戎,裴邈方王玄。

冀州刺史楊淮二子喬與髦,俱總角為成器。淮與裴頠、樂廣友善,遣見之。頠性弘方,愛喬之有高韻,謂淮曰:「喬當及卿,髦小減也。」廣性清淳,愛髦之有神檢,謂淮曰:「喬自及卿,然髦尤精出。」淮笑曰:「我二兒之優劣,乃裴、樂之優劣。」論者評之:以為喬雖高韻,而檢不匝;樂言為得。然並為後出之雋。

劉令言始入洛,見諸名士而嘆曰:「王夷甫太解明,樂彥輔我所敬,張茂先我所不解,周弘武巧於用短,杜方叔拙於用長。」

王夷甫云:「閭丘沖,優於滿奮、郝隆。此三人並是高才,沖最先達。」

王夷甫以王東海比樂令,故王中郎作碑云:「當時標榜,為樂廣之儷。」

庾中郎與王平子鴈行。

王大將軍在西朝時,見周侯輒扇障面不得住。後度江左,不能復爾。王嘆曰:「不知我進,伯仁退?」

會稽虞\ext{𩦎},元皇時與桓宣武同俠,其人有才理勝望。王丞相嘗謂曰:「孔愉有公才而無公望,丁潭有公望而無公才,兼之者其在卿乎?」\ext{𩦎}未達而喪。

明帝問周伯仁:「卿自謂何如郗鑒?」周曰:「鑒方臣,如有功夫。」復問郗。郗曰:「周顗比臣,有國士門風。」

王大將軍下,庾公問:「卿有四友,何者是?」答曰:「君家中郎,我家太尉、阿平、胡毋彥國。阿平故當最劣。」庾曰:「似未肯劣。」庾又問:「何者居其右?」王曰:「自有人。」又問:「何者是?」王曰:「噫!其自有公論。」左右躡公,公乃止。

人問丞相:「周侯何如和嶠?」答曰:「長輿嵯櫱。」

明帝問謝鯤:「君自謂何如庾亮?」答曰:「端委廟堂,使百僚準則,臣不如亮。一丘一壑,自謂過之。」

王丞相二弟不過江,曰頴,曰敞。時論以頴比鄧伯道,敞比溫忠武。議郎、祭酒者也。

明帝問周侯:「論者以卿比郗鑒,云何?」周曰:「陛下不須牽顗比。」

王丞相云:「頃下論以我比安期、千里。亦推此二人。唯共推太尉,此君特秀。」

宋褘曾為王大將軍妾,後屬謝鎮西。鎮西問褘:「我何如王?」答曰:「王比使君,田舍、貴人耳!」鎮西妖冶故也。

明帝問周伯仁:「卿自謂何如庾元規?」對曰:「蕭條方外,亮不如臣;從容廊廟,臣不如亮。」

王丞相辟王藍田為掾,庾公問丞相:「藍田何似?」王曰:「真獨簡貴,不減父祖;然曠澹處,故當不如爾。」

卞望之云:「郗公體中有三反:方於事上,好下佞己,一反。治身清貞,大脩計校,二反。自好讀書,憎人學問,三反。」

世論溫太真,是過江第二流之高者。時名輩共說人物,第一將盡之間,溫常失色。

王丞相云:「見謝仁祖恆令人得上。與何次道語,唯舉手指地曰:『正自爾馨!』」

何次道為宰相,人有譏其信任不得其人。阮思曠慨然曰:「次道自不至此。但布衣超居宰相之位,可恨!唯此一條而已。」

王右軍少時,丞相云:「逸少何緣復減萬安邪?」

郗司空家有傖奴,知及文章,事事有意。王右軍向劉尹稱之。劉問「何如方回?」王曰:「此正小人有意向耳!何得便比方回?」劉曰:「若不如方回,故是常奴耳!」

時人道阮思曠:「骨氣不及右軍,簡秀不如真長,韶潤不如仲祖,思致不如淵源,而兼有諸人之美。」

簡文云:「何平叔巧累於理,嵇叔夜雋傷其道。」

時人共論晉武帝出齊王之與立惠帝,其失孰多?多謂立惠帝為重。桓溫曰:「不然,使子繼父業,弟承家祀,有何不可?」

人問殷淵源:「當世王公以卿比裴叔道,云何?」殷曰:「故當以識通暗處。」

撫軍問殷浩:「卿定何如裴逸民?」良久答曰:「故當勝耳。」

桓公少與殷侯齊名,常有競心。桓問殷:「卿何如我?」殷云:「我與我周旋久,寧作我。」

撫軍問孫興公:「劉真長何如?」曰:「清蔚簡令。」「王仲祖何如?」曰:「溫潤恬和。」「桓溫何如?」曰:「高爽邁出。」「謝仁祖何如?」曰:「清易令達。」「阮思曠何如?」曰:「弘潤通長。」「袁羊何如?」曰:「洮洮清便。」「殷洪遠何如?」曰:「遠有致思。」「卿自謂何如?」曰:「下官才能所經,悉不如諸賢;至於斟酌時宜,籠罩當世,亦多所不及。然以不才,時復託懷玄勝,遠詠老、莊,蕭條高寄,不與時務經懷,自謂此心無所與讓也。」

桓大司馬下都,問真長曰:「聞會稽王語奇進,爾邪?」劉曰:「極進,然故是第二流中人耳!」桓曰:「第一流復是誰?」劉曰:「正是我輩耳!」

殷侯既廢,桓公語諸人曰:「少時與淵源共騎竹馬,我棄去,己輒取之,故當出我下。」

人問撫軍:「殷浩談竟何如?」答曰:「不能勝人,差可獻酬群心。」

簡文云:「謝安南清令不如其弟,學義不及孔巖,居然自勝。」

未廢海西公時,王元琳問桓元子:「箕子、比干,迹異心同,不審明公孰是孰非?」曰:「仁稱不異,寧為管仲。」

劉丹陽、王長史在瓦官寺集,桓護軍亦在坐,共商略西朝及江左人物。或問:「杜弘治何如衛虎?」桓答曰:「弘治膚清,衛虎奕奕神令。」王、劉善其言。

劉尹撫王長史背曰:「阿奴比丞相,但有都長。」

劉尹、王長史同坐,長史酒酣起舞。劉尹曰:「阿奴今日不復減向子期。」

桓公問孔西陽:「安石何如仲文?」孔思未對,反問公曰:「何如?」答曰:「安石居然不可陵踐其處,故乃勝也。」

謝公與時賢共賞說,遏、胡兒並在坐。公問李弘度曰:「卿家平陽,何如樂令?」於是李潸然流涕曰:「趙王篡逆,樂令親授璽綬。亡伯雅正,恥處亂朝,遂至仰藥。恐難以相比!此自顯於事實,非私親之言。」謝公語胡兒曰:「有識者果不異人意。」

王脩齡問王長史:「我家臨川,何如卿家宛陵?」長史未答,脩齡曰:「臨川譽貴。」長史曰:「宛陵未為不貴。」

劉尹至王長史許清言,時苟子年十三,倚床邊聽。既去,問父曰:「劉尹語何如尊?」長史曰:「韶音令辭,不如我;往輒破的,勝我。」

謝萬壽春敗後,簡文問郗超:「萬自可敗,那得乃爾失士卒情?」超曰:「伊以率任之性,欲區別智勇。」

劉尹謂謝仁祖曰:「自吾有四友,門人加親。」謂許玄度曰:「自吾有由,惡言不及於耳。」二人皆受而不恨。

世目殷中軍:「思緯淹通,比羊叔子。」

有人問謝安石、王坦之優劣於桓公。桓公停欲言,中悔曰:「卿喜傳人語,不能復語卿。」

王中郎嘗問劉長沙曰:「我何如苟子?」劉答曰:「卿才乃當不勝苟子,然會名處多。」王笑曰:「痴!」

支道林問孫興公:「君何如許掾?」孫曰:「高情遠致,弟子蚤已服膺;一吟一詠,許將北面。」

王右軍問許玄度:「卿自言何如安石?」許未答,王因曰:「安石故相為雄,阿萬當裂眼爭邪?」

劉尹云:「人言江虨田舍,江乃自田宅屯。」

謝公云:「金谷中蘇紹最勝。」紹是石崇姊夫,蘇則孫,愉子也。

劉尹目庾中郎:「雖言不愔愔似道,突兀差可以擬道。」

孫承公云:「謝公清於無奕,潤於林道。」

或問林公:「司州何如二謝?」林公曰:「故當攀安提萬。」

孫興公、許玄度皆一時名流。或重許高情,則鄙孫穢行;或愛孫才藻,而無取於許。

郗嘉賓道謝公:「造厀雖不深徹,而纏綿綸至。」又曰:「右軍詣嘉賓。」嘉賓聞之云:「不得稱詣,政得謂之朋耳!」謝公以嘉賓言為得。

庾道季云:「思理倫和,吾愧康伯;志力彊正,吾愧文度。自此以還,吾皆百之。」

王僧恩輕林公,藍田曰:「勿學汝兄,汝兄自不如伊。」

簡文問孫興公:「袁羊何似?」答曰:「不知者不負其才;知之者無取其體。」

蔡叔子云:「韓康伯雖無骨榦,然亦膚立。」

郗嘉賓問謝太傅曰:「林公談何如嵇公?」謝云:「嵇公勤著腳,裁可得去耳。」又問:「殷何如支?」謝曰:「正爾有超拔,支乃過殷。然亹亹論辯,恐口欲制支。」

庾道季云:「廉頗、藺相如雖千載上死人,懍懍恆如有生氣。曹蜍、李志雖見在,厭厭如九泉下人。人皆如此,便可結繩而治,但恐狐狸猯狢噉盡。」

衛君長是蕭祖周婦兄,謝公問孫僧奴:「君家道衛君長云何?」孫曰:「云是世業人。」謝曰:「殊不爾,衛自是理義人。」于時以比殷洪遠。

王子敬問謝公:「林公何如庾公?」謝殊不受,答曰:「先輩初無論,庾公自足沒林公。」

謝遏諸人共道竹林優劣,謝公云:「先輩初不臧貶七賢。」

有人以王中郎比車騎,車騎聞之曰:「伊窟窟成就。」

謝太傅謂王孝伯:「劉尹亦奇自知,然不言勝長史。」

王黃門兄弟三人俱詣謝公,子猷、子重多說俗事,子敬寒溫而已。既出,坐客問謝公:「向三賢孰愈?」謝公曰:「小者最勝!」客曰:「何以知之?」謝公曰:「『吉人之辭寡,躁人之辭多。』,推此知之。」

謝公問王子敬:「君書何如君家尊?」答曰:「固當不同。」公曰:「外人論殊不爾。」王曰:「外人那得知?」

王孝伯問謝太傅:「林公何如長史?」太傅曰:「長史韶興。」問:「何如劉尹?」謝曰:「噫!劉尹秀。」王曰:「若如公言,並不如此二人邪?」謝云:「身意正爾也。」

人有問太傅:「子敬可是先輩誰比?」謝曰:「阿敬近撮王、劉之標。」

謝公語孝伯:「君祖比劉尹,故為得逮。」孝伯云:「劉尹非不能逮,直不逮。」

袁彥伯為吏部郎,子敬與郗嘉賓書曰:「彥伯已入,殊足頓興往之氣。故知捶撻自難為人,冀小卻,當復差耳。」

王子猷、子敬兄弟共賞高士傳人及贊。子敬賞井丹高潔,子猷云:「未若長卿慢世。」

有人問袁侍中曰:「殷仲堪何如韓康伯?」答曰:「理義所得,優劣乃復未辨;然門庭蕭寂,居然有名士風流,殷不及韓。」故殷作誄云:「荊門晝掩,閑庭晏然。」

王子敬問謝公:「嘉賓何如道季?」答曰:「道季誠復鈔撮清悟,嘉賓故自上。」

王珣疾,臨困,問王武岡曰:「世論以我家領軍比誰?」武岡曰:「世以比王北中郎。」東亭轉臥向壁,嘆曰:「人固不可以無年!」

王孝伯道謝公:「濃至。」又曰:「長史虛,劉尹秀,謝公融。」

王孝伯問謝公:「林公何如右軍?」謝曰:「右軍勝林公,林公在司州前亦貴徹。」

桓玄為太傅,大會,朝臣畢集。坐裁竟,問王楨之曰:「我何如卿第七叔?」于時賓客為之咽氣。王徐徐答曰:「亡叔是一時之標,公是千載之英。」一坐懽然。

桓玄問劉太常曰:「我何如謝太傅?」劉答曰:「公高,太傅深。」又曰:「何如賢舅子敬?」答曰:「樝、梨、橘、柚,各有其美。」

舊以桓謙比殷仲文。桓玄時,仲文入,桓於庭中望見之,謂同坐曰:「我家中軍,那得及此也!」



\chapter{規箴第十}

漢武帝乳母嘗於外犯事,帝欲申憲;乳母求救東方朔。朔曰:「此非脣舌所爭。爾必望濟者,將去時,但當屢顧帝,慎勿言;此或可萬一冀耳。」乳母既至,朔亦侍側,因謂曰:「汝痴耳!帝豈復憶汝乳哺時恩邪?」帝雖才雄心忍,亦深有情戀;乃淒然愍之,即敕免罪。

京房與漢元帝共論,因問帝:「幽、厲之君何以亡?所任何人?」答曰:「其任人不忠。」房曰:「知不忠而任之,何邪?」曰:「亡國之君,各賢其臣,豈知不忠而任之?」房稽首曰:「將恐今之視古,亦猶後之視今也。」

陳元方遭父喪,哭泣哀慟,軀體骨立。其母愍之,竊以錦被蒙上。郭林宗弔而見之,謂曰:「卿海內之雋才,四方是則,如何當喪,錦被蒙上?孔子曰:『衣夫錦也,食夫稻也,於汝安乎?』吾不取也!」奮衣而去。自後賓客絕百所日。

孫休好射雉,至其時則晨去夕反。群臣莫不止諫:「此為小物,何足甚躭?」休曰:「雖為小物,耿介過人,朕所以好之。」

孫皓問丞相陸凱曰:「卿一宗在朝有幾人?」陸答曰:「二相、五侯、將軍十餘人。」皓曰:「盛哉!」陸曰:「君賢臣忠,國之盛也;父慈子孝,家之盛也。今政荒民弊,覆亡是懼,臣何敢言盛?」

何晏、鄧颺令管輅作卦,云:「不知位至三公不?」卦成,輅稱引古義,深以戒之。颺曰:「此老生之常談。」晏曰:「知幾其神乎!古人以為難。交疎吐誠,今人以為難。今君一面盡二難之道,可謂『明德惟馨』。詩不云乎:『中心藏之,何日忘之!』」

晉武帝既不悟太子之愚,必有傳後意。諸名臣亦多獻直言。帝嘗在陵雲臺上坐,衛瓘在側,欲申其懷,因如醉跪帝前,以手撫床曰:「此坐可惜。」帝雖悟,因笑曰:「公醉邪?」

王夷甫婦,郭泰寧女,才拙而性剛,聚斂無厭,干豫人事;夷甫患之,而不能禁。時其鄉人幽州刺史李陽,京都大俠,猶漢之樓護,郭氏憚之;夷甫驟諫之,乃曰:「非但我言卿不可,李陽亦謂卿不可!」郭氏小為之損。

王夷甫雅尚玄遠,常嫉其婦貪濁,口未嘗言「錢」字。婦欲試之,令婢以錢遶床,不得行。夷甫晨起,見錢閡行,謂婢曰:「舉卻阿堵物。」

王平子年十四、五,見王夷甫妻郭氏貪欲,令婢路上儋糞。平子諫之,並言不可。郭大怒,謂平子曰:「昔夫人臨終,以小郎囑新婦,不以新婦囑小郎!」急捉衣裾,將與杖。平子饒力,爭得脫,踰窗而走。

元帝過江猶好酒,王茂弘與帝有舊,常流涕諫。帝許之,命酌酒,一酣,從是遂斷。

謝鯤為豫章太守,從大將軍下至石頭。敦謂鯤曰:「余不得復為盛德之事矣。」鯤曰:「何為其然?但使自今已後,日亡日去耳!」敦又稱疾不朝,鯤諭敦曰:「近者,明公之舉,雖欲大存社稷,然四海之內,實懷未達。若能朝天子,使群臣釋然,萬物之心,於是乃服。仗民望以從眾懷,盡沖退以奉主上,如斯,則勳侔一匡,名垂千載。」時人以為名言。

元皇帝時,廷尉張闓在小市居,私作都門,早閉晚開。群小患之,詣州府訴,不得理,遂至檛登聞鼓,猶不被判。聞賀司空出,至破岡,連名詣賀訴。賀曰:「身被徵作禮官,不關此事。」群小叩頭曰:「若府君復不見治,便無所訴。」賀未語,令且去,見張廷尉當為及之。張聞,即毀門,自至方山迎賀。賀出見辭之曰:「此不必見關,但與君門情,相為惜之。」張愧謝曰:「小人有如此,始不即知,蚤已毀壞。」

郗太尉晚節好談,既雅非所經,而甚矜之。後朝覲,以王丞相末年多可恨,每見,必欲苦相規誡。王公知其意,每引作它言。臨還鎮,故命駕詣丞相。丞相翹須厲色,上坐便言:「方當乖別,必欲言其所見。」意滿口重,辭殊不流。王公攝其次曰:「後面未期,亦欲盡所懷,願公勿復談。」郗遂大瞋,冰衿而出,不得一言。

王丞相為揚州,遣八部從事之職。顧和時為下傳還,同時俱見。諸從事各奏二千石官長得失,至和獨無言。王問顧曰:「卿何所聞?」答曰:「明公作輔,寧使網漏吞舟,何緣采聽風聞,以為察察之政?」丞相咨嗟稱佳,諸從事自視缺然也。

蘇峻東征沈充,請吏部郎陸邁與俱。將至吳,密勑左右,令入閶門放火以示威。陸知其意,謂峻曰:「吳治平未久,必將有亂。若為亂階,請從我家始。」峻遂止。

陸玩拜司空,有人詣之,索美酒,得,便自起,瀉箸梁柱間地,祝曰:「當今乏才,以爾為柱石之用,莫傾人棟梁。」玩笑曰:「戢卿良箴。」

小庾在荊州,公朝大會,問諸僚佐曰:「我欲為漢高、魏武何如?」一坐莫答,長史江虨曰:「願明公為桓、文之事,不願作漢高、魏武也。」

羅君章為桓宣武從事,謝鎮西作江夏,往檢校之。羅既至,初不問郡事;徑就謝數日,飲酒而還。桓公問有何事?君章云:「不審公謂謝尚何似人?」桓公曰:「仁祖是勝我許人。」君章云:「豈有勝公人而行非者,故一無所問。」桓公奇其意而不責也。

王右軍與王敬仁、許玄度並善。二人亡後,右軍為論議更克。孔巖誡之曰:「明府昔與王、許周旋有情,及逝沒之後,無慎終之好,民所不取。」右軍甚愧。

謝中郎在壽春敗,臨奔走,猶求玉帖鐙。太傅在軍,前後初無損益之言。爾日猶云:「當今豈須煩此?」

王大語東亭:「卿乃復論成不惡,那得與僧彌戲!」

殷覬病困,看人政見半面。殷荊州興晉陽之甲,往與覬別,涕零,屬以消息所患。覬答曰:「我病自當差,正憂汝患耳!」

遠公在廬山中,雖老,講論不輟。弟子中或有墮者,遠公曰:「桑榆之光,理無遠照;但願朝陽之暉,與時並明耳。」執經登坐,諷誦朗暢,詞色甚苦。高足之徒,皆肅然增敬。

桓南郡好獵,每田狩,車騎甚盛。五六十里中,旌旗蔽隰。騁良馬,馳擊若飛,雙甄所指,不避陵壑。或行陳不整,麏兔騰逸,參佐無不被繫束。桓道恭,玄之族也,時為賊曹參軍,頗敢直言。常自帶絳綿繩箸腰中,玄問「此何為?」答曰:「公獵,好縛人士,會當被縛,手不能堪芒也。」玄自此小差。

王緒、王國寶相為脣齒,並上下權要。王大不平其如此,乃謂緒曰:「汝為此歘歘,曾不慮獄吏之為貴乎?」

桓玄欲以謝太傅宅為營,謝混曰:「召伯之仁,猶惠及甘棠;文靖之德,更不保五畝之宅。」玄慙而止。



\chapter{捷悟第十一}

楊德祖為魏武主簿,時作相國門,始搆榱桷,魏武自出看,使人題門,作「活」字,便去。楊見,即令壞之。既竟,曰:「門中『活』,『闊』字。王正嫌門大也。」

人餉魏武一桮酪,魏武噉少許,蓋頭上題「合」字以示眾。眾莫能解。次至楊脩,脩便噉,曰:「公教人噉一口也,復何疑?」

魏武嘗過曹娥碑下,楊脩從,碑背上見題作「黃絹幼婦,外孫齏臼」八字。魏武謂脩曰:「解不?」答曰:「解。」魏武曰:「卿未可言,待我思之。」行三十里,魏武乃曰:「吾已得。」令脩別記所知。脩曰:「黃絹,色絲也,於字為絕。幼婦,少女也,於字為妙。外孫,女子也,於字為好。齏臼,受辛也,於字為辭。所謂『絕妙好辭』也。」魏武亦記之,與脩同,乃嘆曰:「我才不及卿,乃覺三十里。」

魏武征袁本初,治裝餘有數十斛竹片,咸長數寸,眾並謂不堪用,正令燒除。太祖甚惜,思所以用之。謂可為竹椑楯,而未顯其言。馳使問主簿楊德祖。應聲答之,與帝正同。眾伏其辯悟。

王敦引軍垂至大桁,明帝自出中堂。溫嶠為丹陽尹,帝令斷大桁,故未斷,帝大怒,瞋目,左右莫不悚懼。召諸公來。嶠至不謝,但求酒炙。王導須臾至,徒跣下地,謝曰:「天威在顏,遂使溫嶠不容得謝。」嶠於是下謝,帝迺釋然。諸公共嘆王機悟名言。

郗司空在北府,桓宣武惡其居兵權。郗於事機素暗,遣牋詣桓:「方欲共獎王室,脩復園陵。」世子嘉賓出行,於道上聞信至,急取牋,視竟,寸寸毀裂,便回。還更作牋,自陳老病,不堪人間,欲乞閑地自養。宣武得牋大喜,即詔轉公督五郡,會稽太守。

王東亭作宣武主簿,嘗春月與石頭兄弟乘馬出郊野;時彥同遊者,連鑣俱進,唯東亭一人常在前,覺數十步,諸人莫之解。石頭等既疲倦。俄而乘輿向,諸人皆似從官,唯東亭奕奕在前。其悟捷如此。



\chapter{夙惠第十二}

賓客詣陳太丘宿,太丘使元方、季方炊。客與太丘論議,二人進火,俱委而竊聽,炊忘箸箄,飯落釜中。太丘問:「炊何不餾?」元方、季方長跪曰:「大人與客語,乃俱竊聽,炊忘箸箄,今皆成糜。」太丘曰:「爾頗有所識不?」對曰:「彷彿志之。」二子長跪俱說,更相易奪,言無遺失。太丘曰:「如此,但糜自可,何必飯也。」

何晏年七歲,明惠若神,魏武奇愛之;以晏在宮內,因欲以為子。晏乃畫地令方,自處其中。人問其故?答曰:「何氏之廬也。」魏武知之,即遣還外。

晉明帝數歲,坐元帝厀上。有人從長安來,元帝問洛下消息,潸然流涕。明帝問何以致泣?具以東渡意告之。因問明帝:「汝意謂長安何如日遠?」答曰:「日遠。不聞人從日邊來,居然可知。」元帝異之。明日集羣臣宴會,告以此意,更重問之。乃答曰:「日近。」元帝失色,曰:「爾何故異昨日之言邪?」答曰:「舉目見日,不見長安。」

司空顧和與時賢共清言,張玄之、顧敷是中外孫,年並七歲,在床邊戲。于時聞語,神情如不相屬。瞑於燈下,二兒共敘客主之言,都無遺失。顧公越席而提其耳曰:「不意衰宗復生此寶。」

韓康伯年數歲,家酷貧,至大寒,止得襦。母殷夫人自成之,令康伯捉熨斗,謂康伯曰:「且箸襦,尋作複

晉孝武年十二,時冬天,晝日不箸複衣,但箸單練衫五六重,夜則累茵褥。謝公諫曰:「聖體宜令有常。陛下晝過冷,夜過熱,恐非攝養之術。」帝曰:「晝動夜靜。」謝公出嘆曰:「上理不減先帝。」

桓宣武薨,桓南郡年五歲,服始除,桓車騎與送故文武別,因指語南郡:「此皆汝家故吏佐。」玄應聲慟哭,酸感傍人。車騎每自目己坐曰:「靈寶成人,當以此坐還之。」鞠愛過於所生。



\chapter{豪爽第十三}

王大將軍年少時,舊有田舍名,語音亦楚。武帝喚時賢,共言伎藝之事,人人皆多有所知;唯王都無所關,意色殊惡。自言知打鼓吹。帝即令取鼓與之,於坐振袖而起,揚槌奮擊,音節諧捷,神氣豪上,傍若無人。舉坐嘆其雄爽。

王處仲世許高尚之目,嘗荒恣於色,體為之弊。左右諫之,處仲曰:「吾乃不覺爾。如此者,甚易耳!」乃開內後閤,驅諸婢妾數十人出路,任其所之。時人嘆焉。

王大將軍自目:「高朗疎率,學通左氏。」

王處仲每酒後,輒詠「老驥伏櫪,志在千里;烈士暮年,壯心不已」,以如意打唾壺,壺口盡缺。

晉明帝欲起池臺,元帝不許。帝時為太子,好養武士。一夕中作池,比曉便成。今太子西池是也。

王大將軍始欲下都處分樹置,先遣參軍告朝廷,諷旨時賢。祖車騎尚未鎮壽春,瞋目厲聲語使人曰:「卿語阿黑:何敢不遜!催攝面去,須臾不爾,我將三千兵,槊腳令上!」王聞之而止。

庾穉恭既常有中原之志,文康時權重,未在己。及季堅作相,忌兵畏禍,與穉恭歷同異者久之,乃果行。傾荊、漢之力,窮舟車之勢,師次于襄陽。大會參佐,陳其旌甲,親授弧矢曰:「我之此行,若此射矣!」遂三起三疊,徒眾屬目,其氣十倍。

桓宣武平蜀,集參僚置酒於李勢殿,巴、蜀縉紳,莫不來萃。桓既素有雄情爽氣,加爾日音調英發,敘古今成敗由人,存亡繫才。其狀磊落,一坐嘆賞。既散,諸人追味餘言。于時尋陽周馥曰:「恨卿輩不見王大將軍。」

桓公讀高士傳,至於陵仲子,便擲去曰:「誰能作此溪刻自處!」

桓石虔,司空豁之長庶也;小字鎮惡,年十七八,未被舉,而童隸已呼為鎮惡郎。嘗住宣武齋頭,從征枋頭;車騎沖沒陳,左右莫能先救。宣武謂曰:「汝叔落賊,汝知不?」石虔聞,氣甚奮;命朱辟為副,策馬於數萬眾中,莫有抗者。徑致沖還。三軍嘆服。河朔遂以其名斷瘧。

陳林道在西岸,都下諸人共要至牛渚會。陳理既佳,人欲共言折。陳以如意拄頰,望雞籠山嘆曰:「孫伯符志業不遂!」於是竟坐不得談。

王司州在謝公坐,詠「入不言兮出不辭,乘回風兮載雲旗」。語人云:「當爾時,覺一坐無人。」

桓玄西下,入石頭。外白:「司馬梁王奔叛。」玄時事形已濟,在平乘上笳鼓並作,直高詠云:「簫管有遺音,梁王安在哉?」



\chapter{容止第十四}

魏武將見匈奴使,自以形陋,不足雄遠國,使崔季珪代,帝自捉刀立床頭。既畢,令間諜問曰:「魏王何如?」匈奴使答曰:「魏王雅望非常,然床頭捉刀人,此乃英雄也。」魏武聞之,追殺此使。

何平叔美姿儀,面至白,魏明帝疑其傅粉;正夏月,與熱湯餅。既噉,大汗出,以朱衣自拭,色轉皎然。

魏明帝使后弟毛曾與夏侯玄共坐,時人謂「蒹葭倚玉樹」。

時人目「夏侯太初朗朗如日月之入懷,李安國頹唐如玉山之將崩」。

嵇康身長七尺八寸,風姿特秀。見者嘆曰:「蕭蕭肅肅,爽朗清舉。」或云:「肅肅如松下風,高而徐引。」山公曰:「嵇叔夜之為人也,巖巖若孤松之獨立;其醉也,傀俄若玉山之將崩。」

裴令公目王安:「豐眼爛爛如巖下電。」

潘岳妙有姿容,好神情。少時挾彈出洛陽道,婦人遇者,莫不連手共縈之。左太沖絕醜,亦復效岳遊遨,於是群嫗齊共亂唾之,委頓而返。

王夷甫容貌整麗,妙於談玄;恆捉白玉柄麈尾,與手都無分別。

潘安仁、夏侯湛並有美容,喜同行,時人謂之「連璧」。

裴令公有儁容姿,一旦有疾至困,惠帝使王夷甫往看,裴方向壁臥,聞王使至,強回視之。王出語人曰:「雙眸閃閃,若巖下電,精神挺動,體中故小惡。」

有人語王戎曰:「嵇延祖卓卓如野鶴之在雞群。」答曰:「君未見其父耳!」

裴令公有儁容儀,脫冠冕,麤服亂頭皆好。時人以為「玉人」。見者曰:「見裴叔則如玉山上行,光映照人。」

劉伶身長六尺,貌甚醜顇,而悠悠忽忽,土木形骸。

驃騎王武子,是衛玠之舅,雋爽有風姿;見玠,輒嘆曰:「珠玉在側,覺我形穢!」

有人詣王太尉,遇安豐、大將軍、丞相在坐;往別屋,見季胤、平子。還,語人曰:「今日之行,觸目見琳琅珠玉。」

王丞相見衛洗馬曰:「居然有羸形,雖復終日調暢,若不堪羅綺。」

王大將軍稱太尉:「處眾人中,似珠玉在瓦石間。」

庾子嵩長不滿七尺,腰帶十圍,頹然自放。

衛玠從豫章至下都,人久聞其名,觀者如堵牆。玠先有羸疾,體不堪勞,遂成病而死。時人謂「看殺衛玠」。

周伯仁道桓茂倫:「嶔崎歷落可笑人。」或云謝幼輿言。

周侯說王長史父:形貌既偉,雅懷有概,保而用之,可作諸許物也。

祖士少見衛君長云:「此人有旄仗下形。」

石頭事故,朝廷傾覆。溫忠武與庾文康投陶公求救,陶公云:「肅祖顧命不見及,且蘇峻作亂,釁由諸庾,誅其兄弟,不足以謝天下。」于時庾在溫船後聞之,憂怖無計。別日,溫勸庾見陶,庾猶豫未能往,溫曰:「溪狗我所悉,卿但見之,必無憂也!」庾風姿神貌,陶一見便改觀。談宴竟日,愛重頓至。

庾太尉在武昌,秋夜氣佳景清,使吏殷浩、王胡之之徒登南樓理詠。音調始遒,聞函道中有屐聲甚厲,定是庾公。俄而率左右十許人步來,諸賢欲起避之。公徐云:「諸君少住,老子於此處興復不淺!」因便據胡床,與諸人詠謔,竟坐甚得任樂。後王逸少下,與丞相言及此事。丞相曰:「元規爾時風範,不得不小穨。」右軍答曰:「唯丘壑獨存。」

王敬豫有美形,問訊王公。王公撫其肩曰:「阿奴恨才不稱!」又云:「敬豫事事似王公。」

王右軍見杜弘治,嘆曰:「面如凝脂,眼如點漆,此神仙中人!」時人有稱王長史形者,蔡公曰:「恨諸人不見杜弘治耳!」

劉尹道桓公:鬢如反猬皮,眉如紫石稜,自是孫仲謀、司馬宣王一流人。

王敬倫風姿似父,作侍中,加授桓公,公服從大門入。桓公望之,曰:「大奴固自有鳳毛。」

林公道王長史:「斂衿作一來,何其軒軒韶舉!」

時人目王右軍:「飄如遊雲,矯若驚龍。」

王長史嘗病,親疎不通。林公來,守門人遽啟之曰:「一異人在門,不敢不啟。」王笑曰:「此必林公。」

或以方謝仁祖不乃重者。桓大司馬曰:「諸君莫輕道,仁祖企腳北窗下彈琵琶,故自有天際真人想。」

王長史為中書郎,往敬和許。爾時積雪,長史從門外下車,步入尚書,著公服。敬和遙望,嘆曰:「此不復似世中人!」

簡文作相王時,與謝公共詣桓宣武。王珣先在內,桓語王:「卿嘗欲見相王,可住帳裏。」二客既去,桓謂王曰:「定何如?」王曰:「相王作輔,自然湛若神君,公亦萬夫之望。不然,僕射何得自沒?」

海西時,諸公每朝,朝堂猶暗;唯會稽王來,軒軒如朝霞舉。

謝車騎道謝公:「遊肆復無乃高唱,但恭坐捻鼻顧睞,便自有寢處山澤閒儀。」

謝公云:「見林公雙眼,黯黯明黑。」孫興公見林公:「稜稜露其爽。」

庾長仁與諸弟入吳,欲住亭中宿。諸弟先上,見群小滿屋,都無相避意。長仁曰:「我試觀之。」乃策杖將一小兒,始入門,諸客望其神姿,一時退匿。

有人嘆王恭形茂者,云:「濯濯如春月柳。」



\chapter{自新第十五}

周處年少時,兇彊俠氣,為鄉里所患;又義興水中有蛟,山中有邅跡虎,並皆暴犯百姓;義興人謂為「三橫」,而處尤劇。或說處殺虎斬蛟,實冀「三橫」唯餘其一。處即刺殺虎,又入水擊蛟,蛟或浮或沒,行數十里,處與之俱。經三日三夜,鄉里皆謂已死,更相慶。竟殺蛟而出。聞里人相慶,始知為人情所患,有自改意。乃自吳尋二陸。平原不在,正見清河,具以情告,並云:「欲自修改,而年已蹉跎,終無所成!」清河曰:「古人貴朝聞夕死,況君前途尚可。且人患志之不立,亦何憂令名不彰邪?」處遂改勵,終為忠臣孝子。

戴淵少時,遊俠不治行檢,嘗在江、淮間攻掠商旅。陸機赴假還洛,輜重甚盛,淵使少年掠劫;淵在岸上,據胡床,指麾左右,皆得其宜。淵既神姿峰頴,雖處鄙事,神氣猶異。機於船屋上遙謂之曰:「卿才如此,亦復作劫邪?」淵便泣涕,投劍歸機,辭厲非常。機彌重之,定交,作筆荐焉。過江,仕至征西將軍。



\chapter{企羡第十六}

王丞相拜司空,桓廷尉作兩髻、葛帬、策杖,路邊窺之,歎曰:「人言阿龍超,阿龍故自超。」不覺至臺門。

王丞相過江,自說昔在洛水邊,數與裴成公、阮千里諸賢共談道。羊曼曰:「人久以此許卿,何須復爾?」王曰:「亦不言我須此,但欲爾時不可得耳!」

王右軍得人以蘭亭集序方金谷詩序,又以已敵石崇,甚有欣色。

王司州先為庾公記室參軍,後取殷浩為長史。始到,庾公欲遣王使下都。王自啟求住曰:「下官希見盛德,淵源始至,猶貪與少日周旋。」

郗嘉賓得人以己比符堅,大喜。

孟昶未達時,家在京口。嘗見王恭乘高輿,被鶴氅裘。于時微雪,昶於籬間窺之,歎曰:「此真神仙中人!」



\chapter{傷逝第十七}

王仲宣好驢鳴,既葬,文帝臨其喪,顧語同遊曰:「王好驢鳴,可各作一聲以送之。」赴客皆一作驢鳴。

王濬沖為尚書令,著公服,乘軺車,經黃公酒壚下過,顧謂後車客:「吾昔與嵇叔夜、阮嗣宗共酣飲於此壚,竹林之遊,亦預其末。自嵇生夭、阮公亡以來,便為時所羈紲。今日視此雖近,邈若山河。」

孫子荊以有才,少所推服,唯雅敬王武子。武子喪,時名士無不至者;子荊後來,臨屍慟哭,賓客莫不垂涕。哭畢,向靈床曰:「卿常好我作驢鳴,今我為卿作。」體似聲真,賓客皆笑。孫舉頭曰:「使君輩存,令此人死。」

王戎喪兒萬子,山簡往省之,王悲不自勝。簡曰:「孩抱中物,何至於此?」王曰:「聖人忘情,最下不及情;情之所鍾,正在我輩。」簡服其言,更為之慟。

有人哭和長輿曰:「峨峨若千丈松崩。」

衛洗馬以永嘉六年喪,謝鯤哭之,感動路人。咸和中,丞相王公教曰:「衛洗馬當改葬。此君風流名士,海內所瞻,可脩薄祭,以敦舊好。」

顧彥先平生好琴,及喪,家人常以琴置靈床上。張季鷹往哭之,不勝其慟,遂徑上床,鼓琴,作數曲竟,撫琴曰:「顧彥先頗復賞此不?」因又大慟,遂不執孝子手而出。

庾亮兒遭蘇峻難遇害。諸葛道明女為庾兒婦,既寡,將改適,與亮書及之。亮答曰:「賢女尚少,故其宜也。感念亡兒,若在初沒。」

庾文康亡,何揚州臨葬云:「埋玉樹箸土中,使人情何能已已!」

王長史病篤,寢臥燈下,轉麈尾視之,嘆曰:「如此人,曾不得四十!」及亡,劉尹臨殯,以犀柄麈尾箸柩中,因慟絕。

支道林喪法虔之後,精神霣喪,風味轉墜。常謂人曰:「昔匠石廢斤於郢人,牙生輟絃於鍾子,推己外求,良不虛也!冥契既逝,發言莫賞,中心蘊結,余其亡矣!」卻後一年,支遂殞。

郗嘉賓喪,左右白郗公「郎喪」,既聞,不悲,因語左右:「殯時可道。」公往臨殯,一慟幾絕。

戴公見林法師墓,曰:「德音未遠,而拱木已積。冀神理緜緜,不與氣運俱盡耳!」

王子敬與羊綏善。綏清淳簡貴,為中書郎,少亡。王深相痛悼,語東亭云:「是國家可惜人!」

王東亭與謝公交惡。王在東聞謝喪,便出都詣子敬道:「欲哭謝公。」子敬始臥,聞其言,便驚起曰:「所望於法護。」王於是往哭。督帥刁約不聽前,曰:「官平生在時,不見此客。」王亦不與語,直前,哭甚慟,不執末婢手而退。

王子猷、子敬俱病篤,而子敬先亡。子猷問左右:「何以都不聞消息?此已喪矣!」語時了不悲。便索輿來奔喪,都不哭。子敬素好琴,便徑入坐靈床上,取子敬琴彈,弦既不調,擲地云:「子敬!子敬!人琴俱亡。」因慟絕良久,月餘亦卒。

孝武山陵夕,王孝伯入臨,告其諸弟曰:「雖榱桷惟新,便自有黍離之哀!」

羊孚年三十一卒,桓玄與羊欣書曰:「賢從情所信寄,暴疾而殞,祝予之嘆,如何可言!」

桓玄當篡位,語卞鞠云:「昔羊子道恆禁吾此意。今腹心喪羊孚,爪牙失索元,而怱怱作此詆突,詎允天心?」



\chapter{棲逸第十八}

阮步兵嘯,聞數百步。蘇門山中,忽有真人,樵伐者咸共傳說。阮籍往觀,見其人擁厀巖側。籍登嶺就之,箕踞相對。籍商略終古,上陳黃、農玄寂之道,下考三代盛德之美,以問之,仡然不應。復敘有為之教,棲神導氣之術以觀之,彼猶如前,凝矚不轉。籍因對之長嘯。良久,乃笑曰:「可更作。」籍復嘯。意盡,退,還半嶺許,聞上【口酋】然有聲,如數部鼓吹,林谷傳響。顧看,迺向人嘯也。

嵇康遊於汲郡山中,遇道士孫登,遂與之遊。康臨去,登曰:「君才則高矣,保身之道不足。」

山公將去選曹,欲舉嵇康;康與書告絕。

李廞是茂曾第五子,清貞有遠操,而少羸病,不肯婚宦。居在臨海,住兄侍中墓下。既有高名,王丞相欲招禮之,故辟為府掾。廞得牋命,笑曰:「茂弘乃復以一爵假人!」

何驃騎弟以高情避世,而驃騎勸之令仕。答曰:「予第五之名,何必減驃騎?」

阮光祿在東山,蕭然無事,常內足於懷。有人以問王右軍,右軍曰:「此君近不驚寵辱,雖古之沈冥,何以過此?」

孔車騎少有嘉遁意,年四十餘,始應安東命。未仕宦時,常獨寢,歌吹自箴誨,自稱孔郎,遊散名山。百姓謂有道術,為生立廟。今猶有孔郎廟。

南陽劉驎之,高率善史傳,隱於陽岐。于時符堅臨江,荊州刺史桓沖將盡訏謨之益,徵為長史,遣人船往迎,贈貺甚厚。驎之聞命,便升舟;悉不受所餉,緣道以乞窮乏,比至上明亦盡。一見沖,因陳無用,翛然而退。居陽岐積年,衣食有無常與村人共;值己匱乏,村人亦如之;甚厚,為鄉閭所安。

南陽翟道淵與汝南周子南少相友,共隱于尋陽。庾太尉說周以當世之務,周遂仕;翟秉志彌固。其後周詣翟,翟不與語。

孟萬年及弟少孤,居武昌陽新縣。萬年遊宦,有盛名當世;少孤未嘗出,京邑人士思欲見之,乃遣信報少孤,云兄病篤。狼狽至都。時賢見之者,莫不嗟重,因相謂曰:「少孤如此,萬年可死!」

康僧淵在豫章,去郭數十里立精舍,旁連嶺,帶長川,芳林列於軒庭,清流激於堂宇。乃閒居研講,希心理味。庾公諸人,多往看之,觀其運用吐納,風流轉佳;加處之怡然,亦有以自得;聲名乃興。後不堪,遂出。

戴安道既厲操東山,而其兄欲建「式遏」之功。謝太傅曰:「卿兄弟志業,何其太殊?」戴曰:「下官『不堪其憂』,家弟『不改其樂』。」

許玄度隱在永興南幽穴中,每致四方諸侯之遺。或謂許曰:「嘗聞箕山人,似不爾耳!」許曰:「筐篚苞苴,故當輕於天下之寶耳!」

范宣未嘗入公門。韓康伯與同載,遂誘俱入郡。范便於車後趨下。

郗超每聞欲高尚隱退者,輒為辦百萬資,并為造立居宇。在剡為戴公起宅,甚精整。戴始往舊居,與所親書曰:「近至剡,如官舍。」郗為傅約亦辦百萬資,傅隱事差互,故不果遺。

許掾好遊山水,而體便登陟。時人云:「許非徒有勝情,實有濟勝之具。」

郗尚書與謝居士善。常稱:「謝慶緒識見雖不絕人,可以累心處都盡。」



\chapter{賢媛第十九}

陳嬰者,東陽人;少脩德行,箸稱鄉黨。秦末大亂,東陽人欲奉嬰為主,母曰:「不可。自我為汝家婦,少見貧賤;一旦富貴,不祥!不如以兵屬人:事成,少受其利;不成,禍有所歸。」

漢元帝宮人既多,乃令畫工圖之,欲有呼者,輒披圖召之。其中常者,皆行貨賂。王明君姿容甚麗,志不苟求,工遂毀為其狀。後匈奴來和,求美女於漢帝,帝以明君充行。既召見,而惜之;但名字已去,不欲中改,於是遂行。

漢成帝幸趙飛燕,飛燕讒班婕妤祝詛,於是考問。辭曰:「妾聞死生有命,富貴在天。脩善尚不蒙福,為邪欲以何望?若鬼神有知,不受邪佞之訴;若其無知,訴之何益?故不為也。」

魏武帝崩,文帝悉取武帝宮人自侍。及帝病困,卞后出看疾。太后入戶,見直侍並是昔日所愛幸者。太后問:「何時來邪?」云:「正伏魄時過。」因不復前而嘆曰:「狗鼠不食汝餘,死故應爾!」至山陵,亦竟不臨。

趙母嫁女,女臨去,敕之曰:「慎勿為好!」女曰:「不為好,可為惡邪?」母曰:「好尚不可為,其況惡乎?」

許允婦,是阮衛尉女,德如妹,奇醜;交禮竟,允無復入理,家人深以為憂。會允有客至,婦令婢視之,還答曰:「是桓郎。」桓郎者,桓範也。婦云:「無憂,桓必勸入。」桓果語許云:「阮家既嫁醜女與卿,故當有意,卿宜察之。」許便回入內。既見婦,即欲出。婦料其此出,無復入理,便捉裾停之。許因謂曰:「婦有四德,卿有其幾?」婦曰:「新婦所乏唯容爾。然士有百行,君有幾?」許云:「皆備。」婦曰:「夫百行以德為首,君好色不好德,何謂皆備?」允有慚色,遂相敬重。

許允為吏部郎,多用其鄉里,魏明帝遣虎賁收之。其婦出誡允曰:「明主可以理奪,難以情求。」既至,帝覈問之。允對曰:「『舉爾所知』:臣之鄉人,臣所知也。陛下檢校為稱職與不;若不稱職,臣受其罪。」既檢校,皆官得其人,於是乃釋;允衣服敗壞,詔賜新衣。初,允被收,舉家號哭。阮新婦自若云:「勿憂,尋還。」作粟粥待。頃之,允至。

許允為晉景王所誅,門生走入告其婦。婦正在機中,神色不變,曰:「蚤知爾耳!」門人欲藏其兒,婦曰:「無豫諸兒事。」後徙居墓所,景王遣鍾會看之,若才流及父,當收。兒以咨母。母曰:「汝等雖佳,才具不多,率胸懷與語,便無所憂。不須極哀,會止便止。又可少問朝事。」兒從之。會反以狀對,卒免。

王公淵娶諸葛誕女。入室,言語始交,王謂婦曰:「新婦神色卑下,殊不似公休!」婦曰:「大丈夫不能仿佛彥雲,而令婦人比蹤英傑!」

王經少貧苦,仕至二千石,母語之曰:「汝本寒家子,仕至二千石,此可以止乎!」經不能用。為尚書,助魏,不忠於晉,被收。涕泣辭母曰:「不從母敕,以至今日!」母都無慽容,語之曰:「為子則孝,為臣則忠。有孝有忠,何負吾邪?」

山公與嵇、阮一面,契若金蘭。山妻韓氏,覺公與二人異於常交,問公。公曰:「我當年可以為友者,唯此二生耳!」妻曰:「負羈之妻亦親觀狐、趙,意欲窺之,可乎?」他日,二人來,妻勸公止之宿,具酒肉。夜穿墉以視之,達旦忘反。公入曰:「二人何如?」妻曰:「君才致殊不如,正當以識度相友耳。」公曰:「伊輩亦常以我度為勝。」

王渾妻鍾氏生女令淑,武子為妹求簡美對而未得。有兵家子,有雋才,欲以妹妻之,乃白母曰:「誠是才者,其地可遺,然要令我見。」武子乃令兵兒與群小雜處,使母帷中察之。既而,母謂武子曰:「如此衣形者,是汝所擬者,非邪?」武子曰:「是也。」母曰:「此才足以拔萃,然地寒,不有長年,不得申其才用。觀其形骨,必不壽,不可與婚。」武子從之。兵兒數年果亡。

賈充前婦,是李豐女。豐被誅,離婚徙邊。後遇赦得還,充先已取郭配女。武帝特聽置左右夫人。李氏別住外,不肯還充舍。郭氏語充:「欲就省李。」充曰:「彼剛介有才氣,卿往不如不去。」郭氏於是盛威儀,多將侍婢。既至,入戶,李氏起迎,郭不覺腳自屈,因跪再拜。既反,語充,充曰:「語卿道何物?」

賈充妻李氏作女訓,行於世。李氏女,齊獻王妃,郭氏女,惠帝后。充卒,李、郭女各欲令其母合葬,經年不決。賈后廢,李氏乃祔,葬遂定。

王汝南少無婚,自求郝普女。司空以其癡,會無婚處,任其意,便許之。既婚,果有令姿淑德。生東海,遂為王氏母儀。或問汝南何以知之?曰:「嘗見井上取水,舉動容止不失常,未嘗忤觀。以此知之。」

王司徒婦,鍾氏女,太傅曾孫,亦有俊才女德。鍾、郝為娣姒,雅相親重。鍾不以貴陵郝,郝亦不以賤下鍾。東海家內,則郝夫人之法。京陵家內,範鍾夫人之禮。

李平陽,秦州子,中夏名士。于時以比王夷甫。孫秀初欲立威權,咸云:「樂令民望不可殺,減李重者又不足殺。」遂逼重自裁。初,重在家,有人走從門入,出髻中疏示重。重看之色動,入內示其女,女直叫「絕」。了其意,出則自裁。此女甚高明,重每咨焉。

周浚作安東時,行獵,值暴雨,過汝南李氏。李氏富足,而男子不在。有女名絡秀,聞外有貴人,與一婢於內宰豬羊,作數十人飲食,事事精辦,不聞有人聲。密覘之,獨見一女子,狀貌非常,浚因求為妾。父兄不許。絡秀曰:「門戶殄瘁,何惜一女?若連姻貴族,將來或大益。」父兄從之。遂生伯仁兄弟。絡秀語伯仁等:「我所以屈節為汝家作妾,門戶計耳!汝若不與吾家作親親者,吾亦不惜餘年。」伯仁等悉從命。由此李氏在世,得方幅齒遇。

陶公少有大志,家酷貧,與母湛氏同居。同郡范逵素知名,舉孝廉,投侃宿。于時冰雪積日,侃室如懸磬,而逵馬僕甚多。侃母湛氏語侃曰:「汝但出外留客,吾自為計。」湛頭髮委地,下為二髲,賣得數斛米,斫諸屋柱,悉割半為薪,剉諸薦以為馬草。日夕,遂設精食,從者皆無所乏。逵既嘆其才辯,又深愧其厚意。明旦去,侃追送不已,且百里許。逵曰:「路已遠,君宜還。」侃猶不返,逵曰:「卿可去矣!至洛陽,當相為美談。」侃迺返。逵及洛,遂稱之於羊晫、顧榮諸人,大獲美譽。

陶公少時,作魚梁吏,嘗以坩鮓餉母。母封鮓付使,反書責侃曰:「汝為吏,以官物見餉,非唯不益,乃增吾憂也。」

桓宣武平蜀,以李勢妹為妾,甚有寵,常著齋後。主始不知,既聞,與數十婢拔白刃襲之。正值李梳頭,髮委藉地,膚色玉曜,不為動容。徐曰:「國破家亡,無心至此。今日若能見殺,乃是本懷。」主慚而退。

庾玉臺,希之弟也。希誅,將戮玉臺。玉臺子婦,宣武弟桓豁女也。徒跣求進,閽禁不內。女厲聲曰:「是何小人?我伯父門,不聽我前!」因突入,號泣請曰:「庾玉臺常因人腳短三寸,當復能作賊不?」宣武笑曰:「壻故自急。」遂原玉臺一門。

謝公夫人幃諸婢,使在前作伎,使太傅暫見,便下幃。太傅索更開,夫人云:「恐傷盛德。」

桓車騎不好箸新衣。浴後,婦故送新衣與;車騎大怒,催使持去。婦更持還,傳語云:「衣不經新,何由而故?」桓公大笑,箸之。

王右軍郗夫人謂二弟司空、中郎曰:「王家見二謝,傾筐倒庋;見汝輩來,平平爾。汝可無煩復往。」

王凝之謝夫人既往王氏,大薄凝之。既還謝家,意大不說。太傅慰釋之曰:「王郎,逸少之子,人材亦不惡,汝何以恨迺爾?」荅曰:「一門叔父,則有阿大、中郎。群從兄弟,則有封、胡、遏、末。不意天壤之中,乃有王郎!」

韓康伯母,隱古几毀壞,卞鞠見几惡,欲易之。答曰:「我若不隱此,汝何以得見古物?」

王江州夫人語謝遏曰:「汝何以都不復進,為是塵務經心,天分有限。」

郗嘉賓喪,婦兄弟欲迎妹還,終不肯歸。曰:「生縱不得與郗郎同室,死寧不同穴!」

謝遏絕重其姊,張玄常稱其妹,欲以敵之。有濟尼者,並遊張、謝二家。人問其優劣?答曰:「王夫人神情散朗,故有林下風氣。顧家婦清心玉映,自是閨房之秀。」

王尚書惠嘗看王右軍夫人,問:「眼耳未覺惡不?」答曰:「髮白齒落,屬乎形骸;至於眼耳,關於神明,那可便與人隔?」

韓康伯母殷,隨孫繪之之衡陽,於闔廬洲中逢桓南郡。卞鞠是其外孫,時來問訊。謂鞠曰:「我不死,見此豎二世作賊!」在衡陽數年,繪之遇桓景真之難也,殷撫屍哭曰:「汝父昔罷豫章,徵書朝至夕發。汝去郡邑數年,為物不得動,遂及於難,夫復何言?」



\chapter{術解第二十}

荀勖善解音聲,時論謂之闇解。遂調律呂,正雅樂。每至正會,殿庭作樂,自調宮商,無不諧韻。阮咸妙賞,時謂神解。每公會作樂,而心謂之不調。既無一言直勖,意忌之,遂出阮為始平太守。後有一田父耕於野,得周時玉尺,便是天下正尺。荀試以校己所治鐘鼓、金石、絲竹,皆覺短一黍,於是伏阮神識。

荀勖嘗在晉武帝坐上食筍進飯,謂在坐人曰:「此是勞薪炊也。」坐者未之信,密遣問之,實用故車腳。

人有相羊祜父墓,後應出受命君。祜惡其言,遂掘斷墓後,以壞其勢。相者立視之曰:「猶應出折臂三公。」俄而祜墜馬折臂,位果至公。

王武子善解馬性。嘗乘一馬,箸連錢障泥。前有水,終日不肯渡。王云:「此必是惜障泥。」使人解去,便徑渡。

陳述為大將軍掾,甚見愛重。及亡,郭璞往哭之,甚哀,乃呼曰:「嗣祖,焉知非福!」俄而大將軍作亂,如其所言。

晉明帝解占塚宅,聞郭璞為人葬,帝微服往看。因問主人:「何以葬龍角?此法當滅族!」主人曰:「郭云:『此葬龍耳,不出三年,當致天子。』」帝問:「為是出天子邪?」答曰:「非出天子,能致天子問耳。」

郭景純過江,居于暨陽,墓去水不盈百步,時人以為近水。景純曰:「將當為陸。」今沙漲,去墓數十里皆為桑田。其詩曰:「北阜烈烈,巨海混混;壘壘三墳,唯母與昆。」

王丞相令郭璞試作一卦,卦成,郭意色甚惡,云:「公有震厄!」王問:「有可消伏理不?」郭曰:「命駕西出數里,得一柏樹,截斷如公長,置床上常寢處,災可消矣。」王從其語。數日中,果震柏粉碎,子弟皆稱慶。大將軍云:「君乃復委罪於樹木。」

桓公有主簿善別酒,有酒輒令先嘗。好者謂「青州從事」,惡者謂「平原督郵」。青州有齊郡,平原有鬲縣。「從事」言「到臍」,「督郵」言在「鬲上住」。

郗愔信道,甚精勤。常患腹內惡,諸醫不可療,聞于法開有名,往迎之。既來,便脈云:「君侯所患,正是精進太過所致耳。」合一劑湯與之。一服,即大下,去數段許紙如拳大;剖看,乃先所服符也。

殷中軍妙解經脉,中年都廢。有常所給使,忽叩頭流血。浩問其故?云:「有死事,終不可說。」詰問良久,乃云:「小人母年垂百歲,抱疾來久,若蒙官一脈,便有活理。訖就屠戮無恨。」浩感其至性,遂令舁來,為診脉處方。始服一劑湯,便愈。於是悉焚經方。



\chapter{巧蓺第二十一}

彈棊始自魏宮內,用妝奩戲。文帝於此戲特妙,用手巾角拂之,無不中。有客自云能,帝使為之。客箸葛巾角,低頭拂棊,妙踰於帝。

陵雲臺樓觀精巧,先稱平眾木輕重,然後造構,乃無錙銖相負揭。臺雖高峻,常隨風搖動,而終無傾倒之理。魏明帝登臺,懼其勢危,別以大材扶持之,樓即穨壞。論者謂輕重力偏故也。

韋仲將能書。魏明帝起殿,欲安榜,使仲將登梯題之。既下,頭鬢皓然,因敕兒孫:「勿復學書。」

鍾會是荀濟北從舅,二人情好不協。荀有寶劍,可直百萬,常在母鍾夫人許。會善書,學荀手跡,作書與母取劍,仍竊去不還。荀勖知是鍾而無由得也,思所以報之。後鍾兄弟以千萬起一宅,始成,甚精麗,未得移住。荀極善畫,乃潛往畫鍾門堂,作太傅形象,衣冠狀貌如平生。二鍾入門,便大感慟,宅遂空廢。

羊長和博學工書,能騎射,善圍棊。諸羊後多知書,而射、弈餘蓺莫逮。

戴安道就范宣學,視范所為:范讀書亦讀書,范抄書亦抄書。唯獨好畫,范以為無用,不宜勞思於此。戴乃畫南都賦圖;范看畢咨嗟,甚以為有益,始重畫。

謝太傅云:「顧長康畫,有蒼生來所無。」

戴安道中年畫行像甚精妙。庾道季看之,語戴云:「神明太俗,由卿世情未盡。」戴云:「唯務光當免卿此語耳。」

顧長康畫裴叔則,頰上益三毛。人問其故?顧曰:「裴楷儁朗有識具,正此是其識具。」看畫者尋之,定覺益三毛如有神明,殊勝未安時。

王中郎以圍棊是坐隱,支公以圍棊為手談。

顧長康好寫起人形。欲圖殷荊州,殷曰:「我形惡,不煩耳。」顧曰:「明府正為眼爾。但明點童子,飛白拂其上,使如輕雲之蔽日。」

顧長康畫謝幼輿在巖石裏。人問其所以?顧曰:「謝云:『一丘一壑,自謂過之。』此子宜置丘壑中。」

顧長康畫人,或數年不點目精。人問其故?顧曰:「四體妍蚩,本無關於妙處;傳神寫照,正在阿堵中。」

顧長康道畫:「手揮五絃易,目送歸鴻難。」



\chapter{寵禮第二十二}

元帝正會,引王丞相登御床,王公固辭,中宗引之彌苦。王公曰:「使太陽與萬物同暉,臣下何以瞻仰?」

桓宣武嘗請參佐入宿,袁宏、伏滔相次而至,蒞名府中,復有袁參軍,彥伯疑焉,令傳教更質。傳教曰:「參軍是袁、伏之袁,復何所疑?」

王珣、郗超並有奇才,為大司馬所眷拔。珣為主簿,超為記室參軍。超為人多須,珣狀短小。于時荊州為之語曰:「髯參軍,短主簿。能令公喜,能令公怒。」

許玄度停都一月,劉尹無日不往,乃歎曰:「卿復少時不去,我成輕薄京尹!」

孝武在西堂會,伏滔預坐。還,下車呼其兒,語之曰:「百人高會,臨坐未得他語,先問『伏滔何在?在此不?』此故未易得。為人作父如此,何如?」

卞範之為丹陽尹,羊孚南州暫還,往卞許,云:「下官疾動不堪坐。」卞便開帳拂褥,羊徑上大床,入被須枕。卞回坐傾睞,移晨達莫。羊去,卞語曰:「我以第一理期卿,卿莫負我。」



\chapter{任誕第二十三}

陳留阮籍,譙國嵇康,河內山濤,三人年皆相比,康年少亞之。預此契者:沛國劉伶,陳留阮咸,河內向秀,琅邪王戎。七人常集于竹林之下,肆意酣暢,故世謂「竹林七賢」。

阮籍遭母喪,在晉文王坐,進酒肉。司隸何曾亦在坐,曰:「明公方以孝治天下,而阮籍以重喪,顯於公坐,飲酒食肉,宜流之海外,以正風教。」文王曰:「嗣宗毀頓如此,君不能共憂之,何謂?且有疾而飲酒食肉,固喪禮也!」籍飲噉不輟,神色自若。

劉伶病酒,渴甚,從婦求酒。婦捐酒毀器,涕泣諫曰:「君飲太過,非攝生之道,必宜斷之!」伶曰:「甚善。我不能自禁,唯當祝鬼神,自誓斷之耳!便可具酒肉。」婦曰:「敬聞命。」供酒肉於神前,請伶祝誓。伶跪而祝曰:「天生劉伶,以酒為名,一飲一斛,五斗解酲。婦人之言,慎不可聽。」便引酒進肉,隗然已醉矣。

劉公榮與人飲酒,雜穢非類,人或譏之。答曰:「勝公榮者,不可不與飲;不如公榮者,亦不可不與飲;是公榮輩者,又不可不與飲。」故終日共飲而醉。

步兵校尉缺,廚中有貯酒數百斛,阮籍乃求為步兵校尉。

劉伶恆縱酒放達,或脫衣裸形在屋中,人見譏之。伶曰:「我以天地為棟宇,屋室為㡓衣,諸君何為入我㡓中?」

阮籍嫂嘗還家,籍見與別。或譏之。籍曰:「禮豈為我輩設也?」

阮公鄰家婦有美色,當壚酤酒。阮與王安豐常從婦飲酒,阮醉,便眠其婦側。夫始殊疑之,伺察,終無他意。

阮籍當葬母,蒸一肥豚,飲酒二斗,然後臨訣,直言「窮矣」!都得一號,因吐血,廢頓良久。

阮仲容、步兵居道南,諸阮居道北。北阮皆富,南阮貧。七月七日,北阮盛曬衣,皆紗羅錦綺。仲容以竿挂大布犢鼻㡓於中庭。人或怪之,答曰:「未能免俗,聊復爾耳!

阮步兵喪母,裴令公往弔之。阮方醉,散髮坐床,箕踞不哭。裴至,下席於地,哭弔喭畢,便去。或問裴:「凡弔,主人哭,客乃為禮。阮既不哭,君何為哭?」裴曰:「阮方外之人,故不崇禮制;我輩俗中人,故以儀軌自居。」時人嘆為兩得其中。

諸阮皆能飲酒,仲容至宗人閒共集,不復用常桮斟酌,以大甕盛酒,圍坐,相向大酌。時有群豬來飲,直接去上,便共飲之。

阮渾長成,風氣韻度似父,亦欲作達。步兵曰:「仲容已預之,卿不得復爾。」

裴成公婦,王戎女。王戎晨往裴許,不通徑前。裴從床南下,女從北下,相對作賓主,了無異色。

阮仲容先幸姑家鮮卑婢。及居母喪,姑當遠移,初云當留婢,既發,定將去。仲容借客驢箸重服自追之,累騎而返。曰:「人種不可失!」即遙集之母也。

任愷既失權勢,不復自檢括。或謂和嶠曰:「卿何以坐視元裒敗而不救?」和曰:「元裒如北夏門,拉攞自欲壞,非一木所能支。」

劉道真少時,常漁草澤,善歌嘯,聞者莫不留連。有一老嫗,識其非常人,甚樂其歌嘯,乃殺豚進之。道真食豚盡,了不謝。嫗見不飽,又進一豚,食半餘半,迺還之。後為吏部郎,嫗兒為小令史,道真超用之。不知所由,問母;母告之。於是齎牛酒詣道真,道真曰:「去!去!無可復用相報。」

阮宣子常步行,以百錢挂杖頭,至酒店,便獨酣暢。雖當世貴盛,不肯詣也。

山季倫為荊州,時出酣暢。人為之歌曰:「山公時一醉,徑造高陽池。日莫倒載歸,茗艼無所知。復能乘駿馬,倒箸白接籬。舉手問葛彊,何如并州兒?」高陽池在襄陽。彊是其愛將,并州人也。

張季鷹縱任不拘,時人號為江東步兵。或謂之曰:「卿乃可縱適一時,獨不為身後名邪?」答曰:「使我有身後名,不如即時一桮酒!」

畢茂世云:「一手持蟹螯,一手持酒桮,拍浮酒池中,便足了一生。」

賀司空入洛赴命,為太孫舍人。經吳閶門,在船中彈琴。張季鷹本不相識,先在金閶亭,聞絃甚清,下船就賀,因共語。便大相知說。問賀:「卿欲何之?」賀曰:「入洛赴命,正爾進路。」張曰:「吾亦有事北京。」因路寄載,便與賀同發。初不告家,家追問迺知。

祖車騎過江時,公私儉薄,無好服玩。王、庾諸公共就祖,忽見裘袍重疊,珍飾盈列,諸公怪問之。祖曰:「昨夜復南塘一出。」祖于時恆自使健兒鼓行劫鈔,在事之人,亦容而不問。

鴻臚卿孔群好飲酒。王丞相語云:「卿何為恆飲酒?不見酒家覆瓿布,日月糜爛?」群曰:「不爾,不見糟肉,乃更堪久。」群嘗書與親舊:「今年田得七百斛秫米,不了麴蘖事。」

有人譏周僕射:「與親友言戲,穢雜無檢節。」周曰:「吾若萬里長江,何能不千里一曲。」

溫太真位未高時,屢與揚州、淮中估客樗蒱,與輒不競。嘗一過,大輸物,戲屈,無因得反。與庾亮善,於舫中大喚亮曰:「卿可贖我!」庾即送直,然後得還。經此數四。

溫公喜慢語,卞令禮法自居。至庾公許,大相剖擊。溫發口鄙穢,庾公徐曰:「太真終日無鄙言。」

周伯仁風德雅重,深達危亂。過江積年,恆大飲酒。嘗經三日不醒,時人謂之「三日僕射」。

衛君長為溫公長史,溫公甚善之。每率爾提酒脯就衛,箕踞相對彌日。衛往溫許,亦爾。

蘇峻亂,諸庾逃散。庾冰時為吳郡,單身奔亡,民吏皆去。唯郡卒獨以小船載冰出錢塘口,蘧篨覆之。時峻賞募覓冰,屬所在搜檢甚急。卒捨船市渚,因飲酒醉還,舞棹向船曰:「何處覓庾吳郡?此中便是。」冰大惶怖,然不敢動。監司見船小裝狹,謂卒狂醉,都不復疑。自送過淛江,寄山陰魏家,得免。後事平,冰欲報卒,適其所願。卒曰:「出自廝下,不願名器。少苦執鞭,恆患不得快飲酒。使其酒足餘年畢矣,無所復須。」冰為起大舍,市奴婢,使門內有百斛酒,終其身。時謂此卒非唯有智,且亦達生。

殷洪喬作豫章郡,臨去,都下人因附百許函書。既至石頭,悉擲水中,因祝曰:「沈者自沈,浮者自浮,殷洪喬不能作致書郵。」

王長史、謝仁祖同為王公掾。長史云:「謝掾能作異舞。」謝便起舞,神意甚暇。王公熟視,謂客曰:「使人思安豐。」

王、劉共在杭南,酣宴於桓子野家。謝鎮西往尚書墓還,葬後三日反哭。諸人欲要之,初遣一信,猶未許,然已停車。重要,便回駕。諸人門外迎之,把臂便下,裁得脫幘箸帽。酣宴半坐,乃覺未脫衰。

桓宣武少家貧,戲大輸,債主敦求甚切,思自振之方,莫知所出。陳郡袁躭,俊邁多能。宣武欲求救於躭,躭時居艱,恐致疑,試以告焉。應聲便許,略無嫌吝。遂變服懷布帽隨溫去,與債主戲。躭素有蓺名,債主就局曰:「汝故當不辦作袁彥道邪?」遂共戲。十萬一擲,直上百萬數。投馬絕叫,傍若無人,探布帽擲對人曰:「汝竟識袁彥道不?」

王光祿云:「酒,正使人人自遠。」

劉尹云:「孫承公狂士,每至一處,賞翫累日,或回至半路卻返。」

袁彥道有二妹:一適殷淵源,一適謝仁祖。語桓宣武云:「恨不更有一人配卿。」

桓車騎在荊州,張玄為侍中,使至江陵,路經陽岐村,俄見一人,持半小籠生魚,徑來造船云:「有魚,欲寄作膾。」張乃維舟而納之。問其姓字,稱是劉遺民。張素聞其名,大相忻待。劉既知張銜命,問:「謝安、王文度並佳不?」張甚欲話言,劉了無停意。既進膾,便去,云:「向得此魚,觀君船上當有膾具,是故來耳。」於是便去。張乃追至劉家,為設酒,殊不清旨。張高其人,不得已而飲之。方共對飲,劉便先起,云:「今正伐荻,不宜久廢。」張亦無以留之。

王子猷詣郗雍州,雍州在內,見有毾㲪,云:「阿乞那得此物?」令左右送還家。郗出覓之,王曰:「向有大力者負之而趨。」郗無忤色。

謝安始出西戲,失車牛,便杖策步歸。道逢劉尹,語曰:「安石將無傷?」謝乃同載而歸。

襄陽羅友有大韻,少時多謂之癡。嘗伺人祠,欲乞食,往太蚤,門未開。主人迎神出見,問以非時,何得在此?答曰:「聞卿祠,欲乞一頓食耳。」遂隱門側。至曉,得食便退,了無怍容。為人有記功,從桓宣武平蜀,按行蜀城闕觀宇,內外道陌廣狹,植種果竹多少,皆默記之。後宣武漂洲與簡文集,友亦預焉。共道蜀中事,亦有所遺忘,友皆名列,曾無錯漏。宣武驗以蜀城闕簿,皆如其言。坐者嘆服。謝公云:「羅友詎減魏陽元!」後為廣州刺史,當之鎮,刺史桓豁語令莫來宿。答曰:「民已有前期。主人貧,或有酒饌之費,見與甚有舊,請別日奉命。」征西密遣人察之。至日,乃往荊州門下書佐家,處之怡然,不異勝達。在益州語兒云:「我有五百人食器。」家中大驚。其由來清,而忽有此物,定是二百五十沓烏樏。

桓子野每聞清歌,輒喚:「奈何!」謝公聞之曰:「子野可謂一往有深情。」

張湛好於齋前種松柏。時袁山松出遊,每好令左右作挽歌。時人謂「張屋下陳屍,袁道上行殯」。

羅友作荊州從事,桓宣武為王車騎集別。友進坐良久,辭出,宣武曰:「卿向欲咨事,何以便去?」答曰:「友聞白羊肉美,一生未曾得喫,故冒求前耳。無事可咨。今已飽,不復須駐。」了無慚色。

張驎酒後挽歌甚悽苦,桓車騎曰:「卿非田橫門人,何乃頓爾至致?」

王子猷嘗暫寄人空宅住,便令種竹。或問:「暫住何煩爾?」王嘯詠良久,直指竹曰:「何可一日無此君?」

王子猷居山陰,夜大雪,眠覺,開室,命酌酒。四望皎然,因起仿偟,詠左思〈招隱詩〉。忽憶戴安道。時戴在剡,即便夜乘小船就之。經宿方至,造門不前而返。人問其故,王曰:「吾本乘興而行,興盡而返,何必見戴?」

王衛軍云:「酒正自引人箸勝地。」

王子猷出都,尚在渚下。舊聞桓子野善吹笛,而不相識。遇桓於岸上過,王在船中,客有識之者云:「是桓子野。」王便令人與相聞云:「聞君善吹笛,試為我一奏。」桓時已貴顯,素聞王名,即便回下車,踞胡床,為作三調。弄畢,便上車去。客主不交一言。

桓南郡被召作太子洗馬,船泊荻渚。王大服散後已小醉,往看桓。桓為設酒,不能冷飲,頻語左右:「令溫酒來!」桓乃流涕嗚咽,王便欲去。桓以手巾掩淚,因謂王曰:「犯我家諱,何預卿事?」王嘆曰:「靈寶故自達。」

王孝伯問王大:「阮籍何如司馬相如?」王大曰:「阮籍胸中壘塊,故須酒澆之。」

王佛大嘆言:「三日不飲酒,覺形神不復相親。」

王孝伯言:「名士不必須奇才。但使常得無事,痛飲酒,熟讀離騷,便可稱名士。」

王長史登茅山,大慟哭曰:「琅邪王伯輿,終當為情死。」



\chapter{簡傲第二十四}

晉文王功德盛大,坐席嚴敬,擬於王者。唯阮籍在坐,箕踞嘯歌,酣放自若。

王戎弱冠詣阮籍,時劉公榮在坐。阮謂王曰:「偶有二斗美酒,當與君共飲。彼公榮者,無預焉。」二人交觴酬酢,公榮遂不得一桮。而言語談戲,三人無異。或有問之者,阮答曰:「勝公榮者,不得不與飲酒;不如公榮者,不可不與飲酒;唯公榮,可不與飲酒。」

鍾士季精有才理,先不識嵇康。鍾要于時賢雋之士,俱往尋康。康方大樹下鍛,向子期為佐鼓排。康揚槌不輟,傍若無人,移時不交一言。鍾起去,康曰:「何所聞而來?何所見而去?」鍾曰:「聞所聞而來,見所見而去。」

嵇康與呂安善,每一相思,千里命駕。安後來,值康不在,喜出戶延之,不入。題門上作「鳳」字而去。喜不覺,猶以為欣,故作「鳳」字,凡鳥也。

陸士衡初入洛,咨張公所宜詣;劉道真是其一。陸既往,劉尚在哀制中。性嗜酒,禮畢,初無他言,唯問:「東吳有長柄壺盧,卿得種來不?」陸兄弟殊失望,乃悔往。

王平子出為荊州,王太尉及時賢送者傾路。時庭中有大樹,上有鵲巢。平子脫衣巾,徑上樹取鵲子。涼衣拘閡樹枝,便復脫去。得鵲子還,下弄,神色自若,傍若無人。

高坐道人於丞相坐,恆偃臥其側。見卞令,肅然改容云:「彼是禮法人。」

桓宣武作徐州,時謝奕為晉陵。先粗經虛懷,而乃無異常。及桓還荊州,將西之間,意氣甚篤,奕弗之疑。唯謝虎子婦王悟其旨。每曰:「桓荊州用意殊異,必與晉陵俱西矣!」俄而引奕為司馬。奕既上,猶推布衣交。在溫坐,岸幘嘯詠,無異常日。宣武每曰:「我方外司馬。」遂因酒,轉無朝夕禮。桓舍入內,奕輒復隨去。後至奕醉,溫往主許避之。主曰:「君無狂司馬,我何由得相見?」

謝萬在兄前,欲起索便器。于時阮思曠在坐曰:「新出門戶,篤而無禮。」

謝中郎是王藍田女壻,嘗箸白綸巾,肩輿徑至揚州聽事見王,直言曰:「人言君侯癡,君侯信自癡。」藍田曰:「非無此論,但晚令耳。」

王子猷作桓車騎騎兵參軍,桓問曰:「卿何署?」答曰:「不知何署,時見牽馬來,似是馬曹。」桓又問:「官有幾馬?」答曰:「不問馬,何由知其數?」又問:「馬比死多少?」答曰:「未知生,焉知死?」

謝公嘗與謝萬共出西,過吳郡。阿萬欲相與共萃王恬許,太傅云:「恐伊不必酬汝,意不足爾!」萬猶苦要,太傅堅不回,萬乃獨往。坐少時,王便入門內,謝殊有欣色,以為厚待已。良久,乃沐頭散髮而出,亦不坐,仍據胡床,在中庭曬頭,神氣傲邁,了無相酬對意。謝於是乃還。未至船,逆呼太傅。安曰:「阿螭不作爾!」

王子猷作桓車騎參軍。桓謂王曰:「卿在府久,比當相料理。」初不答,直高視,以手版拄頰云:「西山朝來,致有爽氣。」

謝萬北征,常以嘯詠自高,未嘗撫慰眾士。謝公甚器愛萬,而審其必敗,乃俱行。從容謂萬曰:「汝為元帥,宜數喚諸將宴會,以說眾心。」萬從之。因召集諸將,都無所說,直以如意指四坐云:「諸君皆是勁卒。」諸將甚忿恨之。謝公欲深箸恩信,自隊主將帥以下,無不身造,厚相遜謝。及萬事敗,軍中因欲除之。復云:「當為隱士。」故幸而得免。

王子敬兄弟見郗公,躡履問訊,甚脩外生禮。及嘉賓死,皆箸高屐,儀容輕慢。命坐,皆云:「有事,不暇坐。」既去,郗公慨然曰:「使嘉賓不死,鼠輩敢爾!」

王子猷嘗行過吳中,見一士大夫家,極有好竹。主已知子猷當往,乃灑埽施設,在聽事坐相待。王肩輿徑造竹下,諷嘯良久。主已失望,猶冀還當通,遂直欲出門。主人大不堪,便令左右閉門不聽出。王更以此賞主人,乃留坐,盡歡而去。

王子敬自會稽經吳,聞顧辟疆有名園。先不識主人,徑往其家,值顧方集賓友酣燕。而王遊歷既畢,指麾好惡,傍若無人。顧勃然不堪曰:「傲主人,非禮也;以貴驕人,非道也。失此二者,不足齒人,傖耳!」便驅其左右出門。王獨在輿上回轉,顧望左右移時不至,然後令送箸門外,怡然不屑。



\chapter{排調第二十五}

諸葛瑾為豫州,遣別駕到臺,語云:「小兒知談,卿可與語。」連往詣恪,恪不與相見。後於張輔吳坐中相遇,別駕喚恪:「咄咄郎君。」恪因嘲之曰:「豫州亂矣,何咄咄之有?」答曰:「君明臣賢,未聞其亂。」恪曰:「昔唐堯在上,四凶在下。」答曰:「非唯四凶,亦有丹朱。」於是一坐大笑。

晉文帝與二陳共車,過喚鍾會同載,即駛車委去。比出,已遠。既至,因嘲之曰:「與人期行,何以遲遲?望卿遙遙不至。」會答曰:「矯然懿實,何必同群?」帝復問會:「臯繇何如人?」答曰:「上不及堯、舜,下不逮周、孔,亦一時之懿士。」

鍾毓為黃門郎,有機警,在景王坐燕飲。時陳群子玄伯、武周子元夏同在坐,共嘲毓。景王曰:「臯繇何如人?」對曰:「古之懿士。」顧謂玄伯、元夏曰:「君子周而不比,群而不黨。」

嵇、阮、山、劉在竹林酣飲,王戎後往。步兵曰:「俗物已復來敗人意!」王笑曰:「卿輩意,亦復可敗邪?」

晉武帝問孫皓:「聞南人好作爾汝歌,頗能為不?」皓正飲酒,因舉觴勸帝而言曰:「昔與汝為鄰,今與汝為臣。上汝一桮酒,令汝壽萬春。」帝悔之。

孫子荊年少時欲隱,語王武子「當枕石漱流」,誤曰「漱石枕流」。王曰:「流可枕,石可漱乎?」孫曰:「所以枕流,欲洗其耳;所以漱石,欲礪其齒。」

頭責秦子羽云:「子曾不如太原溫顒、潁川荀㝢、范陽張華、士卿劉許、義陽鄒湛、河南鄭詡。此數子者,或謇喫無宮商,或尪陋希言語,或淹伊多姿態,或讙譁少智諝,或口如含膠飴,或頭如巾齏杵。而猶以文采可觀,意思詳序,攀龍附鳳,並登天府。」

王渾與婦鍾氏共坐,見武子從庭過,渾欣然謂婦曰:「生兒如此,足慰人意。」婦笑曰:「若使新婦得配參軍,生兒故可不啻如此!」

荀鳴鶴、陸士龍二人未相識,俱會張茂先坐。張令共語。以其並有大才,可勿作常語。陸舉手曰:「雲間陸士龍。」荀答曰:「日下荀鳴鶴。」陸曰:「既開青雲覩白雉,何不張爾弓,布爾矢?」荀答曰:「本謂雲龍騤騤,定是山鹿野麋。獸弱弩彊,是以發遲。」張乃撫掌大笑。

陸太尉詣王丞相,王公食以酪。陸還遂病。明日與王牋云:「昨食酪小過,通夜委頓。民雖吳人,幾為傖鬼。」

元帝皇子生,普賜群臣。殷洪喬謝曰:「皇子誕育,普天同慶。臣無勳焉,而猥頒厚賚。」中宗笑曰:「此事豈可使卿有勳邪?」

諸葛令、王丞相共爭姓族先後,王曰:「何不言葛、王,而云王、葛?」令曰:「譬言驢馬,不言馬驢,驢寧勝馬邪?」

劉真長始見王丞相,時盛暑之月,丞相以腹熨彈棊局,曰:「何乃渹?」劉既出,人問:「見王公云何?」劉曰:「未見他異,唯聞作吳語耳!」

王公與朝士共飲酒,舉瑠璃盌謂伯仁曰:「此盌腹殊空,謂之寶器,何邪?」答曰:「此盌英英,誠為清徹,所以為寶耳!」

謝幼輿謂周侯曰:「卿類社樹,遠望之,峨峨拂青天;就而視之,其根則群狐所託,下聚溷而已!」答曰:「枝條拂青天,不以為高;群狐亂其下,不以為濁;聚溷之穢,卿之所保,何足自稱?」

王長豫幼便和令,丞相愛恣甚篤。每共圍棊,丞相欲舉行,長豫按指不聽。丞相笑曰:「詎得爾?相與似有瓜葛。」

明帝問周伯仁:「真長何如人?」答曰:「故是千斤犗特。」王公笑其言。伯仁曰:「不如捲角牸,有盤辟之好。」

王丞相枕周伯仁厀,指其腹曰:「卿此中何所有?」答曰:「此中空洞無物,然容卿輩數百人。」

干寶向劉真長敘其搜神記,劉曰:「卿可謂鬼之董狐。」

許文思往顧和許,顧先在帳中眠。許至,便徑就床角枕共語。既而喚顧共行,顧乃命左右取枕上新衣,易己體上所著。許笑曰:「卿乃復有行來衣乎?」

康僧淵目深而鼻高,王丞相每調之。僧淵曰:「鼻者面之山,目者面之淵。山不高則不靈,淵不深則不清。」

何次道往瓦官寺禮拜甚勤。阮思曠語之曰:「卿志大宇宙,勇邁終古。」何曰:「卿今日何故忽見推?」阮曰:「我圖數千戶郡,尚不能得;卿迺圖作佛,不亦大乎!」

庾征西大舉征胡,既成行,止鎮襄陽。殷豫章與書,送一折角如意以調之。庾答書曰:「得所致,雖是敗物,猶欲理而用之。」

桓大司馬乘雪欲獵,先過王、劉諸人許。真長見其裝束單急,問:「老賊欲持此何作?」桓曰:「我若不為此,卿輩亦那得坐談?」

褚季野問孫盛:「卿國史何當成?」孫云:「久應竟,在公無暇,故至今日。」褚曰:「古人『述而不作』,何必在蠶室中?」

謝公在東山,朝命屢降而不動。後出為桓宣武司馬,將發新亭,朝士咸出瞻送。高靈時為中丞,亦往相祖。先時,多少飲酒,因倚如醉,戲曰:「卿屢違朝旨,高臥東山,諸人每相與言:『安石不肯出,將如蒼生何?』今亦蒼生將如卿何?」謝笑而不答。

初,謝安在東山居,布衣,時兄弟已有富貴者,翕集家門,傾動人物。劉夫人戲謂安曰:「大丈夫不當如此乎?」謝乃捉鼻曰:「但恐不免耳!」

支道林因人就深公買印山,深公答曰:「未聞巢、由買山而隱。」

王、劉每不重蔡公。二人嘗詣蔡,語良久,乃問蔡曰:「公自言何如夷甫?」答曰:「身不如夷甫。」王、劉相目而笑曰:「公何處不如?」答曰:「夷甫無君輩客!」

張吳興年八歲,虧齒,先達知其不常,故戲之曰:「君口中何為開狗竇?」張應聲答曰:「正使君輩從此中出入!」

郝隆七月七日出日中仰臥。人問其故?答曰:「我曬書。」

謝公始有東山之志,後嚴命屢臻,勢不獲已,始就桓公司馬。于時人有餉桓公藥草,中有「遠志」。公取以問謝:「此藥又名『小草』,何一物而有二稱?」謝未即答。時郝隆在坐,應聲答曰:「此甚易解:處則為遠志,出則為小草。」謝甚有愧色。桓公目謝而笑曰:「郝參軍此過乃不惡,亦極有會。」

庾園客詣孫監,值行,見齊莊在外,尚幼,而有神意。庾試之曰:「孫安國何在?」即答曰:「庾穉恭家。」庾大笑曰:「諸孫大盛,有兒如此!」又答曰:「未若諸庾之翼翼。」還,語人曰:「我故勝,得重喚奴父名。」

范玄平在簡文坐,談欲屈,引王長史曰:「卿助我。」王曰:「此非拔山力所能助!」

郝隆為桓公南蠻參軍,三月三日會,作詩。不能者,罰酒三升。隆初以不能受罰,既飲,攬筆便作一句云:「娵隅躍清池。」桓問:「娵隅是何物?」答曰:「蠻名魚為娵隅。」桓公曰:「作詩何以作蠻語?」隆曰:「千里投公,始得蠻府參軍,那得不作蠻語也?」

袁羊嘗詣劉恢,恢在內眠未起。袁因作詩調之曰:「角枕粲文茵,錦衾爛長筵。」劉尚晉明帝女,主見詩,不平曰:「袁羊,古之遺狂!」

殷洪遠答孫興公詩云:「聊復放一曲。」劉真長笑其語拙,問曰:「君欲云那放?」殷曰:「㯓臘亦放,何必其鎗鈴邪?」

桓公既廢海西,立簡文,侍中謝公見桓公拜。桓驚笑曰:「安石,卿何事至爾?」謝曰:「未有君拜於前,臣立於後!」

郗重熙與謝公書,道:「王敬仁聞一年少懷問鼎。不知桓公德衰,為復後生可畏?」

張蒼梧是張憑之祖,嘗語憑父曰:「我不如汝。」憑父未解所以。蒼梧曰:「汝有佳兒。」憑時年數歲,歛手曰:「阿翁,詎宜以子戲父?」

習鑿齒、孫興公未相識,同在桓公坐。桓語孫「可與習參軍共語。」孫云:「『蠢爾蠻荊』,敢與大邦為讐?」習云:「『薄伐獫狁』,至于太原。」

桓豹奴是王丹陽外生,形似其舅,桓甚諱之。宣武云:「不恆相似,時似耳!恆似是形,時似是神。」桓逾不說。

王子猷詣謝萬,林公先在坐,瞻矚甚高。王曰:「若林公鬚髮並全,神情當復勝此不?」謝曰:「脣齒相須,不可以偏亡。鬚髮何關於神明?」林公意甚惡。曰:「七尺之軀,今日委君二賢。」

郗司空拜北府,王黃門詣郗門拜,云:「應變將略,非其所長。」驟詠之不已。郗倉謂嘉賓曰:「公今日拜,子猷言語殊不遜,深不可容!」嘉賓曰:「此是陳壽作諸葛評。人以汝家比武侯,復何所言?」

王子猷詣謝公,謝曰:「云何七言詩?」子猷承問,答曰:「昂昂若千里之駒,汎汎若水中之鳧。」

王文度、范榮期俱為簡文所要。范年大而位小,王年小而位大。將前,更相推在前。既移久,王遂在范後。王因謂曰:「簸之揚之,穅秕在前。」范曰:「洮之汰之,沙礫在後。」

劉遵祖少為殷中軍所知,稱之於庾公。庾公甚忻然,便取為佐。既見,坐之獨榻上與語。劉爾日殊不稱,庾小失望,遂名之為「羊公鶴」。昔羊叔子有鶴善舞,嘗向客稱之。客試使驅來,氃氋而不肯舞。故稱比之。

魏長齊雅有體量,而才學非所經。初宦當出,虞存嘲之曰:「與卿約法三章:談者死,文筆者刑,商略抵罪。」魏怡然而笑,無忤於色。

郗嘉賓書與袁虎,道戴安道、謝居士云:「恆任之風,當有所弘耳。」以袁無恆,故以此激之。

范啟與郗嘉賓書曰:「子敬舉體無饒縱,掇皮無餘潤。」郗答曰:「舉體無餘潤,何如舉體非真者?」范性矜假多煩,故嘲之。

二郗奉道,二何奉佛,皆以財賄。謝中郎云:「二郗諂於道,二何佞於佛。」

王文度在西州,與林法師講,韓、孫諸人並在坐。林公理每欲小屈,孫興公曰:「法師今日如著弊絮在荊棘中,觸地挂閡。」

范榮期見郗超俗情不淡,戲之曰:「夷、齊、巢、許,一詣垂名。何必勞神苦形,支策據梧邪?」郗未答。韓康伯曰:「何不使遊刃皆虛?」

簡文在殿上行,右軍與孫興公在後。右軍指簡文語孫曰:「此噉名客!」簡文顧曰:「天下自有利齒兒。」後王光祿作會稽,謝車騎出曲阿祖之。王孝伯罷秘書丞在坐,謝言及此事,因視孝伯曰:「王丞齒似不鈍。」王曰:「不鈍,頗亦驗。」

謝遏夏月嘗仰臥,謝公清晨卒來,不暇著衣,跣出屋外,方躡履問訊。公曰:「汝可謂前倨而後恭。」

顧長康作殷荊州佐,請假還東。爾時例不給布颿,顧苦求之,乃得發。至破冢,遭風大敗。作牋與殷云:「地名破冢,真破冢而出。行人安穩,布颿無恙。」

符朗初過江,王咨議大好事,問中國人物及風土所生,終無極已。朗大患之。次復問奴婢貴賤,朗云:「謹厚有識,中者,乃至十萬;無意為奴婢,問者,止數千耳。」

東府客館是版屋。謝景重詣太傅,時賓客滿中,初不交言,直仰視云:「王乃復西戎其屋。」

顧長康噉甘蔗,恆自尾至本。人問所以,云:「漸至佳境。」

孝武屬王珣求女壻,曰:「王敦、桓溫,磊砢之流,既不可復得,且小如意,亦好豫人家事,酷非所須。正如真長、子敬比,最佳。」珣舉謝混。後袁山松欲擬謝婚,王曰:「卿莫近禁臠。」

桓南郡與殷荊州語次,因共作了語。顧愷之曰:「火燒平原無遺燎。」桓曰:「白布纏棺豎旒旐。」殷曰:「投魚深淵放飛鳥。」次復作危語。桓曰:「矛頭淅米劍頭炊。」殷曰:「百歲老翁攀枯枝。」顧曰:「井上轆轤臥嬰兒。」殷有一參軍在坐,云:「盲人騎瞎馬,夜半臨深池。」殷曰:「咄咄逼人!」仲堪眇目故也。

桓玄出射,有一劉參軍與周參軍朋賭,垂成,唯少一破。劉謂周曰:「卿此起不破,我當撻卿。」周曰:「何至受卿撻!」劉曰:「伯禽之貴,尚不免撻,而況於卿?」周殊無忤色。桓語庾伯鸞曰:劉參軍宜停讀書,周參軍且勤學問。」

桓南郡與道曜講老子,王侍中為主簿在坐。桓曰:「王主簿,可顧名思義。」王未答,且大笑。桓曰:「王思道能作大家兒笑。」

祖廣行恒縮頭。詣桓南郡,始下車,桓曰:「天甚晴朗,祖參軍如從屋漏中來。」

桓玄素輕桓崖,崖在京下有好桃,玄連就求之,遂不得佳者。玄與殷仲文書,以為嗤笑曰:「德之休明,肅慎貢其楛矢;如其不爾,籬壁間物,亦不可得也。」



\chapter{輕詆第二十六}

王太尉問眉子:「汝叔名士,何以不相推重?」眉子曰:「何有名士終日妄語?」

庾元規語周伯仁:「諸人皆以君方樂。」周曰:「何樂?謂樂毅邪?」庾曰:「不爾。樂令耳!」周曰:「何乃刻畫無鹽,以唐突西子也。」

深公云:「人謂庾元規名士,胷中柴棘三斗許。」

庾公權重,足傾王公。庾在石頭,王在冶城坐。大風揚塵,王以扇拂塵曰:「元規塵汙人!」

王右軍少時甚澀訥,在大將軍許,王、庾二公後來,右軍便起欲去。大將軍留之曰:「爾家司空、元規,復可所難?」

王丞相輕蔡公,曰:「我與安期、千里共遊洛水邊,何處聞有蔡充兒?」

褚太傅初渡江,嘗入東,至金昌亭。吳中豪右,燕集亭中。褚公雖素有重名,于時造次不相識別。敕左右多與茗汁,少箸粽,汁盡輒益,使終不得食。褚公飲訖,徐舉手共語云:「褚季野!」於是四座驚散,無不狼狽。

王右軍在南,丞相與書,每嘆子姪不令。云:「虎\ext{㹠}、虎犢,還其所如。」

褚太傅南下,孫長樂於船中視之。言次,及劉真長死,孫流涕,因諷詠曰:「人之云亡,邦國殄瘁。」褚大怒曰:「真長平生,何嘗相比數,而卿今日作此面向人!」孫回泣向褚曰:「卿當念我!」時咸笑其才而性鄙。

謝鎮西書與殷揚州,為真長求會稽。殷答曰:「真長標同伐異,俠之大者。常謂使君降階為甚,乃復為之驅馳邪?」

桓公入洛,過淮、泗,踐北境,與諸僚屬登平乘樓,眺矚中原,慨然曰:「遂使神州陸沈,百年丘墟,王夷甫諸人,不得不任其責!」袁虎率爾對曰:「運自有廢興,豈必諸人之過?」桓公懍然作色,顧謂四坐曰:「諸君頗聞劉景升不?有大牛重千斤,噉芻豆十倍於常牛,負重致遠,曾不若一羸牸。魏武入荊州,烹以饗士卒,于時莫不稱快。」意以況袁。四坐既駭,袁亦失色。

袁虎、伏滔同在桓公府。桓公每遊燕,輒命袁、伏,袁甚恥之,恆嘆曰:「公之厚意,未足以榮國士!與伏滔比肩,亦何辱如之?」

高柔在東,甚為謝仁祖所重。既出,不為王、劉所知。仁祖曰:「近見高柔,大自敷奏,然未有所得。」真長云:「故不可在偏地居,輕在角\ext{䚥}中,為人作議論。」高柔聞之,云:「我就伊無所求。」人有向真長學此言者,真長曰:「我寔亦無可與伊者。」然遊燕猶與諸人書:「可要安固?」安固者,高柔也。

劉尹、江虨、王叔虎、孫興公同坐,江、王有相輕色。虨以手歙叔虎云:「酷吏!」詞色甚彊。劉尹顧謂:「此是瞋邪?非特是醜言聲,拙視瞻。」

孫綽作列仙商丘子贊曰:「所牧何物?殆非真豬。儻遇風雲,為我龍攄。」時人多以為能。王藍田語人云:「近見孫家兒作文,道何物、真豬也。」

桓公欲遷都,以張拓定之業。孫長樂上表,諫此議甚有理。桓見表心服,而忿其為異,令人致意孫云:「君何不尋遂初賦,而彊知人家國事?」

孫長樂兄弟就謝公宿,言至款雜。劉夫人在壁後聽之,具聞其語。謝公明日還,問:「昨客何似?」劉對曰:「亡兄門,未有如此賓客!」謝深有愧色。

簡文與許玄度共語,許云:「舉君、親以為難。」簡文便不復答。許去後而言曰:「玄度故可不至於此!」

謝萬壽春敗後,還,書與王右軍云:「慙負宿顧。」右軍推書曰:「此禹、湯之戒。」

蔡伯喈睹睞笛椽,孫興公聽妓,振且擺折。王右軍聞,大嗔曰:「三祖壽樂器,虺瓦弔,孫家兒打折。」

王中郎與林公絕不相得。王謂林公詭辯,林公道王云:「箸膩顏帢,布單衣,挾左傳,逐鄭康成車後,問是何物塵垢囊!」

孫長樂作王長史誄云:「余與夫子,交非勢利,心猶澄水,同此玄味。」王孝伯見曰:「才士不遜,亡祖何至與此人周旋!」

謝太傅謂子姪曰:「中郎始是獨有千載!」車騎曰:「中郎衿抱未虛,復那得獨有?」

庾道季詫謝公曰:「裴郎云:『謝安謂裴郎乃可不惡,何得為復飲酒?』裴郎又云:『謝安目支道林,如九方臯之相馬,略其玄黃,取其儁逸。』」謝公云:「都無此二語,裴自為此辭耳!」庾意甚不以為好,因陳東亭經酒壚下賦。讀畢,都不下賞裁,直云:「君乃復作裴氏學!」於此語林遂廢。今時有者,皆是先寫,無復謝語。

王北中郎不為林公所知,乃箸論沙門不得為高士論。大略云:「高士必在於縱心調暢,沙門雖云俗外,反更束於教,非情性自得之謂也。」

人問顧長康:「何以不作洛生詠?」答曰:「何至作老婢聲!」

殷顗、庾恆並是謝鎮西外孫。殷少而率悟,庾每不推。嘗俱詣謝公,謝公熟視殷曰:「阿巢故似鎮西。」於是庾下聲語曰:「定何似?」謝公續復云:「巢頰似鎮西。」庾復云:「頰似,足作健不?」

舊目韓康伯:將肘無風骨。

符宏叛來歸國。謝太傅每加接引,宏自以有才,多好上人,坐上無折之者。適王子猷來,太傅使共語。子猷直孰視良久,回語太傅云:「亦復竟不異人!」宏大慚而退。

支道林入東,見王子猷兄弟。還,人問:「見諸王何如?」答曰:「見一群白頸烏,但聞喚啞啞聲。」

王中郎舉許玄度為吏部郎。郗重熙曰:「相王好事,不可使阿訥在坐頭。」

王興道謂:謝望蔡霍霍如失鷹師。

桓南郡每見人不快,輒嗔云:「君得哀家梨,當復不烝食不?」



\chapter{假譎第二十七}

魏武少時,嘗與袁紹好為游俠,觀人新婚,因潛入主人園中,夜叫呼云:「有偷兒賊!」青廬中人皆出觀,魏武乃入,抽刃劫新婦與紹還出,失道,墜枳棘中,紹不能得動,復大叫云:「偷兒在此!」紹遑迫自擲出,遂以俱免。

魏武行役,失汲道,軍皆渴,乃令曰:「前有大梅林,饒子,甘酸,可以解渴。」士卒聞之,口皆出水,乘此得及前源。

魏武常言:「人欲危己,己輒心動。」因語所親小人曰:「汝懷刃密來我側,我必說『心動』。執汝使行刑;汝但勿言其使,無他,當厚相報。」執者信焉,不以為懼,遂斬之。此人至死不知也。左右以為實,謀逆者挫氣矣。

魏武常云:「我眠中不可妄近,近便斫人,亦不自覺,左右宜深慎此!」後陽眠,所幸一人竊以被覆之,因便斫殺。自爾每眠,左右莫敢近者。

袁紹年少時,曾遣人夜以劍擲魏武,少下,不著。魏武揆之,其後來必高,因帖臥床上。劍至果高。

王大將軍既為逆,頓軍姑孰。晉明帝以英武之才,猶相猜憚,乃著戎服,騎巴賨馬,齎一金馬鞭,陰察軍形勢。未至十餘里,有一客姥,居店賣食。帝過愒之,謂姥曰:「王敦舉兵圖逆,猜害忠良,朝廷駭懼,社稷是憂。故劬勞晨夕,用相覘察,恐形迹危露,或致狼狽。追迫之日,姥其匿之。」便與客姥馬鞭而去。行敦營匝而出,軍士覺,曰:「此非常人也!」敦臥心動,曰:「此必黃須鮮卑奴來!」命騎追之,已覺多許里,追士因問向姥:「不見一黃須人騎馬度此邪?」姥曰:「去已久矣,不可復及。」於是騎人息意而反。

王右軍年裁十歲時,大將軍甚愛之,恆置帳中眠。大將軍嘗先出,右軍猶未起;須臾,錢鳳入,屏人論事,都忘右軍在帳中,便言逆節之謀。右軍覺,既聞所論,知無活理,乃剔吐汙頭面被褥,詐熟眠。敦論事造半,方意右軍未起,相與大驚曰:「不得不除之!」及開帳,乃見吐唾從橫,信其實熟眠,於是得全。于時稱其有智。

陶公自上流來,赴蘇峻之難,令誅庾公。謂必戮庾,可以謝峻。庾欲奔竄,則不可;欲會,恐見執,進退無計。溫公勸庾詣陶,曰:「卿但遙拜,必無它。我為卿保之。」庾從溫言詣陶。至,便拜。陶自起止之,曰:「庾元規何緣拜陶士衡?」畢,又降就下坐。陶又自要起同坐。坐定,庾乃引咎責躬,深相遜謝。陶不覺釋然。

溫公喪婦,從姑劉氏,家值亂離散,唯有一女,甚有姿慧,姑以屬公覓婚。公密有自婚意,答云:「佳壻難得,但如嶠比云何?」姑云:「喪敗之餘,乞粗存活,便足慰吾餘年,何敢希汝比?」卻後少日,公報姑云:「已覓得婚處,門地粗可,壻身名宦,盡不減嶠。」因下玉鏡臺一枚。姑大喜。既婚,交禮,女以手披紗扇,撫掌大笑曰:「我固疑是老奴,果如所卜!」玉鏡臺,是公為劉越石長史,北征劉聰所得。

諸葛令女,庾氏婦,既寡,誓云:「不復重出!」此女性甚正彊,無有登車理。恢既許江思玄婚,乃移家近之。初,誑女云:「宜徙。」於是家人一時去,獨留女在後。比其覺,已不復得出。江郎莫來,女哭詈彌甚,積日漸歇。江虨暝入宿,恆在對床上。後觀其意轉帖,虨乃詐厭,良久不悟,聲氣轉急。女乃呼婢云:「喚江郎覺!」江於是躍來就之曰:「我自是天下男子,厭,何預卿事而見喚邪?既爾相關,不得不與人語。」女默然而慙,情義遂篤。

愍度道人始欲過江,與一傖道人為侶,謀曰:「用舊義往江東,恐不辦得食。」便共立「心無義」。既而此道人不成渡,愍度果講義積年。後有傖人來,先道人寄語云:「為我致意愍度,無義那可立?治此計,權救饑爾!無為遂負如來也。」

王文度弟阿智,惡乃不翅,當年長而無人與婚。孫興公有一女,亦僻錯,又無嫁娶理。因詣文度,求見阿智。既見,便陽言:「此定可,殊不如人所傳,那得至今未有婚處?我有一女,乃不惡,但吾寒士,不宜與卿計,欲令阿智娶之。」文度欣然而啟藍田云:「興公向來,忽言欲與阿智婚。」藍田驚喜。既成婚,女之頑嚚,欲過阿智。方知興公之詐。

范玄平為人,好用智數,而有時以多數失會。嘗失官居東陽,桓大司馬在南州,故往投之。桓時方欲招起屈滯,以傾朝廷;且玄平在京,素亦有譽,桓謂遠來投己,喜躍非常。比入至庭,傾身引望,語笑歡甚。顧謂袁虎曰:「范公且可作太常卿。」范裁坐,桓便謝其遠來意。范雖實投桓,而恐以趨時損名,乃曰:「雖懷朝宗,會有亡兒瘞在此,故來省視。」桓悵然失望,向之虛佇,一時都盡。

謝遏年少時,好著紫羅香囊,垂覆手。太傅患之,而不欲傷其意,乃譎與賭,得即燒之。



\chapter{黜免第二十八}

諸葛厷在西朝,少有清譽,為王夷甫所重,時論亦以擬王。後為繼母族黨所讒,誣之為狂逆。將遠徙,友人王夷甫之徒,詣檻車與別。厷問:「朝廷何以徙我?」王曰:「言卿狂逆。」厷曰:「逆則應殺,狂何所徙?」

桓公入蜀,至三峽中,部伍中有得猨子者。其母緣岸哀號,行百餘里不去,遂跳上船,至便即絕。破視其腹中,腸皆寸寸斷。公聞之,怒,命黜其人。

殷中軍被廢,在信安,終日恆書空作字。揚州吏民尋義逐之,竊視,唯作「咄咄怪事」四字而已。

桓公坐有參軍椅烝薤不時解,共食者又不助,而椅終不放,舉坐皆笑。桓公曰:「同盤尚不相助,況復危難乎?」敕令免官。

殷中軍廢後,恨簡文曰:「上人著百尺樓上,儋梯將去。」

鄧竟陵免官後赴山陵,過見大司馬桓公。公問之曰:「卿何以更瘦?」鄧曰:「有愧於叔達,不能不恨於破甑!」

桓宣武既廢太宰父子,仍上表曰:「應割近情,以存遠計。若除太宰父子,可無後憂。」簡文手答表曰:「所不忍言,況過於言?」宣武又重表,辭轉苦切。簡文更答曰:「若晉室靈長,明公便宜奉行此詔。如大運去矣,請避賢路!」桓公讀詔,手戰流汗,於此乃止。太宰父子,遠徙新安。

桓玄敗後,殷仲文還為大司馬咨議,意似二三,非復往日。大司馬府聽前,有一老槐,甚扶疎。殷因月朔,與眾在聽,視槐良久,嘆曰:「槐樹婆娑,無復生意!」

殷仲文既素有名望,自謂必當阿衡朝政。忽作東陽太守,意甚不平。及之郡,至富陽,慨然嘆曰:「看此山川形勢,當復出一孫伯符!」



\chapter{儉嗇第二十九}

和嶠性至儉,家有好李,王武子求之,與不過數十。王武子因其上直,率將少年能食之者,持斧詣園,飽共噉畢,伐之,送一車枝與和公。問曰:「何如君李?」和既得,唯笑而已。

王戎儉吝,其從子婚,與一單衣,後更責之。

司徒王戎,既貴且富,區宅僮牧,膏田水碓之屬,洛下無比。契疏鞅掌,每與夫人燭下散籌算計。

王戎有好李,賣之,恐人得其種,恆鑽其核。

王戎女適裴頠,貸錢數萬。女歸,戎色不說。女遽還錢,乃釋然。

衛江州在尋陽,有知舊人投之,都不料理,唯餉「王不留行」一斤。此人得餉,便命駕。李弘範聞之曰:「家舅刻薄,乃復驅使草木。」

王丞相儉節,帳下甘果,盈溢不散。涉春爛敗,都督白之,公令舍去。曰:「慎不可令大郎知。」

蘇峻之亂,庾太尉南奔見陶公。陶公雅相賞重。陶性儉吝,及食,噉薤,庾因留白。陶問:「用此何為?」庾云:「故可種。」於是大嘆庾非唯風流,兼有治實。

郗公大聚歛,有錢數千萬。嘉賓意甚不同,常朝旦問訊。郗家法:子弟不坐。因倚語移時,遂及財貨事。郗公曰:「汝正當欲得吾錢耳!」迺開庫一日,令任意用。郗公始正謂損數百萬許。嘉賓遂一日乞與親友,周旋略盡。郗公聞之,驚怪不能已。



\chapter{汰侈第三十}

石崇每要客燕集,常令美人行酒,客飲酒不盡者,使黃門交斬美人。王丞相與大將軍嘗共詣崇,丞相素不能飲,輒自勉彊,至於沉醉。每至大將軍,固不飲,以觀其變。已斬三人,顏色如故,尚不肯飲。丞相讓之,大將軍曰:「自殺伊家人,何預卿事!」

石崇廁,常有十餘婢侍列,皆麗服藻飾。置甲煎粉、沈香汁之屬,無不畢備。又與新衣著令出,客多羞不能如廁。王大將軍往,脫故衣,著新衣,神色傲然。羣婢相謂曰:「此客必能作賊。」

武帝嘗降王武子家,武子供饌,並用瑠璃器。婢子百餘人,皆綾羅絝\ext{𧟌},以手擎飲食。烝\ext{㹠}肥美,異於常味。帝怪而問之,答曰:「以人乳飲\ext{㹠}。」帝甚不平,食未畢,便去。王、石所未知作。

王君夫以\ext{𥹋}糒澳釜,石季倫用蠟燭作炊。君夫作紫絲布步障碧綾裹四十里,石崇作錦步障五十里以敵之。石以椒為泥,王以赤石脂泥壁。

石崇為客作豆粥,咄嗟便辦。恆冬天得韭蓱\ext{䪢}。又牛形狀氣力不勝王愷牛,而與愷出遊,極晚發,爭入洛城,崇牛數十步後,迅若飛禽,愷牛絕走不能及。每以此三事為搤腕。乃密貨崇帳下都督及御車人,問所以。都督曰:「豆至難煮,唯豫作熟末,客至,作白粥以投之。韭蓱\ext{䪢}是搗韭根,雜以麥苗爾。」復問馭人牛所以駛。馭人云:「牛本不遲,由將車人不及制之爾。急時聽偏轅,則駛矣。」愷悉從之,遂爭長。石崇後聞,皆殺告者。

王君夫有牛,名「八百里駮」,常瑩其蹄角。王武子語君夫:「我射不如卿,今指賭卿牛,以千萬對之。」君夫既恃手快,且謂駿物無有殺理,便相然可。令武子先射。武子一起便破的,卻據胡床,叱左右:「速探牛心來!」須臾,炙至,一臠便去。

王君夫嘗責一人無服餘衵,因直內著曲閤重閨裏,不聽人將出。遂饑經日,迷不知何處去。後因緣相為垂死,迺得出。

石崇與王愷爭豪,並窮綺麗,以飾輿服。武帝,愷之甥也,每助愷。嘗以一珊瑚樹,高二尺許賜愷。枝柯扶疎,世罕其比。愷以示崇。崇視訖,以鐵如意擊之,應手而碎。愷既惋惜,又以為疾己之寶,聲色甚厲。崇曰:「不足恨,今還卿。」乃命左右悉取珊瑚樹,有三尺四尺,條榦絕世,光彩溢目者六七枚,如愷許比甚眾。愷惘然自失。

王武子被責,移第北邙下。于時人多地貴,濟好馬射,買地作埒,編錢匝地竟埒。時人號曰「金溝」。

石崇每與王敦入學戲,見顏、原象而嘆曰:「若與同升孔堂,去人何必有間!」王曰:「不知餘人云何?子貢去卿差近。」石正色云:「士當令身名俱泰,何至以甕牖語人!」

彭城王有快牛,至愛惜之。王太尉與射,賭得之。彭城王曰:「君欲自乘則不論;若欲噉者,當以二十肥者代之。既不廢噉,又存所愛。」王遂殺噉。

王右軍少時,在周侯末坐,割牛心噉之。於此改觀。



\chapter{忿狷第三十一}

魏武有一妓,聲最清高,而情性酷惡。欲殺則愛才,欲置則不堪。於是選百人一時俱教。少時,果有一人聲及之,便殺惡性者。

王藍田性急。嘗食雞子,以筯刺之,不得,便大怒,舉以擲地。雞子於地圓轉未止,仍下地以屐齒蹍之,又不得,瞋甚,復於地取內口中,齧破即吐之。王右軍聞而大笑曰:「使安期有此性,猶當無一豪可論,況藍田邪?」

王司州嘗乘雪往王螭許。司州言氣少有牾逆於螭,便作色不夷。司州覺惡,便輿床就之,持其臂曰:「汝詎復足與老兄計?」螭撥其手曰:「冷如鬼手馨,彊來捉人臂!」

桓宣武與袁彥道樗蒱,袁彥道齒不合,遂厲色擲去五木。溫太真云:「見袁生遷怒,知顏子為貴。」

謝無奕性麤彊。以事不相得,自往數王藍田,肆言極罵。王正色面壁不敢動,半日。謝去良久,轉頭問左右小吏曰:「去未?」答云:「已去。」然後復坐。時人嘆其性急而能有所容。

王令詣謝公,值習鑿齒已在坐,當與併榻。王徙倚不坐,公引之與對榻。去後,語胡兒曰:「子敬實自清立,但人為爾多矜咳,殊足損其自然。」

王大、王恭嘗俱在何僕射坐。恭時為丹陽尹,大始拜荊州。訖將乖之際,大勸恭酒。恭不為飲,大逼彊之,轉苦,便各以帬帶繞手。恭府近千人,悉呼入齋,大左右雖少,亦命前,意便欲相殺。射無計,因起排坐二人之間,方得分散。所謂勢利之交,古人羞之。

桓南郡小兒時,與諸從兄弟各養鵝共鬥。南郡鵝每不如,甚以為忿;迺夜往鵝欄間,取諸兄弟鵝悉殺之。既曉,家人咸以驚駭,云是變怪,以白車騎。車騎曰:「無所致怪,當是南郡戲耳!」問,果如之。



\chapter{讒險第三十二}

王平子形甚散朗,內實勁俠。

袁悅有口才,能短長說,亦有精理。始作謝玄參軍,頗被禮遇。後丁艱,服除還都,唯齎戰國策而已。語人曰:「少年時讀論語、老子,又看莊、易,此皆是病痛事,當何所益邪?天下要物,正有戰國策。」既下,說司馬孝文王,大見親待,幾亂機軸。俄而見誅。

孝武甚親敬王國寶、王雅。雅薦王珣於帝,帝欲見之。嘗夜與國寶、雅相對,帝微有酒色,令喚珣。垂至,已聞卒傳聲,國寶自知才出珣下,恐傾奪其寵,因曰:「王珣當今名流,陛下不宜有酒色見之,自可別詔也。」帝然其言,心以為忠,遂不見珣。

王緒數讒殷荊州於王國寶,殷甚患之,求術於王東亭。曰:「卿但數詣王緒,往輒屏人,因論它事,如此,則二王之好離矣。」殷從之。國寶見王緒問曰:「比與仲堪屏人何所道?」緒云:「故是常往來,無它所論。」國寶謂緒於己有隱,果情好日疎,讒言以息。



\chapter{尤悔第三十三}

魏文帝忌弟任城王驍壯。因在卞太后閤共圍棊,並噉棗,文帝以毒置諸棗蔕中。自選可食者而進,王弗悟,遂雜進之。既中毒,太后索水救之。帝預敕左右毀缾罐,太后徒跣趨井,無以汲。須臾,遂卒。復欲害東阿,太后曰:「汝已殺我任城,不得復殺我東阿。」

王渾後妻,琅邪顏氏女。王時為徐州刺史,交禮拜訖,王將答拜,觀者咸曰:「王侯州將,新婦州民,恐無由答拜。」王乃止。武子以其父不答拜,不成禮,恐非夫婦;不為之拜,謂為顏妾。顏氏恥之。以其門貴,終不敢離。

陸平原河橋敗,為盧志所讒,被誅。臨刑嘆曰:「欲聞華亭鶴唳,可復得乎!」

劉琨善能招延,而拙於撫御。一日雖有數千人歸投,其逃散而去亦復如此。所以卒無所建。

王平子始下,丞相語大將軍:「不可復使羌人東行。」平子面似羌。

王大將軍起事,丞相兄弟詣闕謝。周侯深憂諸王,始入,甚有憂色。丞相呼周侯曰:「百口委卿!」周直過不應。既入,苦相存救。既釋,周大說,飲酒。及出,諸王故在門。周曰:「今年殺諸賊奴,當取金印如斗大繫肘後。」大將軍至石頭,問丞相曰:「周侯可為三公不?」丞相不答。又問:「可為尚書令不?」又不應。因云:「如此,唯當殺之耳!」復默然。逮周侯被害,丞相後知周侯救己,嘆曰:「我不殺周侯,周侯由我而死。幽冥中負此人!」

王導、溫嶠俱見明帝,帝問溫前世所以得天下之由。溫未答。頃,王曰:「溫嶠年少未諳,臣為陛下陳之。」王迺具敘宣王創業之始,誅夷名族,寵樹同己。及文王之末,高貴鄉公事。明帝聞之,覆面著床曰:「若如公言,祚安得長!」

王大將軍於眾坐中曰:「諸周由來未有作三公者。」有人答曰:「唯周侯邑五馬領頭而不克。」大將軍曰:「我與周洛下相遇,一面頓盡。值世紛紜,遂至於此!」因為流涕。

溫公初受劉司空使勸進,母崔氏固駐之,嶠絕裾而去。迄於崇貴,鄉品猶不過也。每爵皆發詔。

庾公欲起周子南,子南執辭愈固。庾每詣周,庾從南門入,周從後門出。庾嘗一往奄至,周不及去,相對終日。庾從周索食,周出蔬食,庾亦彊飯,極歡;并語世故,約相推引,同佐世之任。既仕,至將軍二千石,而不稱意。中宵慨然曰:「大丈夫乃為庾元規所賣!」一嘆,遂發背而卒。

阮思曠奉大法,敬信甚至。大兒年未弱冠,忽被篤疾。兒既是偏所愛重,為之祈請三寶,晝夜不懈。謂至誠有感者,必當蒙祐。而兒遂不濟。於是結恨釋氏,宿命都除。

桓宣武對簡文帝,不甚得語。廢海西後,宜自申敘,乃豫撰數百語,陳廢立之意。既見簡文,簡文便泣下數十行。宣武矜愧,不得一言。

桓公臥語曰:「作此寂寂,將為文、景所笑!」既而屈起坐曰:「既不能流芳後世,亦不足復遺臭萬載邪?」

謝太傅於東船行,小人引船,或遲或速,或停或待,又放船從橫,撞人觸岸。公初不呵譴。人謂公常無嗔喜。曾送兄征西葬還,日莫雨駛,小人皆醉,不可處分。公乃於車中,手取車柱撞馭人,聲色甚厲。夫以水性沈柔,入隘奔激。方之人情,固知迫隘之地,無得保其夷粹。

簡文見田稻不識,問是何草?左右答是稻。簡文還,三日不出,云:「寧有賴其末,而不識其本?」

桓車騎在上明畋獵。東信至,傳淮上大捷。語左右云:「群謝年少,大破賊。」因發病薨。談者以為此死,賢於讓揚之荊。

桓公初報破殷荊州,曾講論語,至「富與貴,是人之所欲,不以其道得之,不處」。玄意色甚惡。



\chapter{紕漏第三十四}

王敦初尚主,如廁,見漆箱盛乾棗,本以塞鼻,王謂廁上亦下果,食遂至盡。既還,婢擎金澡盤盛水,瑠璃盌盛澡豆;因倒箸水中而飲之,謂是乾飯。群婢莫不掩口而笑之。

元皇初見賀司空,言及吳時事,問:「孫皓燒鋸截一賀頭,是誰?」司空未得言,元皇自憶曰:「是賀劭。」司空流涕曰:「臣父遭遇無道,創巨痛深,無以仰答明詔。」元皇愧慙,三日不出。

蔡司徒渡江,見彭蜞,大喜曰:「蟹有八足,加以二螯。」令烹之。既食,吐下委頓,方知非蟹。後向謝仁祖說此事,謝曰:「卿讀爾雅不熟,幾為勸學死。」

任育長年少時,甚有令名。武帝崩,選百二十挽郎,一時之秀彥,育長亦在其中。王安豐選女壻,從挽郎搜其勝者,且擇取四人,任猶在其中。童少時神明可愛,時人謂育長影亦好。自過江,便失志。王丞相請先度時賢共至石頭迎之,猶作疇日相待,一見便覺有異。坐席竟,下飲,便問人云:「此為茶?為茗?」覺有異色,乃自申明云:「向問飲為熱,為冷耳。」嘗行從棺邸下度,流涕悲哀。王丞相聞之曰:「此是有情癡。」

謝虎子嘗上屋熏鼠。胡兒既無由知父為此事,聞人道「癡人有作此者」。戲笑之。時道此非復一過。太傅既了己之不知,因其言次,語胡兒曰:「世人以此謗中郎,亦言我共作此。」胡兒懊熱,一月日閉齋不出。太傅虛託引己之過,以相開悟,可謂德教。

殷仲堪父病虛悸,聞床下蟻動,謂是牛鬬。孝武不知是殷公,問仲堪「有一殷,病如此不?」仲堪流涕而起曰:「臣進退唯谷。」

虞嘯父為孝武侍中,帝從容問曰:「卿在門下,初不聞有所獻替。」虞家富春,近海,謂帝望其意氣,對曰:「天時尚煗,\ext{䱥}魚蝦\ext{䱹}未可致,尋當有所上獻。」帝撫掌大笑。

王大喪後,朝論或云「國寶應作荊州」。國寶主簿夜函白事,云:「荊州事已行。」國寶大喜,而夜開閤,喚綱紀話勢,雖不及作荊州,而意色甚恬。曉遣參問,都無此事。即喚主簿數之曰:「卿何以誤人事邪?」



\chapter{惑溺第三十五}

魏甄后惠而有色,先為袁熙妻,甚獲寵。曹公之屠鄴也,令疾召甄,左右白:「五官中郎已將去。」公曰:「今年破賊正為奴。」

荀奉倩與婦至篤,冬月婦病熱,乃出中庭自取冷,還以身熨之。婦亡,奉倩後少時亦卒。以是獲譏於世。奉倩曰:「婦人德不足稱,當以色為主。」裴令聞之曰:「此乃是興到之事,非盛德言,冀後人未昧此語。」

賈公閭後妻郭氏酷妒,有男兒名黎民,生載周,充自外還,乳母抱兒在中庭,兒見充喜踊,充就乳母手中嗚之。郭遙望見,謂充愛乳母,即殺之。兒悲思啼泣,不飲它乳,遂死。郭後終無子。

孫秀降晉,晉武帝厚存寵之,妻以姨妹蒯氏,室家甚篤。妻嘗妒,乃罵秀為「貉子」。秀大不平,遂不復入。蒯氏大自悔責,請救於帝。時大赦,群臣咸見。既出,帝獨留秀,從容謂曰:「天下曠蕩,蒯夫人可得從其例不?」秀免冠而謝,遂為夫婦如初。

韓壽美姿容,賈充辟以為掾。充每聚會,賈女於青璅中看,見壽,說之。恆懷存想,發於吟詠。後婢往壽家,具述如此,并言女光麗。壽聞之心動,遂請婢潛修音問。及期往宿。壽蹻捷絕人,踰牆而入,家中莫知。自是充覺女盛自拂拭,說暢有異於常。後會諸吏,聞壽有奇香之氣,是外國所貢,一著人,則歷月不歇。充計武帝唯賜己及陳騫,餘家無此香,疑壽與女通,而垣牆重密,門閤急峻,何由得爾?乃託言有盜,令人修牆。使反曰:「其餘無異,唯東北角如有人跡。而牆高,非人所踰。」充乃取女左右婢考問,即以狀對。充秘之,以女妻壽。

王安豐婦,常卿安豐。安豐曰:「婦人卿婿,於禮為不敬,後勿復爾。」婦曰:「親卿愛卿,是以卿卿;我不卿卿,誰當卿卿?」遂恆聽之。

王丞相有幸妾姓雷,頗預政事納貨。蔡公謂之「雷尚書」。



\chapter{仇隟第三十六}

孫秀既恨石崇不與綠珠,又憾潘岳昔遇之不以禮。後秀為中書令,岳省內見之,因喚曰:「孫令,憶疇昔周旋不?」秀曰:「中心藏之,何日忘之?」岳於是始知必不免。後收石崇、歐陽堅石,同日收岳。石先送市,亦不相知。潘後至,石謂潘曰:「安仁,卿亦復爾邪?」潘曰:「可謂『白首同所歸』。」潘金谷集詩云:「投分寄石友,白首同所歸。」乃成其讖。

劉璵兄弟少時為王愷所憎,嘗召二人宿,欲默除之。令作阬,阬畢,垂加害矣。石崇素與璵、琨善,聞就愷宿,知當有變,便夜往詣愷,問二劉所在?愷卒迫不得諱,答云:「在後齋中眠。」石便徑入,自牽出,同車而去。語曰:「少年,何以輕就人宿?」

王大將軍執司馬愍王,夜遣世將載王於車而殺之,當時不盡知也。雖愍王家,亦未之皆悉,而無忌兄弟皆稺。王胡之與無忌,長甚相暱,胡之嘗共遊,無忌入告母,請為饌。母流涕曰:「王敦昔肆酷汝父,假手世將。吾所以積年不告汝者,王氏門彊,汝兄弟尚幼,不欲使此聲著,蓋以避禍耳!」無忌驚號,抽刃而出,胡之去已遠。

應鎮南作荊州,王脩載、譙王子無忌同至新亭與別,坐上賓甚多,不悟二人俱到。有一客道:「譙王丞致禍,非大將軍意,正是平南所為耳。」無忌因奪直兵參軍刀,便欲斫。脩載走投水,舸上人接取,得免。

王右軍素輕藍田,藍田晚節論譽轉重,右軍尤不平。藍田於會稽丁艱,停山陰治喪。右軍代為郡,屢言出弔,連日不果。後詣門自通,主人既哭,不前而去,以陵辱之。於是彼此嫌隟大構。後藍田臨揚州,右軍尚在郡,初得消息,遣一參軍詣朝廷,求分會稽為越州,使人受意失旨,大為時賢所笑。藍田密令從事數其郡諸不法,以先有隟,令自為其宜。右軍遂稱疾去郡,以憤慨致終。

王東亭與孝伯語,後漸異。孝伯謂東亭曰:「卿便不可復測!」答曰:「王陵廷爭,陳平從默,但問克終云何耳。」

王孝伯死,縣其首於大桁。司馬太傅命駕出至標所,孰視首,曰:「卿何故趣,欲殺我邪?」

桓玄將篡,桓脩欲因玄在脩母許襲之。庾夫人云:「汝等近,過我餘年,我養之,不忍見行此事。」



